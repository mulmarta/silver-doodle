% !TEX root = main.tex
% !TeX spellcheck = en_US
\section{Details of The \saik Protocol}\label{sec:saik-details}
In this section, we formally describe the \saik protocol. \saik inherits most of its mechanisms from \protITK, the CGKA
of \mls. The major difference between \saik and \protITK is the use of \mmPKE and signing only the tag. There are also
some smaller differences, such as the use of $q$-ary\dnote{Do we still do this?} trees. Another small difference is that \protITK ensures agreement on the ``transcript hash'', binding all past messages, while \saik ensures agreement only on all past group modifications. Indeed, a transcript hash does not make sense for saCGKA, since parties see different messages.

\subsection{Authenticated Key Service (AKS)}
\saik relies on an Authenticated Key Service (AKS) which authentically distributes so-called key packages (also called key bundles or pre-keys) used to add new members to the group without interacting with them. A key package should only be used once. For simplicity, we use an idealized AKS which guarantees that a fresh, authentic and honestly generated key package of any party is always available to any other party.

Formally, the AKS is modelled as the functionality $\funcPKI$ defined in \cref{fig:aks}. \saik works in the $\funcPK$-hybrids model. This means that $\funcPKI$ is available in the real world and emulated by the simulator in the ideal world.
$\funcPKI$ works as follows. When a party $\id$ wants to fetch a key package of another party $\id'$, $\funcPKI$ generates a new key package for $\id'$ using \saik's key-package generation algorithm (formally, the algorithm is a parameter of $\funcPKI$). It sends (the public part of) the package to $\id$ and to the adversary. Note that since $\funcPKI$ exists in the real world, the adversary should be thought of as the environment. The secrets for the key package can be fetched by $\id_t$ later, when it decides to join the group. Once fetched, secrets are deleted, which means that $\funcPKI$ cannot be used as secure storage.

\begin{figure}[!tbp]
	\vspace*{-1.5em}\begin{systembox}{$\funcPKI$}
		\begin{flushleft}
			\mbox{Parameter: key-package generation algorithm $\genPKIkeys$.}

			\hrulefill
			\vspace*{-0.5em}
		\end{flushleft}
			\begin{minipage}[t]{0.49\linewidth}
				{\bf Initialization}
				\begin{algorithmic}
					\State $\mathsf{SK}[\cdot, \cdot] \gets \bot$
				\end{algorithmic}


        {\bf Input $\keyword{GetSK}(PK)$ from $\id$}
				\begin{algorithmic}
          \State $SK \gets \mathsf{SK}[\id,PK]$
          \State $\mathsf{SK}[\id,PK] \gets \bot$
          \State \Return $SK$
				\end{algorithmic}
			\end{minipage}
			\hfill
			\begin{minipage}[t]{0.49\linewidth}
				{\bf Input $(\keyword{GetPK},\id')$ from $\id$}
				\begin{algorithmic}
          \State $(PK, SK) \gets \genPKIkeys()$
					\State $\mathsf{SK}[\id',PK] \gets SK$
					\State Send $(\id', PK)$ to adv.
					\State \Return $PK$
				\end{algorithmic}
			\end{minipage}
	\end{systembox}

	\caption{The Authenticated Key service Functionality.\vspace{0.5cm}}
	\label{fig:aks}
\end{figure}

To conclude, we mention the most important aspects in which $\funcPKI$ differs from a more realistic AKS. First, in a typical implementation of an AKS, parties generate key packages themselves and upload them to an untrusted server, authenticated with long-term so-called identity keys. This means that a realistic attacker model is one where parties can be corrupted before they join, in which case the secrets for their key packages and long-term keys leak. This allows an active adversary to inject arbitrary key packages on their behalf. Such abilities are not considered in our model. However, we stress that we do consider attacks where the adversary injects on behalf of current group members messages that add parties with arbitrary key packages.

Further, $\funcPKI$ identifies key packages by public keys. Looking ahead, this means that a party adding $\id'$ has to send the whole public part of the package so that $\id'$ can identify it when it joins. In reality, this would be implemented by hashes.

\subsection{Ratchet Trees}
% !TEX root = main.tex
% !TeX spellcheck = en_US
A ratchet tree is a left-balanced $q$-ary tree (a formal definition can be found in \cref{sec:addleafprf}). This generalizes \protITK' binary trees. Using $q\neq2$ can be beneficial in certain situations.
A ratchet tree, as well as its nodes, have a number of labels listed in \cref{tab:node_labels}.
We also define a number of helper methods in \cref{tab:node_helpers}.
%
\begin{table*}[!t]
  \begin{minipage}[t]{.48\textwidth}
  	\begin{tabularx}{\textwidth}{| l | X |}
      \hline
    	$\tree.\rt$ & The root.\\
%    	\hline
%    	$\tree.\nodes$ & The set of all nodes in $\tree$.\\
    	\hline
    	$v.\isroot$ & True iff $v = \tree.\rt$.\\
    	\hline
    	$v.\isleaf$ & True iff $v$ has no children.\\
    	\hline
    	$v.\parent$ & The parent node of $v$ (or $\bot$ if $v.\isroot$).\\
    	\hline
    	$v.\children$ & If $\neg v.\isleaf$: ordered list of $v$'s children.\\
    	\hline
    	$v.\nodeIndex$ & The node index of $v$.\\
      \hline
      $v.\depth$ & The length of the path from $v$ to $\tree.\rt$. \\
  		\hline
  		$v.\mmpkepk$ & An mmPKE encryption key.\\
  		\hline
  		$v.\mmpkesk$ & The corresponding decryption key.\\
  		\hline
      $v.\rsvk$ & If $v.\isleaf$: a signature verification key.\\
  		\hline
      $v.\rssk$ & If $v.\isleaf$: the corresponding signing key.\\
  		\hline
  		$v.\unmergedLeaves$ & The set of indices of the leaves below $v$ whose owner $\id$ does not know $v.\pkesk$.\\
  		\hline
  		$v.\id$ & If $v.\isleaf$: the $\id$ associated with that leaf.\\
  		\hline
  	\end{tabularx}

	\caption{Labels of a ratchet-tree $\tree$ and its nodes.}
	\label{tab:node_labels}
  \end{minipage}
  \hfill
  \begin{minipage}[t]{.48\textwidth}
  	\begin{tabularx}{\textwidth}{| l | X |}
  		\hline
  		$\itkSt.\groupId$ & The identifier of the group.\\
  		\hline
  		$\itkSt.\tree$ & The ratchet tree.\\
  		\hline
  		$\itkSt.\leaf$ & The party's leaf in $\tree$.\\
  		\hline
  		$\itkSt.\treeHash$ & A hash of the public part of $\tree$.\\
  		\hline
      $\itkSt.\lastAct$ & The last modification of the group state and the user who initiated it.\\
  		\hline
  		$\itkSt.\applicationSecret$ & The current epoch's CGKA key. Exposed to the application layer.\\
  		\hline
  		$\itkSt.\initSecret$ & The next epoch's init secret.\\
  		\hline
      $\itkSt.\membershipKey$ & The next epoch's membership secret for authenticating messages.\\
  		\hline
  		$\itkSt.\groupContext()$ & Returns $(\itkSt.\groupId, \itkSt.\treeHash, \itkSt.\lastAct)$.\\
  		\hline
        $\itkSt.\confTag$ & The confirmation tag, which is signed to ensure authenticity.\\
  		\hline
  	\end{tabularx}
    \caption{The protocol state of a party $\id$ and the helper method for computing the context.}
    \label{tab:prot-state}
  \end{minipage}
  \begin{minipage}{.48\textwidth}
    \begin{tabularx}{\textwidth}{| l | X |}
      \hline
      $\pathSecret$ & The path secrets $s_2,\ldots,s_n$ used to derive the keypairs in each node. Sent via the \mmPKE
                      encryption to keep tree invariant intact.\\
      \hline
      $\commitSecret$ & The path secret in the root node. Used as seed for the key schedule together with \initSecret
                        from the previous epoch\\
      \hline
      $\joinerSecret$ & Secret sent to new group members. Together
                        with the group context, enables computation of the $\epochSecret$.\\
      \hline
      $\epochSecret$ & Base secret used to derive all other secrets, i.e. \applicationSecret,
                       \membershipKey,\initSecret, \confTag \\
      \hline
    \end{tabularx}
    \caption{Intermediate values computed by the protocol that are not part of the state.}
    \label{tab:prot-intermediate-state}
  \end{minipage}
  
%  \end{table*}
%\begin{table*}[!t]
  \begin{minipage}{\textwidth}
  	\begin{tabularx}{.48\textwidth}{| l | X |}
  		\hline
  		$\tree.\clone()$ & Returns a copy of $\tree$.\\
  		\hline
  		$\tree.\public()$ & Returns a copy of $\tree$ with all labels $v.\rssk$ and $v.\pkesk$ set to $\bot$.\\
  		\hline
  		$\tree.\roster()$ & Returns $\id$'s of all parties in $\tree$.\\
  		\hline
  		$\tree.\leaves()$ & Returns the list of all leaves in the tree, sorted from left to right.\\
  		\hline
  		$\tree.\leafof(\id)$ & Returns the leaf $v$ with $v.\id = \id$.\\
  		\hline
  		$\tree.\getleaf()$ & Returns leftmost $v$ s.t. $\neg v.\inuse()$. If no such $v$ exists, adds a new leaf using $\addleaf(\tree)$ and returns it.\\
      \hline
      $\tree.\blankpath(v)$ & For all $u\in\tree.\directpath(v)$ calls $u.\blank()$. \\
      \hline
  		$\tree.\inSubtree(u,v)$ & Returns true if $u$ is in $v$'s subtree.\\
  		\hline
  		$v.\inuse()$ & Returns $\false$ iff all labels are $\bot$.\\
  		\hline
  		$v.\blank()$ & Sets all labels of $v$ to $\bot$.\\
  		\hline
  	\end{tabularx}
    \hfill
    \begin{tabularx}{.48\textwidth}{| l | X |}
  		\hline
  		$\tree.\lca(u, v)$ & Returns the lowest common ancestor of the two leafs.\\
  		\hline
  		$\tree.\directpath(v)$ & Returns the path from $v$'s parent to the root.\\
  		\hline
  		$\tree.\mergeleaves(v)$ & Sets $u.\unmergedLeaves \gets \emptyset$ for all $u \in
                                    \tree.\directpath(v)$\\
  		\hline
  		$\tree.\unmergeleaf(v)$ & Sets $u.\unmergedLeaves \setadd v$ for all $u$ returned by $\tree.\directpath(v)$\\
  		\hline
      $v.\resolution()$ &
      If $v.\inuse$, return $(v) \append v.\unmergedLeaves$. Else if $v.\isleaf$, return $()$. Else, return $v.\children[1].\resolution() $ $\append \dots \append v.\children[n].\resolution()$\\
%  		Return
%  		$\begin{cases}
%  			(v) \append v.\unmergedLeaves & \text{if } v.\inuse() \\
%        \term{concatChildResolution}(v) & \text{else if } \neg v.\isleaf \\
%  			\emptylist & \text{else},
%  		\end{cases}$\\
%      &where $\term{concatChildResolution}(v) = v.\children[1].\resolution() \append \dots \append v.\children[n].\resolution()$.\\
  		\hline
  		$v.\resolvent(u)$ & Returns the ancestor of $u$ in $v.\resolution() \setminus (v)$ (or $\bot$ if $u$ is not a descendant of $v$).\\
  		\hline
  	\end{tabularx}
	  \caption{Helper methods for a ratchet tree $\tree$ and its nodes.\vspace{-1em}}
	  \label{tab:node_helpers}
  \end{minipage}
\end{table*}

%%% Local Variables:
%%% mode: latex
%%% TeX-master: "main"
%%% End:


Importantly, the \emph{direct path} of a leaf $u$ consists of (the ordered list of) all nodes on the path from $u$ to the root, without $u$.
The \emph{resolution} of a node $v$ is the minimal set of descendant non-blank nodes that covers the whole sub-tree rooted at $v$. % i.e., such that for every descendant $u$ of $v$ there exists node $w$ in the resolution such that $w$ is non-blank and $w$ an ancestor of $u$.


%%% Local Variables:
%%% mode: latex
%%% TeX-master: "main"
%%% End:


\subsection{Protocol State}
% !TEX root = main.tex
% !TeX spellcheck = en_US

The state of $\saik$ consists of a number of variables, outlined in \cref{tab:prot-state}. The table also includes short
descriptions of the roles of the secrets in the key schedule. The protocol will ensure that states of any two parties in
the same epoch differ at most in labels of nodes of $\itkSt.\tree$ that describe secret keys and the label
$\itkSt.\leaf$. This means that they agree on the secrets $\itkSt.\applicationSecret$ and $\itkSt.\initSecret$, as well
as on the public context, computed by the helper method $\groupContext()$ in \cref{tab:prot-state}.



%%% Local Variables:
%%% mode: latex
%%% TeX-master: "main"
%%% End:


\subsection{Protocol Algorithms}
The protocol algorithms are defined in \cref{fig:prot1,fig:prot-helpers2}.
% !TEX root = main.tex
% !TeX spellcheck = en_US


\begin{figure*}[!p]%\vspace*{-2em}\hspace*{-1.5em}
%  \begin{minipage}{\linewidth+2em}
	\begin{anybox}{\sffamily\bfseries \saik : Algorithms}
			\begin{minipage}[t]{0.48\linewidth}
        {\bf Initialization}
        \begin{algorithmic}
          \If{$\id = \pgod$}
            \State $\itkSt \gets \method{new-state}()$
            \State $\itkSt.\groupId, \itkSt.\initSecret, \itkSt.\membershipKey, \itkSt.\applicationSecret \getsr \bits^\secparam$
            \State $\itkSt.\tree \gets \method{new-LBT}()$
            \State $\itkSt.\leaf \gets \itkSt.\tree.\leaves[0]$
            \State $(\itkSt.\leaf.\rsvk, \itkSt.\leaf.\rssk) \gets \sigkg()$
          \EndIf
        \end{algorithmic}

        \medskip
        {\bf Input $(\keyword{Send}, \hgact), \hgact \in \{\hglu , \hglr\md\id_t, \hgla\md\id_t\}$ from $\id$}
				\begin{algorithmic}
					\State $\KwReq\ \itkSt \neq \bot$
          \State \vspace*{-.5em}\Comment{In case of add, fetch $\id_t$'s keys from AKS (AKS runs $\genPKIkeys$).}
          \If{$\hgact = \hgla\md\id_t$}
            \State $(\mmpkepk_t, \rsvk_t, \mmpkepk_t') \gets \KwQuery\ (\keyword{GetPk},\id_t) \text{ to } \funcKB$
            \State $\hgact \gets \hgla\md\id_t\md(\mmpkepk_t, \rsvk_t,\mmpkepk_t')$
          \EndIf
          \State \smashedComment{Create the state and secrets for the new epoch.}
          \State \KwTry{} $(\itkSt', \pathSecrets, \joinerSecret)  \gets\provState(\hgact)$
          \State \Comment{Encrypt the path secrets using the new epoch's ratchet tree. For adds, also encrypt the joiner secret.}
          \If{$\hgact \in \{\hglu , \hglr\md\id_t\}$}
            \State $\pathSecCtxt \gets \encSecrets(\itkSt', \pathSecrets, \bot, \bot, \bot)$
          \ElsIf{$\hgact = \hgla\md\id_t\md(\mmpkepk_t, \rsvk_t,\mmpkepk_t')$}
            \State $\pathSecCtxt \gets \encSecrets(\itkSt', \pathSecrets, \id_t, \mmpkepk_t', \joinerSecret)$
          \EndIf
%          \State \smashedComment{Sign data under current epoch's secrets.}
%          \State $\updatedPks \gets ((\itkSt'.\leaf.\mmpkepk, \itkSt'.\leaf.\rsvk))$
%          \State $\updatedPks \gets \updatedPks \concat (v.\mmpkepk : v \in \itkSt'.\tree.\directpath(\itkSt'.\leaf))$
%          \State $(\vec{\variable{tbs}}, \rdclass) \gets \method{to-be-signed}(\pathSecCtxt, \updatedPks, \hgact)$
%          \State $\sig \gets \rssignL(\itkSt.\tree.\leafof(\id).\rssk, \itkSt.\membershipKey, \vec{\variable{tbs}}, \rdclass)$
%          \State $\itkSt \gets \itkSt'$
          \State $\ssk \gets \itkSt.\tree.\leafof(\id).\ssk$
          \State  $\sig \gets \sigsign(\ssk, \underline{\itkSt'.\confTag})$
          \If{$\hgact = \hglr\md\id_t$}
            \State \smashedComment{Authenticate removal message for $\id_t$}
            \State $\sig_t \gets \sigsign(\ssk, (\id, \hglr\md\id_t))$
            \State $\macsig_t \gets \mactag(\itkSt.\membershipKey, (\id, \hglr\md\id_t, \underline{\itkSt.\confTag}))$
            \State \Return $(\id, \hgact, \underline{\pathSecCtxt}, \updatedPks, \sig, \sig_t, \macsig_t)$
          \EndIf
          \State $\itkSt \gets \itkSt'$
          \If{$\hgact = \hgla\md\id_t\md(\mmpkepk_t, \rsvk_t,\mmpkepk_t')$}
            \State \smashedComment{Send additional data for $\id_t$.}
            \State $\variable{welcomeData} \gets (\itkSt.\groupId, \itkSt.\tree.\public(), \mmpkepk_t')$
            \State \Return $(\id, \hgact, \underline{\pathSecCtxt}, \updatedPks, \sig, \variable{welcomeData})$
          \EndIf
          \State \Return $(\id, \hgact, \underline{\pathSecCtxt}, \updatedPks, \sig)$
				\end{algorithmic}

      \end{minipage}
		\hfill
			\begin{minipage}[t]{0.5\linewidth}
        \bf Input {$\keyword{Key}$ from $\id$}
				\begin{algorithmic}
					\State $\KwReq\ \itkSt \neq \bot$
					\State $k \gets \itkSt.\applicationSecret$
					\State $\itkSt.\applicationSecret \gets \bot$
					\State \Return $k$
				\end{algorithmic}

        \medskip
        {\bf Input $(\keyword{Receive}, (\id_s, \literal{removed}, \sig_t, \macsig_t))$ from $\id$}

        \textnormal{\smashedComment{Receiver is removed.}}
        \begin{algorithmic}
          \State $\ersvk \gets \itkSt.\tree.\leafof(\id_s).\rsvk$
          \State \KwReq{} $\sigvrf(\ersvk, (\id_s, \hglr\md\id), \sig_t)$
          \State \KwReq{} $\macvrf(\itkSt.\membershipKey, (\id_s, \hglr\md\id, \underline{\itkSt.\confTag}), \macsig_t)$
%          \State $\vec{\variable{tbv}} \gets ((\id_s, \hglr\md\id, \itkSt.\groupId))$
%          \State \vspace*{-.5em}\Comment{\normalfont Check if removing allowed and compute the reduction pattern $\rd=(\ell,0,1)$.}
%          \State \KwTry{} $\itkSt' \gets \applyact(\itkSt.\clone(), \id_s, \hglr\md\id)$
%          \State $\ell \gets \getWeights(\itkSt', \id_s)$
%          \State $\ersvk \gets \itkSt.\tree.\leafof(\id_s).\rsvk$
%          \State \KwReq{} $\rsvrfyL(\rsvk, \itkSt.\membershipKey, \vec{\variable{tbv}}, (\ell,0,1), \redactedSig)$
          \State $\itkSt \gets \bot$
          \State \Return $(\id_s, \hglr\md\id)$
        \end{algorithmic}

        \medskip
        \textbf{Input $(\keyword{Receive}, (\id_s, \hgact, \underline{\pathSecCtxtInd}, \underline{\redactedUpPks}, \sig))$ from $\id$}

        \textnormal{\smashedComment{Receiver is a member.}}
        \begin{algorithmic}
          \State \KwTry{} $\itkSt' \gets \applyact(\itkSt.\clone(), \id_s, \hgact)$
%          \State \smashedComment{\normalfont Get the expected reduction pattern using the new state.}
%          \State \KwTry{} $\rd \gets \myReduction(\itkSt'.\tree, \id_s)$
%          \State $\vec{\variable{tbv}} \gets (\pathSecCtxtInd) \concat ((\id_s, \hgact, \itkSt.\groupId)) \concat \redactedUpPks$
%          \State $\ersvk \gets \itkSt.\tree.\leafof(\id_s).\rsvk$
%          \State \KwReq{} $\rsvrfyL(\rsvk, \itkSt.\membershipKey, \vec{\variable{tbv}}, \rd, \redactedSig)$
%          \State \smashedComment{\normalfont Transition to next epoch.}
          \State \KwTry{} $(\itkSt, \confTag) \gets \nextState(\itkSt', \underline{\pathSecCtxtInd}, \underline{\redactedUpPks}, \id_s, \hgact)$
          \State $\ersvk \gets \itkSt.\tree.\leafof(\id_s).\rsvk$
          \State \KwReq{} $\sigvrf(\ersvk, \underline{\confTag}, \sig)$
          \If{$\hgact=\hgla\md\id_t\md(\mmpkepk_t,\rsvk_t)$}
            \Return $(\id_s, \hgla\md\id_t)$
          \Else\
            \Return $(\id_s, \hgact)$
          \EndIf
        \end{algorithmic}

        \medskip
        \textbf{Input $(\keyword{Receive}, (\id_s, \hgact, \encGroupSecret_1, \encGroupSecret_2, \variable{welcomeData})))$ from $\id$}

        \textnormal{\smashedComment{Receiver joins.}}
        \begin{algorithmic}
          \State $\KwReq\ \itkSt = \bot$
          \State \KwParse{} $(\groupId, \tree, \mmpkepk') \gets \variable{welcomeData}$
          \State $\itkSt \gets \method{new-state}$
          \State $(\itkSt.\groupId, \itkSt.\tree, \itkSt.\lastAct) \gets (\groupId, \tree, (\id_s, \hgla\md\id))$
          \State $v \gets \itkSt.\tree.\leafof(\id)$
          \State \KwTry{} $(\mmpkesk, \rssk, \mmpkesk') \gets \KwQuery\ \keyword{GetSk}((v.\mmpkepk, v.\rsvk, \mmpkepk')) \textnormal{ to } \funcKB$
          \State $(v.\mmpkesk, v.\rssk) \gets (\mmpkesk, \rssk)$
          \State $\itkSt \gets \setTreeHash(\itkSt)$
          \State \KwTry{} $(\itkSt, \confTag) \gets \populateSecrets(\itkSt, \mmpkesk', \encGroupSecret_1, \encGroupSecret_2, \id_s)$
%          \State $\ersvk \gets \itkSt.\tree.\leafof(\id_s).\rsvk$
%          \State \KwReq{} $\sigvrf(\ersvk, \confTag)$
					\State \Return $(\itkSt.\tree.\roster(), \id_s)$
        \end{algorithmic}
      \end{minipage}
  \end{anybox}

  \medskip
  \begin{anybox}{\sffamily\bfseries \saik : Helpers for encryption and key generation for $\funcPKI$}
			\begin{minipage}[t]{.49\linewidth}
        {\bf {helper $\encSecrets(\itkSt', \pathSecrets, \id_t, \mmpkepk_t', \joinerSecret)$}}
				\begin{algorithmic}
          \State $L \gets \getPathSecsMap(\itkSt'.\tree, \id)$
          \State $\vec m, \vec \mmpkepk \gets ()$
          \For{$j=1$ \bf to $\len(L)$}
            \State $(i, v) \gets L[j]$
            \State $\vec m \listapp \pathSecrets[i]$
            \If{$\id_t\neq\bot \land v = \itkSt'.\tree.\leafof(\id_t)$}
              $\vec \mmpkepk \listapp \mmpkepk_t'$
            \Else\
              $\vec \mmpkepk \listapp \vec v.\mmpkepk$
            \EndIf
          \EndFor
          \If{$\id_t\neq\bot$}
            \State $\vec m \listapp \joinerSecret$
            \State $\vec\mmpkepk \listapp \mmpkepk_t'$
          \EndIf
          \State \Return $\underline{\mmpkeEncL}(\vec{\mmpkepk}, \vec m)$
        \end{algorithmic}
      \end{minipage}
		\hfill
			\begin{minipage}[t]{0.49\linewidth}
        {\bf {helper $\decSecrets(\itkSt', \id_s, \pathSecCtxtInd)$}}
				\begin{algorithmic}
          \State $v \gets \lca(\itkSt'.\tree.\leafof(\id_s), \itkSt'.\leaf).\resolvent(\itkSt'.\leaf)$
          \State \Return $\underline{\mmpkeDecL}(v.\mmpkesk, \pathSecCtxtInd)$
        \end{algorithmic}

        \medskip
        {\bf {helper $\genPKIkeys()$}}
				\begin{algorithmic}
          \State $(\mmpkepk, \mmpkesk) \gets \underline{\mmpkeKeyGenL}()$
          \State $(\rsvk, \rssk) \gets \sigkg()$
          \State $(\mmpkepk', \mmpkesk') \gets \underline{\mmpkeKeyGenL}()$
          \State \Return $((\mmpkepk, \rsvk, \mmpkepk'), (\mmpkesk, \rssk, \mmpkesk'))$
        \end{algorithmic}
      \end{minipage}
  \end{anybox}
	\vspace*{-0.7em}
	\caption{The algorithms of \saik.}
	\label{fig:prot1}
%  \end{minipage}
\end{figure*}


%\begin{figure}[!tpb]\vspace*{-5em}\hspace*{-1.5em}
%  \begin{minipage}{\linewidth+2em}
%	\begin{anybox}{\sffamily\bfseries \saik : Helpers for encryption and authentication}
%		\scalebox{0.7}{
%			\begin{minipage}[t]{.6\linewidth}
%        {\bf {helper $\encSecrets(\itkSt', \pathSecrets, \id_t, \mmpkepk_t', \joinerSecret)$}}
%				\begin{algorithmic}
%          \State $L \gets \getPathSecsMap(\itkSt'.\tree, \id)$
%          \State $\vec m, \vec \mmpkepk \gets ()$
%          \For{$j=1$ \bf to $\len(L)$}
%            \State $(i, v) \gets L[j]$
%            \State $\vec m \listapp \pathSecrets[i]$
%            \If{$\id_t\neq\bot \land v = \itkSt'.\tree.\leafof(\id_t)$}
%              $\vec \mmpkepk \listapp \mmpkepk_t'$
%            \Else\
%              $\vec \mmpkepk \listapp \vec v.\mmpkepk$
%            \EndIf
%          \EndFor
%          \If{$\id_t\neq\bot$}
%            \State $\vec m \listapp \joinerSecret$
%            \State $\vec\mmpkepk \listapp \mmpkepk_t'$
%          \EndIf
%          \State \Return $(\mmpkeEncL(\vec{\mmpkepk}, \vec m))$
%        \end{algorithmic}
%
%        \medskip
%        {\bf {helper $\decSecrets(\itkSt', \id_s, \pathSecCtxtInd)$}}
%				\begin{algorithmic}
%          \State $v \gets \lca(\itkSt'.\tree.\leafof(\id_s), \itkSt'.\leaf).\resolvent(\itkSt'.\leaf)$
%          \State \Return $\mmpkeDecL(v.\mmpkesk, \pathSecCtxtInd)$
%        \end{algorithmic}

%        \medskip
%        {\bf \mbox{helper $\method{to-be-signed}(\itkSt', \pathSecCtxt, \updatedPks, \hgact)$}}
% 				\begin{algorithmic}
%          \State $\vec w \gets \getWeights(\itkSt'.\tree, \id)$
%          \State $\vec{\variable{tbs}} \gets ()$
%          \For{$j=1$ \bf to $\abs{\vec w}$}
%            \State $\vec{\variable{tbs}} \listapp \mmpkeExtL(\pathSecCtxt,j)$
%          \EndFor
%          \State $\vec{\variable{tbs}} \listapp (\id, \hgact, \itkSt.\groupId)$
%          \State $\vec{\variable{tbs}} \listapp \updatedPks$
%          \State \Return $(\vec{\variable{tbs}}, \rdclassBGM_{\abs{\vec w},\vec w})$
%        \end{algorithmic}
%      \end{minipage}}
%		\hfill\scalebox{.7}{
%			\begin{minipage}[t]{0.75\linewidth}
%        {\bf \mbox{helper $\myReduction(\tree', \id_s)$}}
% 				\begin{algorithmic}
%          \State $L \gets \getPathSecsMap(\tree', \id_s)$
%          \For{$j=1$ \bf to $\len(L)$}
%            \State $(i, v) \gets L[j]$
%            \If{$\itkSt'.\tree.\inSubtree(\tree'.\leafof(\id), v)$}
%              \State \vspace*{-.5em}\Comment{We want the $j$-th ciphertext out of $\len(L)$ and the first $i+1$ items on the prefix list: the aux data, the leaf $\mmpkepk$ and $i-1$ $\mmpkepk$'s on $\id_s$'s direct path.}
%              \State \Return $(\len(L), j, i)$
%            \EndIf
%          \EndFor
%          \State \Return $\bot$
%        \end{algorithmic}

%        \medskip
%        {\bf {helper $\getPathSecsMap(\tree', \id_s)$}}
%				\begin{algorithmic}
%          \State\vspace*{-.7em}{\Comment{Returns a list of tuples $(i,v)$, denoting that when $\id_s$ commits in $\tree'$, the $i$-th path secret is encrypted under node $v$'s keys.}}
%          \State $L \gets ()$
%          \State $\fullpath \gets (\tree'.\leafof(\id_s)) \concat \tree'.\directpath(\tree'.\leafof(\id_s))$
%          \For{$i=1$ \bf to $\len(\fullpath)-1$}
%            \State $\vec v \gets \fullpath[i+1].\resolution() \setminus \fullpath[i].\resolution()$
%            \For{$j=1$ \bf to $\abs{\vec v}$}
%              \State $L \listapp (i, \vec v[j])$
%            \EndFor
%          \EndFor
%          \State \Return $L$
%        \end{algorithmic}

%        \medskip
%        {\bf {helper $\getWeights(\tree', \id_s)$}}
%				\begin{algorithmic}
%          \State\vspace*{-.7em}{\Comment{Returns a list of weights for $\rdclassBGM_{\ell,\vec w}$ when $\id_s$ commits in $\tree'$. Also allows to compute $\ell = \abs{\vec w}$.}}
%          \State $\vec w \gets ()$
%          \State $\fullpath \gets (\tree'.\leafof(\id_s)) \concat \tree'.\directpath(\tree'.\leafof(\id_s))$
%          \For{$i=2$ \bf to $\len(\fullpath)$}
%            \State $\vec v \gets \fullpath[i].\resolution() \setminus \fullpath[i-1].\resolution()$
%            \For{$j=1$ \bf to $\abs{\vec w}$}
%              \State \mbox{$\vec w \listapp  \big|\{u \in  \tree'.\leaves \mid u.\inuse() \land \tree'.\inSubtree(u,\vec v[j]) \}\big|$}
%            \EndFor
%          \EndFor
%          \State \Return $\vec w$
%        \end{algorithmic}
%      \end{minipage}}
%  \end{anybox}
%	\vspace*{-0.7em}
%	\caption{The algorithms of \saik.}
%	\label{fig:prot}
%  \end{minipage}
%\end{figure}


\begin{figure*}[!p]
  	\begin{anybox}{\sffamily\bfseries \saik : Creating epochs}
			\begin{minipage}[t]{.49\linewidth}
        {\bf {helper $\provState(\itkSt, \id, \hgact)$}}
				\begin{algorithmic}
          \State $\itkSt' \gets \itkSt.\clone()$
          \State \vspace*{-.7em}\Comment{Apply the action to the tree. Fails if the action is not allowed.}
          \State $\KwTry\ \itkSt' \gets \applyact(\itkSt', \id, \hgact)$
          \State \smashedComment{Re-key the direct path.}
          \State $\directpath \gets \itkSt'.\tree.\directpath(\itkSt'.\leaf)$
          \State $\pathSecrets[\wc] \gets \bot$
          \State $\pathSecrets[1] \getsr \{0,1\}^\secparam$
					\For{$i=1$ \textbf{to} $\len(\directpath)-1$}
            \State $v \gets \directpath[i]$
					  \State $r \gets \hkdfexp(\pathSecrets[i], \literal{node})$
            \State $(v.\mmpkepk, v.\mmpkesk) \gets \mmpkeKeyGenL(r)$
            \State $\pathSecrets[i+1] \gets \hkdfexp(\pathSecret[i], \literal{path})$
          \EndFor
          \State $\itkSt'.\tree.\mergeleaves(\itkSt'.\leaf)$
          \State \smashedComment{Re-key the leaf.}
          \State $(\itkSt'.\leaf.\mmpkepk, \itkSt'.\leaf.\mmpkesk) \gets \mmpkeKeyGenL()$
%          \State $(\itkSt'.\leaf.\rsvk, \itkSt'.\leaf.\rssk) \gets \rskeygenL()$
          \State $(\itkSt'.\leaf.\rsvk, \itkSt'.\leaf.\rssk) \gets \sigkg()$
          \State \vspace*{-.5em}\Comment{Set all context variables and then derive epoch secrets.}
          \State $\itkSt'.\lastAct \gets (\id, \hgact)$
          \State $\itkSt' \gets \setTreeHash(\itkSt')$
          \State $\commitSecret \gets \pathSecrets[\len(\pathSecrets)]$
          \State $(\itkSt', \joinerSecret) \gets \deriveKeys(\itkSt', \commitSecret)$
          \State \Return $(\itkSt', \pathSecrets, \joinerSecret)$
        \end{algorithmic}

        \medskip
        {\bf {helper $\applyact(\itkSt', \id_s, \hgact)$}}
				\begin{algorithmic}
          \State \KwReq{} $\id_s \in \itkSt'.\tree.\roster()$
          \If{$\hgact=\hglr\md\id_t$}
            \State \KwReq{} $\id_s \neq \id_t \land \id_t \in \itkSt'.\tree.\roster()$
            \State $\itkSt'.\tree.\blankpath(\itkSt'.\tree.\leafof(\id_t))$
            \State $\itkSt'.\tree.\leafof(\id_t).\blank()$
          \ElsIf{$\hgact=\hgla\md\id_t\md(\mmpkepk_t, \rsvk_t)$}
            \State \KwReq{} $\id_t \notin \itkSt'.\tree.\roster()$
            \State $v \gets \itkSt'.\tree.\getleaf()$
            \State $(v.\id, v.\mmpkepk, v.\rsvk) \gets (\id_t, \mmpkepk_t, \rsvk_t)$
            \State $\itkSt.\tree.\unmergeleaf(v)$
          \EndIf
        \end{algorithmic}

      \end{minipage}
		\hfill
			\begin{minipage}[t]{0.49\linewidth}
        {\bf {helper $\nextState(\itkSt', \pathSecCtxtInd, \redactedUpPks, \id_s, \hgact)$}}
				\begin{algorithmic}
          \State \smashedComment{Set keys on the re-keyed path.}
          \State $v_s \gets \itkSt'.\tree.\leafof(\id_s)$
          \State $\directpath \gets \itkSt'.\tree.\directpath(v_s)$
          \State $(v_s.\mmpkepk, v_s.\rsvk) \gets \redactedUpPks[1]$
          \State $i \gets 1$
          \While{$\directpath[i] \notin \{\itkSt'.\tree.\lca(\itkSt'.\leaf, v_s), \itkSt'.\tree.\rt\}$}
            \State \smashedComment{If message contains too few ek's, reject it.}
            \State \KwReq{} $i+1\leq\len(\redactedUpPks)$
            \State $\directpath[i].\mmpkepk \gets \redactedUpPks[i+1]$
            \State $i\inc$
          \EndWhile
          \State \smashedComment{Decrypt the path secret using the updated tree.}
          \State \KwTry{} $\pathSecret \gets \decSecrets(\itkSt', \id_s, \pathSecCtxtInd)$
          \While{$i < \len(\directpath)$}
            \State $v \gets \directpath[i]$
					  \State $r \gets \hkdfexp(\pathSecrets[i], \literal{node})$
            \State $(v.\pkpk, v.\pksk) \gets \mmpkeKeyGenL(r)$
            \State $\pathSecret \gets \hkdfexp(\pathSecret, \literal{path})$
            \State $i\inc$
          \EndWhile
          \State $\commitSecret \gets \pathSecret$
          \State $\itkSt'.\tree.\mergeleaves(v_s)$
          \State \smashedComment{Set all context variables and then derive epoch secrets.}
          \State $\itkSt'.\lastAct \gets (\id_s, \hgact)$
          \State $\itkSt' \gets \setTreeHash(\itkSt')$
          \State $(\itkSt', \joinerSecret)  \gets \deriveKeys(\itkSt', \commitSecret)$
          \State \Return $\itkSt'$
        \end{algorithmic}

        \medskip
        {\bf {helper $\populateSecrets(\itkSt', \pksk', \encGroupSecret_1, \encGroupSecret_2, \id_s)$}}
				\begin{algorithmic}
					\State \KwTry{} $\pathSecret \gets \underline{\mmpkeDecL}(\pksk, \encGroupSecret_1)$
          \State \KwTry{} $\joinerSecret \gets \underline{\mmpkeDecL}(\pksk, \encGroupSecret_2)$
					\State $v \gets \itkSt'.\tree.\lca(\itkSt'.\leaf, \itkSt'.\tree.\leafof(\id_s))$
					\While{$v \neq \itkSt'.\tree.\rt$}
					  \State $r \gets \hkdfexp(\pathSecret, \literal{node})$
					  \State $(\mmpkepk, v.\mmpkesk) \gets \underline{\mmpkeKeyGenL}(r)$
					  \State $\KwReq\ v.\mmpkepk = \mmpkepk$
					  \State $\pathSecret \gets \hkdfexp(\pathSecret, \literal{path})$
					  \State $v \gets v.\parent$
					\EndWhile
          \State $\itkSt' \gets \deriveEpochKeys(\itkSt', \joinerSecret)$
          \State \Return $\itkSt'$
        \end{algorithmic}
      \end{minipage}
  \end{anybox}
%\caption{}\label{fig:prot-helpers1}
%\end{figure}
%\begin{figure}[tbp]
	\begin{tcbraster}[raster columns=2, raster equal height]
		\begin{anybox}{\sffamily\bfseries \saik : Key schedule}
				\begin{minipage}[t]{\linewidth}
					{\bf {helper $\deriveKeys(\itkSt, \itkSt', \commitSecret)$}}
					\begin{algorithmic}
						\State $\joinerSecret \gets \hkdfext(\itkSt.\initSecret, \commitSecret)$
						\State $\itkSt' \gets \deriveEpochKeys(\itkSt', \joinerSecret)$
						\State \Return $\itkSt', \joinerSecret$
					\end{algorithmic}

					\medskip
					{\bf {helper $\deriveEpochKeys(\itkSt', \joinerSecret)$}}
					\begin{algorithmic}
            \State $\epochSecret \gets \hkdfext(\joinerSecret, \itkSt'.\groupContext())$

						\State $\itkSt'.\applicationSecret \gets \hkdfexp(\epochSecret, \literal{app})$
						\State $\itkSt'.\membershipKey \gets \hkdfexp(\epochSecret, \literal{membership})$
						\State $\itkSt'.\initSecret \gets \hkdfexp(\epochSecret, \literal{init})$
            \State $\itkSt'.\confTag \gets \hkdfexp(\epochSecret, \literal{confirmation})$
						\State \Return $\itkSt'$
					\end{algorithmic}
				\end{minipage}
		\end{anybox}
		%
		\begin{anybox}{\sffamily\bfseries \saik : Tree hash}
				\begin{minipage}[t]{\linewidth}
					{\bf {helper $\setTreeHash(\itkSt')$}}
        \begin{algorithmic}
					\State $\itkSt'.\treeHash \gets \computeTreeHash(\itkSt'.\tree.\rt)$
					\State \Return $\itkSt'$
				\end{algorithmic}

        \medskip
        {\bf {helper $\computeTreeHash(v)$}}
				\begin{algorithmic}
					\If{$v.\isleaf$}
					  \State \Return $\hash(v.\nodeIndex, v.\mmpkepk, v.\ersvk)$
					\Else
            \State $\ell \gets \len(v.\children)$
            \For{$i\in[\ell]$}
  					  $h_i \gets \computeTreeHash(v.\children[i])$
            \EndFor
            \State $h \gets (h_1, \dots, h_\ell)$
            \State \Return $\hash(v.\nodeIndex, v.\mmpkepk, v.\unmergedLeaves, h)$
					\EndIf
				\end{algorithmic}
				\end{minipage}
		\end{anybox}
	\end{tcbraster}

	\caption{Additional helper methods for \saik.}
	\label{fig:prot-helpers2}
%\end{minipage}
\end{figure*}

%%% Local Variables:
%%% mode: latex
%%% TeX-master: "main"
%%% End:


\subsection{Extraction Procedure for the Mailboxing Service}
{\color{red}\bf TODO}
In this section, we describe how in practice the mailboxing service can compute \saik messages delivered to parties. Recall that this is not formally part of our model, because we do not consider correctness guarantees.

We will use a method $\method{get-index}(\tree, \id_s, \id_t) \to (i, j)$. It takes as input a ratchet tree $\tree$ and two parties within $\tree$ and outputs two indices interpreted as follows. Say $\id_s$ creates an epoch with the new tree $\tree$ and sends a multi-recipient ciphertext $Ctxt$ and a vector of public keys $\updatedPks$. First, $i$ is the receiver index of $\id_t$'s public key in $Ctxt$. Second, $j$ is such that $\id_t$ expects to receive the first $j$ elements in $\updatedPks$. Observe that $i$ and $j$ can be computed by

Say a party $\id_s$ performing an operation $\hgact$ sends a message $C$ to the service. A receiver $\id_r$ expects a message in one of three formats, depending on $\hgact$:
\begin{description}
  \item[] {\it Case $\id_r$ is removed: } The service sends to $\id_r$ the values $\id_s$, $\sig_t$ and $\macsig_t$ contained in $C$.
  \item[] {\it Case $\id_r$ is in the group: } The service sends $\id_s$ and $\hgact$ to $\id_r$. Then, $\id_r$ computes its index $i$ in the multi-recipient ciphertext $Ctxt$ included in

$\id_r$ executes the receive algorithm until calling \decSecrets
  \item[] {\it Case $\id_r$ joins: } The service sends $\variable{welcomeData}$ to $\id_r$. Recall that $\id_r$ acts as two receivers of $\pathSecCtxt$: the $i$-th one for its path secret and the last, $n$-th one for its joiner secret. The service first retrieves the ratchet tree $\tree$ from $\variable{welcomeData}$ included in $C$. Then, it computes $i$ and $n$ based on $\tree$, $\id_s$ and $\id_r$ (formally, it computes $i$ and $n$ by executing the \method{encrypt} method from \cref{fig:prot1} with all secrets set to $0$). Then, it sends to $\id_r$ the ciphertexts $\pathSecCtxtInd_1=\mmpkeExtL(\textit{Ctxt},i)$ and $\mathit{ctxt}_2 = \mmpkeExtL(\mathit{Ctxt},n)$, as well as $\id_s$, $\hgact$ and $\variable{welcomeData}$, all included in $C$.

\end{description}

%\begin{description}
%  \item[] {\it Case $\id_r$ is removed: } The service sends to $\id_r$ the values $\id_s$, $\sig_t$ and $\macsig_t$ contained in $C$.
%%
%%First, $\id_r$ sends to the service the reduction pattern $\rd=(\ell,0,1)$ it expects, $\id_s$'s verification key $\ersvk$ and $\groupId$ (to compute them, it executes the first 4 lines of Receive). The service computes the message $\vec{\variable{tbs}} = (\mathit{ctxt}_1,\dots,\mathit{ctxt}_\ell) \concat ((\id_s,\hgact,\groupId)) \concat \updatedPks$ signed by $\id_s$ (see \cref{fig:prot1}), where $\mathit{ctxt}_i = \mmpkeExtL(\textit{Ctxt},i)$ for all $i$.\footnote{In typical constructions, including ours, this is very efficient} Then, it sends to $\id_r$ its signature $\sig' = \ersred(\ersvk, \sig, \vec{\variable{tbv}},\rd)$.
%  \item[] {\it Case $\id_r$ joins: } The service sends $\variable{welcomeData}$ to $\id_r$. Recall that $\id_r$ acts as two receivers of $\pathSecCtxt$: the $i$-th one for its path secret and the last, $n$-th one for its joiner secret. The service first retrieves the ratchet tree $\tree$ from $\variable{welcomeData}$ included in $C$. Then, it computes $i$ and $n$ based on $\tree$, $\id_s$ and $\id_r$ (formally, it computes $i$ and $n$ by executing the \method{encrypt} method from \cref{fig:prot1} with all secrets set to $0$). Then, it sends to $\id_r$ the ciphertexts $\pathSecCtxtInd_1=\mmpkeExtL(\textit{Ctxt},i)$ and $\mathit{ctxt}_2 = \mmpkeExtL(\mathit{Ctxt},n)$, as well as $\id_s$, $\hgact$ and $\variable{welcomeData}$, all included in $C$.
%%  Therefore, $\id_r$ sends to the service $i$ and $n$ computed based on the ratchet tree $\tree$ in $\variable{welcomeData}$ and $\id_s$. (In detail, $\id_r$ executes the helper method $\myReduction(\tree, \id_s)$ from \cref{fig:prot} which outputs $(n,i,\wc)$.) The service sends back $\pathSecCtxtInd_1=\mmpkeExtL(\textit{Ctxt},i)$ and $\mathit{ctxt}_2 = \mmpkeExtL(\mathit{Ctxt},n)$.
%  \item[] {\it Else: } The service sends $\id_s$ and $\hgact$ to $\id_r$. Then, $\id_r$ executes the receive algorithm until calling \decSecrets
%%  $\id_r$ sends to the service the reduction pattern $\rd=(\ell,i,j)$ it expects, $\id_s$'s verification key $\ersvk$ and $\groupId$ (to compute them, it executes the first 4 lines of Receive). The service computes $\vec{\variable{tbs}}$ as in the case where $\id_r$ is removed and sends to $\id_r$ its individual ciphertext $\mathit{ctxt} =  \mmpkeExtL(\textit{Ctxt},i)$, its public keys $\updatedPks[1], \dots, \updatedPks[j]$ and signature $\sig'  = \ersred(\ersvk, \sig, \vec{\variable{tbs}},\rd)$.
%\end{description}


The service delivers a message in one of

Recall that according \saik, a party $\id_s$ performing operation $\hgact$ sends to the mailboxing service $\id_s$ and $\hgact$ as well as a multi-recipient ciphertext $\pathSecCtxt$, a list of updated keys  $\updatedPks$ and a signature $\sig$, where $\pathSecCtxt$ is a multi-recipient ciphertext encrypting path secrets and $\updatedPks$ is a list of new keys for $\id_s$'s path. In case of an add, $\id_s$ also sends $\variable{welcomeData}$ for the joiner.

When a receiver $\id_r$ wants to download its message, the service first sends $\id_s$ and $\hgact$ to $\id_r$. Then, we have three cases:
\begin{description}
  \item[] {\it $\id_r$ is removed: } First, $\id_r$ sends to the service the reduction pattern $\rd=(\ell,0,1)$ it expects, $\id_s$'s verification key $\ersvk$ and $\groupId$ (to compute them, it executes the first 4 lines of Receive). The service computes the message $\vec{\variable{tbs}} = (\mathit{ctxt}_1,\dots,\mathit{ctxt}_\ell) \concat ((\id_s,\hgact,\groupId)) \concat \updatedPks$ signed by $\id_s$ (see \cref{fig:prot1}), where $\mathit{ctxt}_i = \mmpkeExtL(\textit{Ctxt},i)$ for all $i$.\footnote{In typical constructions, including ours, this is very efficient} Then, it sends to $\id_r$ its signature $\sig' = \ersred(\ersvk, \sig, \vec{\variable{tbv}},\rd)$.
  \item[] {\it $\id_r$ joins: } The service sends $\variable{welcomeData}$ to $\id_r$. Recall that $\id_r$ acts as two receivers of $\pathSecCtxt$: the $i$-th one for its path secret and the last, $n$-th one for its joiner secret. Therefore, $\id_r$ sends to the service $i$ and $n$ computed based on the ratchet tree $\tree$ in $\variable{welcomeData}$ and $\id_s$. (In detail, $\id_r$ executes the helper method $\myReduction(\tree, \id_s)$ from \cref{fig:prot} which outputs $(n,i,\wc)$.) The service sends back $\pathSecCtxtInd_1=\mmpkeExtL(\textit{Ctxt},i)$ and $\mathit{ctxt}_2 = \mmpkeExtL(\mathit{Ctxt},n)$.
  \item[] {\it else: } $\id_r$ sends to the service the reduction pattern $\rd=(\ell,i,j)$ it expects, $\id_s$'s verification key $\ersvk$ and $\groupId$ (to compute them, it executes the first 4 lines of Receive). The service computes $\vec{\variable{tbs}}$ as in the case where $\id_r$ is removed and sends to $\id_r$ its individual ciphertext $\mathit{ctxt} =  \mmpkeExtL(\textit{Ctxt},i)$, its public keys $\updatedPks[1], \dots, \updatedPks[j]$ and signature $\sig'  = \ersred(\ersvk, \sig, \vec{\variable{tbs}},\rd)$.
\end{description}
\paragraph{An alternative solution.}
In the solution described above, receiving a message requires interaction between the delivery service and $\id_r$. This
is not a problem in typical scenarios, because they are both online at that moment. However, we note that there is an
alternative solution which does not require interaction. Specifically, the messages sent by \saik contain enough
information for the service to compute the public part of the ratchet tree in any epoch. The tree, in turn, is sufficient to compute the message delivered to any party.
The downside of this solution is that it requires the mailboxing service to store many ratchet trees (or many messages to re-compute them) for parties in different epochs.

%%% Local Variables:
%%% mode: latex
%%% TeX-master: "main"
%%% End:
