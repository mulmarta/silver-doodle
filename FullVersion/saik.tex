% !TEX root = main.tex
% !TeX spellcheck = en_US

\section{The \saik Protocol}\label{sec:saik}
\saik inherits most of its mechanisms from \protITK, the CGKA of \mls. We briefly recall \protITK in
\cref{sec:intuition1}.
%\dnote{Move to appendix if we can't fit the page limit? Or replace with a short paragraph on which
%concepts we assume to be known? I.e. ratchet trees, etc.}
Readers familiar with \protITK can jump directly to \cref{sec:intuition2} which gives intuition how \saik improves on \protITK. The detailed description of \saik can be found in \cref{sec:saik-details}.

\subsection{Intuition for the \protITK Protocol}\label{sec:intuition1}

\paragraph{Ratchet trees.}
The operation of \protITK relies on a data structure called ratchet trees. A ratchet tree $\tree$ is a tree where leaves are assigned to group members, each storing its owner's identity and signature key pair. Moreover, most non-root nodes in $\tree$, store encryption key pairs. Nodes without a key pair are called \emph{blank}.

\protITK maintains the following \emph{tree invariant}:
%\begin{enumerate}[itemsep=0pt]
%  \item[]
  {\it Each member knows the secret keys of the nodes on the path from their leaf to the root, and only those, as well as all public keys in $\tree$.}
%\end{enumerate}
This allows to efficiently encrypt messages to subgroups: If a node $v$ is not blank, then a message $m$ can be
encrypted to all members in the subtree of a node $v$ by encrypting it under $v$'s public key. If $v$ is blank, then the
same can be achieved by encrypting $m$ under each key in $v$'s \emph{resolution}, i.e., the minimal set of non-blank
nodes covering all leaves in $v$'s subtree (Note that leaves are never blank, so there is always a resolution covering
all leaves).

%\saik uses generalized $q$-ary ratchet trees, generalizing binary trees used by \protITK (and all its variants). While in most applications $q=2$ is optimal, choosing $q>2$ is more efficient if \saik is instantiated with (an mmPKE built from) a post-quantum secure mKEM \cite{AC:KKPP20}.

\paragraph{Ratchet tree evolution.}
Each group modification corresponds to a modification of the ratchet tree $\tree$. Most importantly, an update performed by a member $\id$ corresponds to refreshing all key pairs with secret keys known to $\id$, i.e., those in the nodes on the path from $\id$'s leaf to the root. $\id$ generates the new key pairs and, to maintain the tree invariant, communicates the secret keys to some group members. This is done efficiently as follows.
\begin{enumerate}[itemsep=1pt,topsep=1pt,parsep=1pt]
  \item Let $v_1, \dots, v_n$ denote the nodes on the path from $\id$'s leaf $v_1$ to the root $v_n$.
  $\id$ generates a sequence of \emph{path secrets} $s_2, \dots, s_n$: $s_2$ is a random bitstring, $s_{i+1} = \hash(s_i,\literal{path})$.
  \item $\id$ generates a fresh key pair for $v_1$. For each $i\in[2,n-1]$, the new key pair of $v_i$ is computed by
    running the key generation with randomness $\hash(s_i,\literal{rand})$. The last secret $s_n$ will be used in the key schedule, described soon.
  \item $\id$ encrypts each $s_{i+1}$ to the sibling of $v_i$. This allows parties in the subtree of $v_i$ (and only those) to derive $s_{i+1}, \dots, s_n$.
\end{enumerate}

Each add and remove is immediately followed by an implicit update.
Removing a member $\id_t$ corresponds to removing all keys known to it, i.e., blanking all nodes on the path from its leaf to the root.
%
Adding a member $\id_t$ corresponds to inserting a new leaf into $\tree$. The leaf's public signature and encryption keys are fetched from the AKS. Further, the new leaf becomes an \emph{unmerged leaf} of all nodes on the path from it to the root. A leaf $l$ being unmerged at a node $v$ indicates that the $l$'s owner doesn't know the secret key in $v$, so messages should be encrypted directly to $l$. When $v$'s key is refreshed during an update, its set of unmerged leaves is cleared.

\paragraph{Key schedule.}
Apart from the ratchet tree, all group members store a number of shared symmetric keys, unique to the current epoch. These are: the \emph{application secret} --- the group key exported to the E2E application, the \emph{membership key} used to authenticate sent messages and the \emph{init key} --- mixed in the next epoch's secrets for FS.

The secrets are derived when an epoch is created, i.e. after the implicit update following each modification. The update
generates the last path secret $s_n$, which we now call the \emph{commit secret}. Then, the following secrets are
derived. First, the commit sercert and the old epoch's init secrets are hashed together to obtain the \emph{joiner
  secret}. Then, the \emph{epoch secret} is obtained by hashing the joiner secret with the new epoch's \emph{context}, which we explain next. (The context is not mixed directly with init and commit secrets, because the joiner secret is needed by new members; see below.) Finally, the new epoch's application, membership and init secrets are obtained by hashing the epoch secret with different labels.

The context includes all relevant information about the epoch, e.g. (the hash of) the ratchet tree (which includes the member set). The purpose of mixing it into the key schedule is ensuring that parties in different epochs derive independent epoch secrets.

\paragraph{Joining.}
When an $\id_t$ joins a group, the party inviting them encrypts to them two secrets under a key fetched from the AKS. First, this is $\id_t$'s path secret from the implicit update following the add. Second, this is the new joiner secret, from which $\id_t$ derives other epoch secrets.
Importantly, the new member hashes the joiner secret with the context, which means that it agrees on the epoch's state with all current members transitioning to it.
%\begin{table*}[!t]
  \begin{minipage}[t]{.48\textwidth}
  	\begin{tabularx}{\textwidth}{| l | X |}
      \hline
    	$\tree.\rt$ & The root.\\
%    	\hline
%    	$\tree.\nodes$ & The set of all nodes in $\tree$.\\
    	\hline
    	$v.\isroot$ & True iff $v = \tree.\rt$.\\
    	\hline
    	$v.\isleaf$ & True iff $v$ has no children.\\
    	\hline
    	$v.\parent$ & The parent node of $v$ (or $\bot$ if $v.\isroot$).\\
    	\hline
    	$v.\children$ & If $\neg v.\isleaf$: ordered list of $v$'s children.\\
    	\hline
    	$v.\nodeIndex$ & The node index of $v$.\\
      \hline
      $v.\depth$ & The length of the path from $v$ to $\tree.\rt$. \\
  		\hline
  		$v.\mmpkepk$ & An mmPKE encryption key.\\
  		\hline
  		$v.\mmpkesk$ & The corresponding decryption key.\\
  		\hline
      $v.\rsvk$ & If $v.\isleaf$: a signature verification key.\\
  		\hline
      $v.\rssk$ & If $v.\isleaf$: the corresponding signing key.\\
  		\hline
  		$v.\unmergedLeaves$ & The set of indices of the leaves below $v$ whose owner $\id$ does not know $v.\pkesk$.\\
  		\hline
  		$v.\id$ & If $v.\isleaf$: the $\id$ associated with that leaf.\\
  		\hline
  	\end{tabularx}

	\caption{Labels of a ratchet-tree $\tree$ and its nodes.}
	\label{tab:node_labels}
  \end{minipage}
  \hfill
  \begin{minipage}[t]{.48\textwidth}
  	\begin{tabularx}{\textwidth}{| l | X |}
  		\hline
  		$\itkSt.\groupId$ & The identifier of the group.\\
  		\hline
  		$\itkSt.\tree$ & The ratchet tree.\\
  		\hline
  		$\itkSt.\leaf$ & The party's leaf in $\tree$.\\
  		\hline
  		$\itkSt.\treeHash$ & A hash of the public part of $\tree$.\\
  		\hline
      $\itkSt.\lastAct$ & The last modification of the group state and the user who initiated it.\\
  		\hline
  		$\itkSt.\applicationSecret$ & The current epoch's CGKA key. Exposed to the application layer.\\
  		\hline
  		$\itkSt.\initSecret$ & The next epoch's init secret.\\
  		\hline
      $\itkSt.\membershipKey$ & The next epoch's membership secret for authenticating messages.\\
  		\hline
  		$\itkSt.\groupContext()$ & Returns $(\itkSt.\groupId, \itkSt.\treeHash, \itkSt.\lastAct)$.\\
  		\hline
        $\itkSt.\confTag$ & The confirmation tag, which is signed to ensure authenticity.\\
  		\hline
  	\end{tabularx}
    \caption{The protocol state of a party $\id$ and the helper method for computing the context.}
    \label{tab:prot-state}
  \end{minipage}
  \begin{minipage}{.48\textwidth}
    \begin{tabularx}{\textwidth}{| l | X |}
      \hline
      $\pathSecret$ & The path secrets $s_2,\ldots,s_n$ used to derive the keypairs in each node. Sent via the \mmPKE
                      encryption to keep tree invariant intact.\\
      \hline
      $\commitSecret$ & The path secret in the root node. Used as seed for the key schedule together with \initSecret
                        from the previous epoch\\
      \hline
      $\joinerSecret$ & Secret sent to new group members. Together
                        with the group context, enables computation of the $\epochSecret$.\\
      \hline
      $\epochSecret$ & Base secret used to derive all other secrets, i.e. \applicationSecret,
                       \membershipKey,\initSecret, \confTag \\
      \hline
    \end{tabularx}
    \caption{Intermediate values computed by the protocol that are not part of the state.}
    \label{tab:prot-intermediate-state}
  \end{minipage}
  
%  \end{table*}
%\begin{table*}[!t]
  \begin{minipage}{\textwidth}
  	\begin{tabularx}{.48\textwidth}{| l | X |}
  		\hline
  		$\tree.\clone()$ & Returns a copy of $\tree$.\\
  		\hline
  		$\tree.\public()$ & Returns a copy of $\tree$ with all labels $v.\rssk$ and $v.\pkesk$ set to $\bot$.\\
  		\hline
  		$\tree.\roster()$ & Returns $\id$'s of all parties in $\tree$.\\
  		\hline
  		$\tree.\leaves()$ & Returns the list of all leaves in the tree, sorted from left to right.\\
  		\hline
  		$\tree.\leafof(\id)$ & Returns the leaf $v$ with $v.\id = \id$.\\
  		\hline
  		$\tree.\getleaf()$ & Returns leftmost $v$ s.t. $\neg v.\inuse()$. If no such $v$ exists, adds a new leaf using $\addleaf(\tree)$ and returns it.\\
      \hline
      $\tree.\blankpath(v)$ & For all $u\in\tree.\directpath(v)$ calls $u.\blank()$. \\
      \hline
  		$\tree.\inSubtree(u,v)$ & Returns true if $u$ is in $v$'s subtree.\\
  		\hline
  		$v.\inuse()$ & Returns $\false$ iff all labels are $\bot$.\\
  		\hline
  		$v.\blank()$ & Sets all labels of $v$ to $\bot$.\\
  		\hline
  	\end{tabularx}
    \hfill
    \begin{tabularx}{.48\textwidth}{| l | X |}
  		\hline
  		$\tree.\lca(u, v)$ & Returns the lowest common ancestor of the two leafs.\\
  		\hline
  		$\tree.\directpath(v)$ & Returns the path from $v$'s parent to the root.\\
  		\hline
  		$\tree.\mergeleaves(v)$ & Sets $u.\unmergedLeaves \gets \emptyset$ for all $u \in
                                    \tree.\directpath(v)$\\
  		\hline
  		$\tree.\unmergeleaf(v)$ & Sets $u.\unmergedLeaves \setadd v$ for all $u$ returned by $\tree.\directpath(v)$\\
  		\hline
      $v.\resolution()$ &
      If $v.\inuse$, return $(v) \append v.\unmergedLeaves$. Else if $v.\isleaf$, return $()$. Else, return $v.\children[1].\resolution() $ $\append \dots \append v.\children[n].\resolution()$\\
%  		Return
%  		$\begin{cases}
%  			(v) \append v.\unmergedLeaves & \text{if } v.\inuse() \\
%        \term{concatChildResolution}(v) & \text{else if } \neg v.\isleaf \\
%  			\emptylist & \text{else},
%  		\end{cases}$\\
%      &where $\term{concatChildResolution}(v) = v.\children[1].\resolution() \append \dots \append v.\children[n].\resolution()$.\\
  		\hline
  		$v.\resolvent(u)$ & Returns the ancestor of $u$ in $v.\resolution() \setminus (v)$ (or $\bot$ if $u$ is not a descendant of $v$).\\
  		\hline
  	\end{tabularx}
	  \caption{Helper methods for a ratchet tree $\tree$ and its nodes.\vspace{-1em}}
	  \label{tab:node_helpers}
  \end{minipage}
\end{table*}

%%% Local Variables:
%%% mode: latex
%%% TeX-master: "main"
%%% End:

\subsection{Intuition for the \saik Protocol}\label{sec:intuition2}
\paragraph{mmPKE.}
In \protITK, a member performing an update generates a sequence of path secrets $s_1, \dots, s_n$ and encrypts each $s_i$ to each public key from a set of recipient public keys $S_i$ using regular encryption.
In contrast, \saik redraws its internal abstraction boundaries viewing the sequence of encryptions as a single call to mmPKE.
This allows it to use the DDH-based mmPKE construction of
~\cite{ASIACCS:PinPoeSch14}. Compared to \protITK, this cuts the
computational complexity of encrypting $\vec m$ and the ciphertext
size in half (asymptotically as $n$ grows).


\paragraph{Authentication.}
The goal of authentication is to make sure that a member accepts a message from $\id$ only if $\id$ knows 1) the signing key for the verification key stored in $\id$'s leaf in the current ratchet tree and 2) the current key schedule.
In \protITK, where every member gets the same message, this is achieved by simply signing it and MACing with the current membership key.
In \saik, to optimize bandwidth, the mailboxing service forwards to each receiver
only the data it needs. E.g., it does not forward ciphretexts for other members. Therefore, we have to achieve authentication differently.

One trivial solution would be that the sender uploads multiple signatures, one for each receiver. However, this clearly does not scale. Can we do something better?
A crucial observation is that the goal of saCGKA is to authenticate created epochs and \emph{not message
  bitstrings}. That is, we want to guarantee that if Alice thinks that a message $c$ transitions her to an epoch $E$
created by Bob, then Bob indeed created $E$. It is not an attack if the adversary can make Alice accept a message that
is not extracted with the honest procedure (e.g., it has reordered fields), as long as it transitions her to $E$.

Therefore, instead of signing the whole message, in \saik we can sign and MAC only a single short tag that identifies the new epoch and is known to all members. In particular, this value is derived in the key schedule for the new epoch, alongside the other secrets, by hashing the epoch secret with an appropriate label.
%
This way of efficient authentication is enabled by our new security notion.% Note that existing definitions such as \cite{EPRINT:AlwJosMul20} require that the receiver authenticates the whole sent message, which cannot be done without downloading it.



%\subsection{Authenticated Key Service (AKS)}
%Formally, the Authenticated Key Service (AKS) is modeled as a UC functionality $\funcPKI$ (\saik works in the $\funcPKI$). Readers not familiar with UC can think of $\funcPKI$ as a trusted server communicating with instances of \saik executed by different parties, each identified by an $\id$, via secure channels.
%
%$\funcPKI$ is parameterized by a key-package generation algorithm $\genPKIkeys$ (which will be implemented by \saik). Each user $\id$ (more precisely, its protocol instance) can repeatedly send to $\funcPKI$ messages $(\keyword{GetPK}, \id')$. Upon such a message $\funcPKI$ generates a new key package $(\textit{PK}, \textit{SK})$for $\id'$ and sends the public part \textit{PK} to $\id$. $\funcPKI$ is not keeping public keys secret, so it also immediately reveals \textit{PK} to the adversary. Once online, $\id'$ can fetch the corresponding secret key by sending $\keyword{GetSK}(\textit{PK})$ to $\funcPKI$. The details of $\funcPKI$ can be found in \cref{sec:pki}.
%
%We note that in reality $\id'$ would generate the key pair itself and pre-upload it to the server. However, we decided to use an oversimplified PKI (see \cref{sec:simplifications}). So, since $\funcPKI$ is trusted, it does not matter that it generates keys for $\id'$. Further, doing this at the moment $\id$ fetches the keys corresponds to $\id'$ pre-uploading the necessary number of keys before the execution begins.
%
%\subsection{Ratchet Trees}
%% !TEX root = main.tex
% !TeX spellcheck = en_US
A ratchet tree is a left-balanced $q$-ary tree (a formal definition can be found in \cref{sec:addleafprf}). This generalizes \protITK' binary trees. Using $q\neq2$ can be beneficial in certain situations.
A ratchet tree, as well as its nodes, have a number of labels listed in \cref{tab:node_labels}.
We also define a number of helper methods in \cref{tab:node_helpers}.
%
\begin{table*}[!t]
  \begin{minipage}[t]{.48\textwidth}
  	\begin{tabularx}{\textwidth}{| l | X |}
      \hline
    	$\tree.\rt$ & The root.\\
%    	\hline
%    	$\tree.\nodes$ & The set of all nodes in $\tree$.\\
    	\hline
    	$v.\isroot$ & True iff $v = \tree.\rt$.\\
    	\hline
    	$v.\isleaf$ & True iff $v$ has no children.\\
    	\hline
    	$v.\parent$ & The parent node of $v$ (or $\bot$ if $v.\isroot$).\\
    	\hline
    	$v.\children$ & If $\neg v.\isleaf$: ordered list of $v$'s children.\\
    	\hline
    	$v.\nodeIndex$ & The node index of $v$.\\
      \hline
      $v.\depth$ & The length of the path from $v$ to $\tree.\rt$. \\
  		\hline
  		$v.\mmpkepk$ & An mmPKE encryption key.\\
  		\hline
  		$v.\mmpkesk$ & The corresponding decryption key.\\
  		\hline
      $v.\rsvk$ & If $v.\isleaf$: a signature verification key.\\
  		\hline
      $v.\rssk$ & If $v.\isleaf$: the corresponding signing key.\\
  		\hline
  		$v.\unmergedLeaves$ & The set of indices of the leaves below $v$ whose owner $\id$ does not know $v.\pkesk$.\\
  		\hline
  		$v.\id$ & If $v.\isleaf$: the $\id$ associated with that leaf.\\
  		\hline
  	\end{tabularx}

	\caption{Labels of a ratchet-tree $\tree$ and its nodes.}
	\label{tab:node_labels}
  \end{minipage}
  \hfill
  \begin{minipage}[t]{.48\textwidth}
  	\begin{tabularx}{\textwidth}{| l | X |}
  		\hline
  		$\itkSt.\groupId$ & The identifier of the group.\\
  		\hline
  		$\itkSt.\tree$ & The ratchet tree.\\
  		\hline
  		$\itkSt.\leaf$ & The party's leaf in $\tree$.\\
  		\hline
  		$\itkSt.\treeHash$ & A hash of the public part of $\tree$.\\
  		\hline
      $\itkSt.\lastAct$ & The last modification of the group state and the user who initiated it.\\
  		\hline
  		$\itkSt.\applicationSecret$ & The current epoch's CGKA key. Exposed to the application layer.\\
  		\hline
  		$\itkSt.\initSecret$ & The next epoch's init secret.\\
  		\hline
      $\itkSt.\membershipKey$ & The next epoch's membership secret for authenticating messages.\\
  		\hline
  		$\itkSt.\groupContext()$ & Returns $(\itkSt.\groupId, \itkSt.\treeHash, \itkSt.\lastAct)$.\\
  		\hline
        $\itkSt.\confTag$ & The confirmation tag, which is signed to ensure authenticity.\\
  		\hline
  	\end{tabularx}
    \caption{The protocol state of a party $\id$ and the helper method for computing the context.}
    \label{tab:prot-state}
  \end{minipage}
  \begin{minipage}{.48\textwidth}
    \begin{tabularx}{\textwidth}{| l | X |}
      \hline
      $\pathSecret$ & The path secrets $s_2,\ldots,s_n$ used to derive the keypairs in each node. Sent via the \mmPKE
                      encryption to keep tree invariant intact.\\
      \hline
      $\commitSecret$ & The path secret in the root node. Used as seed for the key schedule together with \initSecret
                        from the previous epoch\\
      \hline
      $\joinerSecret$ & Secret sent to new group members. Together
                        with the group context, enables computation of the $\epochSecret$.\\
      \hline
      $\epochSecret$ & Base secret used to derive all other secrets, i.e. \applicationSecret,
                       \membershipKey,\initSecret, \confTag \\
      \hline
    \end{tabularx}
    \caption{Intermediate values computed by the protocol that are not part of the state.}
    \label{tab:prot-intermediate-state}
  \end{minipage}
  
%  \end{table*}
%\begin{table*}[!t]
  \begin{minipage}{\textwidth}
  	\begin{tabularx}{.48\textwidth}{| l | X |}
  		\hline
  		$\tree.\clone()$ & Returns a copy of $\tree$.\\
  		\hline
  		$\tree.\public()$ & Returns a copy of $\tree$ with all labels $v.\rssk$ and $v.\pkesk$ set to $\bot$.\\
  		\hline
  		$\tree.\roster()$ & Returns $\id$'s of all parties in $\tree$.\\
  		\hline
  		$\tree.\leaves()$ & Returns the list of all leaves in the tree, sorted from left to right.\\
  		\hline
  		$\tree.\leafof(\id)$ & Returns the leaf $v$ with $v.\id = \id$.\\
  		\hline
  		$\tree.\getleaf()$ & Returns leftmost $v$ s.t. $\neg v.\inuse()$. If no such $v$ exists, adds a new leaf using $\addleaf(\tree)$ and returns it.\\
      \hline
      $\tree.\blankpath(v)$ & For all $u\in\tree.\directpath(v)$ calls $u.\blank()$. \\
      \hline
  		$\tree.\inSubtree(u,v)$ & Returns true if $u$ is in $v$'s subtree.\\
  		\hline
  		$v.\inuse()$ & Returns $\false$ iff all labels are $\bot$.\\
  		\hline
  		$v.\blank()$ & Sets all labels of $v$ to $\bot$.\\
  		\hline
  	\end{tabularx}
    \hfill
    \begin{tabularx}{.48\textwidth}{| l | X |}
  		\hline
  		$\tree.\lca(u, v)$ & Returns the lowest common ancestor of the two leafs.\\
  		\hline
  		$\tree.\directpath(v)$ & Returns the path from $v$'s parent to the root.\\
  		\hline
  		$\tree.\mergeleaves(v)$ & Sets $u.\unmergedLeaves \gets \emptyset$ for all $u \in
                                    \tree.\directpath(v)$\\
  		\hline
  		$\tree.\unmergeleaf(v)$ & Sets $u.\unmergedLeaves \setadd v$ for all $u$ returned by $\tree.\directpath(v)$\\
  		\hline
      $v.\resolution()$ &
      If $v.\inuse$, return $(v) \append v.\unmergedLeaves$. Else if $v.\isleaf$, return $()$. Else, return $v.\children[1].\resolution() $ $\append \dots \append v.\children[n].\resolution()$\\
%  		Return
%  		$\begin{cases}
%  			(v) \append v.\unmergedLeaves & \text{if } v.\inuse() \\
%        \term{concatChildResolution}(v) & \text{else if } \neg v.\isleaf \\
%  			\emptylist & \text{else},
%  		\end{cases}$\\
%      &where $\term{concatChildResolution}(v) = v.\children[1].\resolution() \append \dots \append v.\children[n].\resolution()$.\\
  		\hline
  		$v.\resolvent(u)$ & Returns the ancestor of $u$ in $v.\resolution() \setminus (v)$ (or $\bot$ if $u$ is not a descendant of $v$).\\
  		\hline
  	\end{tabularx}
	  \caption{Helper methods for a ratchet tree $\tree$ and its nodes.\vspace{-1em}}
	  \label{tab:node_helpers}
  \end{minipage}
\end{table*}

%%% Local Variables:
%%% mode: latex
%%% TeX-master: "main"
%%% End:


Importantly, the \emph{direct path} of a leaf $u$ consists of (the ordered list of) all nodes on the path from $u$ to the root, without $u$.
The \emph{resolution} of a node $v$ is the minimal set of descendant non-blank nodes that covers the whole sub-tree rooted at $v$. % i.e., such that for every descendant $u$ of $v$ there exists node $w$ in the resolution such that $w$ is non-blank and $w$ an ancestor of $u$.


%%% Local Variables:
%%% mode: latex
%%% TeX-master: "main"
%%% End:

%
%\subsection{\saik State and Algorithms}
%% !TEX root = main.tex
% !TeX spellcheck = en_US

The state of $\saik$ consists of a number of variables, outlined in \cref{tab:prot-state}. The table also includes short
descriptions of the roles of the secrets in the key schedule. The protocol will ensure that states of any two parties in
the same epoch differ at most in labels of nodes of $\itkSt.\tree$ that describe secret keys and the label
$\itkSt.\leaf$. This means that they agree on the secrets $\itkSt.\applicationSecret$ and $\itkSt.\initSecret$, as well
as on the public context, computed by the helper method $\groupContext()$ in \cref{tab:prot-state}.



%%% Local Variables:
%%% mode: latex
%%% TeX-master: "main"
%%% End:

%
%%\subsection{\saik Algorithms}
%\saik's algorithms are defined in \cref{fig:prot1,fig:prot-helpers2}. Apart from initialization, there are three main algorithms (the rest of the code are subroutines) exposed to a user (or a higher-level application). They are identified by keywords \keyword{Send}, \keyword{Receive} and \keyword{Key}, respectively. First, \keyword{Send} is used to create a new epoch. When the user inputs \keyword{Send} followed by the intended group modification (update, add or remove), the protocol applies the modification and returns a message, which the user can upload to the mailboxing service. Second, \keyword{Receive} is used to process messages downloaded from the mailboxing service. Third, using \keyword{Key} user obtains the current group key.
%
%The formal syntax of saCGKA protocols is defined as part of our security definition in \cref{sec:model}. In particular, an saCGKA protocol must expose the same interface as the ideal CGKA functionality.
%
%% !TEX root = main.tex
% !TeX spellcheck = en_US


\begin{figure*}[!p]%\vspace*{-2em}\hspace*{-1.5em}
%  \begin{minipage}{\linewidth+2em}
	\begin{anybox}{\sffamily\bfseries \saik : Algorithms}
			\begin{minipage}[t]{0.48\linewidth}
        {\bf Initialization}
        \begin{algorithmic}
          \If{$\id = \pgod$}
            \State $\itkSt \gets \method{new-state}()$
            \State $\itkSt.\groupId, \itkSt.\initSecret, \itkSt.\membershipKey, \itkSt.\applicationSecret \getsr \bits^\secparam$
            \State $\itkSt.\tree \gets \method{new-LBT}()$
            \State $\itkSt.\leaf \gets \itkSt.\tree.\leaves[0]$
            \State $(\itkSt.\leaf.\rsvk, \itkSt.\leaf.\rssk) \gets \sigkg()$
          \EndIf
        \end{algorithmic}

        \medskip
        {\bf Input $(\keyword{Send}, \hgact), \hgact \in \{\hglu , \hglr\md\id_t, \hgla\md\id_t\}$ from $\id$}
				\begin{algorithmic}
					\State $\KwReq\ \itkSt \neq \bot$
          \State \vspace*{-.5em}\Comment{In case of add, fetch $\id_t$'s keys from AKS (AKS runs $\genPKIkeys$).}
          \If{$\hgact = \hgla\md\id_t$}
            \State $(\mmpkepk_t, \rsvk_t, \mmpkepk_t') \gets \KwQuery\ (\keyword{GetPk},\id_t) \text{ to } \funcKB$
            \State $\hgact \gets \hgla\md\id_t\md(\mmpkepk_t, \rsvk_t,\mmpkepk_t')$
          \EndIf
          \State \smashedComment{Create the state and secrets for the new epoch.}
          \State \KwTry{} $(\itkSt', \pathSecrets, \joinerSecret)  \gets\provState(\hgact)$
          \State \Comment{Encrypt the path secrets using the new epoch's ratchet tree. For adds, also encrypt the joiner secret.}
          \If{$\hgact \in \{\hglu , \hglr\md\id_t\}$}
            \State $\pathSecCtxt \gets \encSecrets(\itkSt', \pathSecrets, \bot, \bot, \bot)$
          \ElsIf{$\hgact = \hgla\md\id_t\md(\mmpkepk_t, \rsvk_t,\mmpkepk_t')$}
            \State $\pathSecCtxt \gets \encSecrets(\itkSt', \pathSecrets, \id_t, \mmpkepk_t', \joinerSecret)$
          \EndIf
%          \State \smashedComment{Sign data under current epoch's secrets.}
%          \State $\updatedPks \gets ((\itkSt'.\leaf.\mmpkepk, \itkSt'.\leaf.\rsvk))$
%          \State $\updatedPks \gets \updatedPks \concat (v.\mmpkepk : v \in \itkSt'.\tree.\directpath(\itkSt'.\leaf))$
%          \State $(\vec{\variable{tbs}}, \rdclass) \gets \method{to-be-signed}(\pathSecCtxt, \updatedPks, \hgact)$
%          \State $\sig \gets \rssignL(\itkSt.\tree.\leafof(\id).\rssk, \itkSt.\membershipKey, \vec{\variable{tbs}}, \rdclass)$
%          \State $\itkSt \gets \itkSt'$
          \State $\ssk \gets \itkSt.\tree.\leafof(\id).\ssk$
          \State  $\sig \gets \sigsign(\ssk, \underline{\itkSt'.\confTag})$
          \If{$\hgact = \hglr\md\id_t$}
            \State \smashedComment{Authenticate removal message for $\id_t$}
            \State $\sig_t \gets \sigsign(\ssk, (\id, \hglr\md\id_t))$
            \State $\macsig_t \gets \mactag(\itkSt.\membershipKey, (\id, \hglr\md\id_t, \underline{\itkSt.\confTag}))$
            \State \Return $(\id, \hgact, \underline{\pathSecCtxt}, \updatedPks, \sig, \sig_t, \macsig_t)$
          \EndIf
          \State $\itkSt \gets \itkSt'$
          \If{$\hgact = \hgla\md\id_t\md(\mmpkepk_t, \rsvk_t,\mmpkepk_t')$}
            \State \smashedComment{Send additional data for $\id_t$.}
            \State $\variable{welcomeData} \gets (\itkSt.\groupId, \itkSt.\tree.\public(), \mmpkepk_t')$
            \State \Return $(\id, \hgact, \underline{\pathSecCtxt}, \updatedPks, \sig, \variable{welcomeData})$
          \EndIf
          \State \Return $(\id, \hgact, \underline{\pathSecCtxt}, \updatedPks, \sig)$
				\end{algorithmic}

      \end{minipage}
		\hfill
			\begin{minipage}[t]{0.5\linewidth}
        \bf Input {$\keyword{Key}$ from $\id$}
				\begin{algorithmic}
					\State $\KwReq\ \itkSt \neq \bot$
					\State $k \gets \itkSt.\applicationSecret$
					\State $\itkSt.\applicationSecret \gets \bot$
					\State \Return $k$
				\end{algorithmic}

        \medskip
        {\bf Input $(\keyword{Receive}, (\id_s, \literal{removed}, \sig_t, \macsig_t))$ from $\id$}

        \textnormal{\smashedComment{Receiver is removed.}}
        \begin{algorithmic}
          \State $\ersvk \gets \itkSt.\tree.\leafof(\id_s).\rsvk$
          \State \KwReq{} $\sigvrf(\ersvk, (\id_s, \hglr\md\id), \sig_t)$
          \State \KwReq{} $\macvrf(\itkSt.\membershipKey, (\id_s, \hglr\md\id, \underline{\itkSt.\confTag}), \macsig_t)$
%          \State $\vec{\variable{tbv}} \gets ((\id_s, \hglr\md\id, \itkSt.\groupId))$
%          \State \vspace*{-.5em}\Comment{\normalfont Check if removing allowed and compute the reduction pattern $\rd=(\ell,0,1)$.}
%          \State \KwTry{} $\itkSt' \gets \applyact(\itkSt.\clone(), \id_s, \hglr\md\id)$
%          \State $\ell \gets \getWeights(\itkSt', \id_s)$
%          \State $\ersvk \gets \itkSt.\tree.\leafof(\id_s).\rsvk$
%          \State \KwReq{} $\rsvrfyL(\rsvk, \itkSt.\membershipKey, \vec{\variable{tbv}}, (\ell,0,1), \redactedSig)$
          \State $\itkSt \gets \bot$
          \State \Return $(\id_s, \hglr\md\id)$
        \end{algorithmic}

        \medskip
        \textbf{Input $(\keyword{Receive}, (\id_s, \hgact, \underline{\pathSecCtxtInd}, \underline{\redactedUpPks}, \sig))$ from $\id$}

        \textnormal{\smashedComment{Receiver is a member.}}
        \begin{algorithmic}
          \State \KwTry{} $\itkSt' \gets \applyact(\itkSt.\clone(), \id_s, \hgact)$
%          \State \smashedComment{\normalfont Get the expected reduction pattern using the new state.}
%          \State \KwTry{} $\rd \gets \myReduction(\itkSt'.\tree, \id_s)$
%          \State $\vec{\variable{tbv}} \gets (\pathSecCtxtInd) \concat ((\id_s, \hgact, \itkSt.\groupId)) \concat \redactedUpPks$
%          \State $\ersvk \gets \itkSt.\tree.\leafof(\id_s).\rsvk$
%          \State \KwReq{} $\rsvrfyL(\rsvk, \itkSt.\membershipKey, \vec{\variable{tbv}}, \rd, \redactedSig)$
%          \State \smashedComment{\normalfont Transition to next epoch.}
          \State \KwTry{} $(\itkSt, \confTag) \gets \nextState(\itkSt', \underline{\pathSecCtxtInd}, \underline{\redactedUpPks}, \id_s, \hgact)$
          \State $\ersvk \gets \itkSt.\tree.\leafof(\id_s).\rsvk$
          \State \KwReq{} $\sigvrf(\ersvk, \underline{\confTag}, \sig)$
          \If{$\hgact=\hgla\md\id_t\md(\mmpkepk_t,\rsvk_t)$}
            \Return $(\id_s, \hgla\md\id_t)$
          \Else\
            \Return $(\id_s, \hgact)$
          \EndIf
        \end{algorithmic}

        \medskip
        \textbf{Input $(\keyword{Receive}, (\id_s, \hgact, \encGroupSecret_1, \encGroupSecret_2, \variable{welcomeData})))$ from $\id$}

        \textnormal{\smashedComment{Receiver joins.}}
        \begin{algorithmic}
          \State $\KwReq\ \itkSt = \bot$
          \State \KwParse{} $(\groupId, \tree, \mmpkepk') \gets \variable{welcomeData}$
          \State $\itkSt \gets \method{new-state}$
          \State $(\itkSt.\groupId, \itkSt.\tree, \itkSt.\lastAct) \gets (\groupId, \tree, (\id_s, \hgla\md\id))$
          \State $v \gets \itkSt.\tree.\leafof(\id)$
          \State \KwTry{} $(\mmpkesk, \rssk, \mmpkesk') \gets \KwQuery\ \keyword{GetSk}((v.\mmpkepk, v.\rsvk, \mmpkepk')) \textnormal{ to } \funcKB$
          \State $(v.\mmpkesk, v.\rssk) \gets (\mmpkesk, \rssk)$
          \State $\itkSt \gets \setTreeHash(\itkSt)$
          \State \KwTry{} $(\itkSt, \confTag) \gets \populateSecrets(\itkSt, \mmpkesk', \encGroupSecret_1, \encGroupSecret_2, \id_s)$
%          \State $\ersvk \gets \itkSt.\tree.\leafof(\id_s).\rsvk$
%          \State \KwReq{} $\sigvrf(\ersvk, \confTag)$
					\State \Return $(\itkSt.\tree.\roster(), \id_s)$
        \end{algorithmic}
      \end{minipage}
  \end{anybox}

  \medskip
  \begin{anybox}{\sffamily\bfseries \saik : Helpers for encryption and key generation for $\funcPKI$}
			\begin{minipage}[t]{.49\linewidth}
        {\bf {helper $\encSecrets(\itkSt', \pathSecrets, \id_t, \mmpkepk_t', \joinerSecret)$}}
				\begin{algorithmic}
          \State $L \gets \getPathSecsMap(\itkSt'.\tree, \id)$
          \State $\vec m, \vec \mmpkepk \gets ()$
          \For{$j=1$ \bf to $\len(L)$}
            \State $(i, v) \gets L[j]$
            \State $\vec m \listapp \pathSecrets[i]$
            \If{$\id_t\neq\bot \land v = \itkSt'.\tree.\leafof(\id_t)$}
              $\vec \mmpkepk \listapp \mmpkepk_t'$
            \Else\
              $\vec \mmpkepk \listapp \vec v.\mmpkepk$
            \EndIf
          \EndFor
          \If{$\id_t\neq\bot$}
            \State $\vec m \listapp \joinerSecret$
            \State $\vec\mmpkepk \listapp \mmpkepk_t'$
          \EndIf
          \State \Return $\underline{\mmpkeEncL}(\vec{\mmpkepk}, \vec m)$
        \end{algorithmic}
      \end{minipage}
		\hfill
			\begin{minipage}[t]{0.49\linewidth}
        {\bf {helper $\decSecrets(\itkSt', \id_s, \pathSecCtxtInd)$}}
				\begin{algorithmic}
          \State $v \gets \lca(\itkSt'.\tree.\leafof(\id_s), \itkSt'.\leaf).\resolvent(\itkSt'.\leaf)$
          \State \Return $\underline{\mmpkeDecL}(v.\mmpkesk, \pathSecCtxtInd)$
        \end{algorithmic}

        \medskip
        {\bf {helper $\genPKIkeys()$}}
				\begin{algorithmic}
          \State $(\mmpkepk, \mmpkesk) \gets \underline{\mmpkeKeyGenL}()$
          \State $(\rsvk, \rssk) \gets \sigkg()$
          \State $(\mmpkepk', \mmpkesk') \gets \underline{\mmpkeKeyGenL}()$
          \State \Return $((\mmpkepk, \rsvk, \mmpkepk'), (\mmpkesk, \rssk, \mmpkesk'))$
        \end{algorithmic}
      \end{minipage}
  \end{anybox}
	\vspace*{-0.7em}
	\caption{The algorithms of \saik.}
	\label{fig:prot1}
%  \end{minipage}
\end{figure*}


%\begin{figure}[!tpb]\vspace*{-5em}\hspace*{-1.5em}
%  \begin{minipage}{\linewidth+2em}
%	\begin{anybox}{\sffamily\bfseries \saik : Helpers for encryption and authentication}
%		\scalebox{0.7}{
%			\begin{minipage}[t]{.6\linewidth}
%        {\bf {helper $\encSecrets(\itkSt', \pathSecrets, \id_t, \mmpkepk_t', \joinerSecret)$}}
%				\begin{algorithmic}
%          \State $L \gets \getPathSecsMap(\itkSt'.\tree, \id)$
%          \State $\vec m, \vec \mmpkepk \gets ()$
%          \For{$j=1$ \bf to $\len(L)$}
%            \State $(i, v) \gets L[j]$
%            \State $\vec m \listapp \pathSecrets[i]$
%            \If{$\id_t\neq\bot \land v = \itkSt'.\tree.\leafof(\id_t)$}
%              $\vec \mmpkepk \listapp \mmpkepk_t'$
%            \Else\
%              $\vec \mmpkepk \listapp \vec v.\mmpkepk$
%            \EndIf
%          \EndFor
%          \If{$\id_t\neq\bot$}
%            \State $\vec m \listapp \joinerSecret$
%            \State $\vec\mmpkepk \listapp \mmpkepk_t'$
%          \EndIf
%          \State \Return $(\mmpkeEncL(\vec{\mmpkepk}, \vec m))$
%        \end{algorithmic}
%
%        \medskip
%        {\bf {helper $\decSecrets(\itkSt', \id_s, \pathSecCtxtInd)$}}
%				\begin{algorithmic}
%          \State $v \gets \lca(\itkSt'.\tree.\leafof(\id_s), \itkSt'.\leaf).\resolvent(\itkSt'.\leaf)$
%          \State \Return $\mmpkeDecL(v.\mmpkesk, \pathSecCtxtInd)$
%        \end{algorithmic}

%        \medskip
%        {\bf \mbox{helper $\method{to-be-signed}(\itkSt', \pathSecCtxt, \updatedPks, \hgact)$}}
% 				\begin{algorithmic}
%          \State $\vec w \gets \getWeights(\itkSt'.\tree, \id)$
%          \State $\vec{\variable{tbs}} \gets ()$
%          \For{$j=1$ \bf to $\abs{\vec w}$}
%            \State $\vec{\variable{tbs}} \listapp \mmpkeExtL(\pathSecCtxt,j)$
%          \EndFor
%          \State $\vec{\variable{tbs}} \listapp (\id, \hgact, \itkSt.\groupId)$
%          \State $\vec{\variable{tbs}} \listapp \updatedPks$
%          \State \Return $(\vec{\variable{tbs}}, \rdclassBGM_{\abs{\vec w},\vec w})$
%        \end{algorithmic}
%      \end{minipage}}
%		\hfill\scalebox{.7}{
%			\begin{minipage}[t]{0.75\linewidth}
%        {\bf \mbox{helper $\myReduction(\tree', \id_s)$}}
% 				\begin{algorithmic}
%          \State $L \gets \getPathSecsMap(\tree', \id_s)$
%          \For{$j=1$ \bf to $\len(L)$}
%            \State $(i, v) \gets L[j]$
%            \If{$\itkSt'.\tree.\inSubtree(\tree'.\leafof(\id), v)$}
%              \State \vspace*{-.5em}\Comment{We want the $j$-th ciphertext out of $\len(L)$ and the first $i+1$ items on the prefix list: the aux data, the leaf $\mmpkepk$ and $i-1$ $\mmpkepk$'s on $\id_s$'s direct path.}
%              \State \Return $(\len(L), j, i)$
%            \EndIf
%          \EndFor
%          \State \Return $\bot$
%        \end{algorithmic}

%        \medskip
%        {\bf {helper $\getPathSecsMap(\tree', \id_s)$}}
%				\begin{algorithmic}
%          \State\vspace*{-.7em}{\Comment{Returns a list of tuples $(i,v)$, denoting that when $\id_s$ commits in $\tree'$, the $i$-th path secret is encrypted under node $v$'s keys.}}
%          \State $L \gets ()$
%          \State $\fullpath \gets (\tree'.\leafof(\id_s)) \concat \tree'.\directpath(\tree'.\leafof(\id_s))$
%          \For{$i=1$ \bf to $\len(\fullpath)-1$}
%            \State $\vec v \gets \fullpath[i+1].\resolution() \setminus \fullpath[i].\resolution()$
%            \For{$j=1$ \bf to $\abs{\vec v}$}
%              \State $L \listapp (i, \vec v[j])$
%            \EndFor
%          \EndFor
%          \State \Return $L$
%        \end{algorithmic}

%        \medskip
%        {\bf {helper $\getWeights(\tree', \id_s)$}}
%				\begin{algorithmic}
%          \State\vspace*{-.7em}{\Comment{Returns a list of weights for $\rdclassBGM_{\ell,\vec w}$ when $\id_s$ commits in $\tree'$. Also allows to compute $\ell = \abs{\vec w}$.}}
%          \State $\vec w \gets ()$
%          \State $\fullpath \gets (\tree'.\leafof(\id_s)) \concat \tree'.\directpath(\tree'.\leafof(\id_s))$
%          \For{$i=2$ \bf to $\len(\fullpath)$}
%            \State $\vec v \gets \fullpath[i].\resolution() \setminus \fullpath[i-1].\resolution()$
%            \For{$j=1$ \bf to $\abs{\vec w}$}
%              \State \mbox{$\vec w \listapp  \big|\{u \in  \tree'.\leaves \mid u.\inuse() \land \tree'.\inSubtree(u,\vec v[j]) \}\big|$}
%            \EndFor
%          \EndFor
%          \State \Return $\vec w$
%        \end{algorithmic}
%      \end{minipage}}
%  \end{anybox}
%	\vspace*{-0.7em}
%	\caption{The algorithms of \saik.}
%	\label{fig:prot}
%  \end{minipage}
%\end{figure}


\begin{figure*}[!p]
  	\begin{anybox}{\sffamily\bfseries \saik : Creating epochs}
			\begin{minipage}[t]{.49\linewidth}
        {\bf {helper $\provState(\itkSt, \id, \hgact)$}}
				\begin{algorithmic}
          \State $\itkSt' \gets \itkSt.\clone()$
          \State \vspace*{-.7em}\Comment{Apply the action to the tree. Fails if the action is not allowed.}
          \State $\KwTry\ \itkSt' \gets \applyact(\itkSt', \id, \hgact)$
          \State \smashedComment{Re-key the direct path.}
          \State $\directpath \gets \itkSt'.\tree.\directpath(\itkSt'.\leaf)$
          \State $\pathSecrets[\wc] \gets \bot$
          \State $\pathSecrets[1] \getsr \{0,1\}^\secparam$
					\For{$i=1$ \textbf{to} $\len(\directpath)-1$}
            \State $v \gets \directpath[i]$
					  \State $r \gets \hkdfexp(\pathSecrets[i], \literal{node})$
            \State $(v.\mmpkepk, v.\mmpkesk) \gets \mmpkeKeyGenL(r)$
            \State $\pathSecrets[i+1] \gets \hkdfexp(\pathSecret[i], \literal{path})$
          \EndFor
          \State $\itkSt'.\tree.\mergeleaves(\itkSt'.\leaf)$
          \State \smashedComment{Re-key the leaf.}
          \State $(\itkSt'.\leaf.\mmpkepk, \itkSt'.\leaf.\mmpkesk) \gets \mmpkeKeyGenL()$
%          \State $(\itkSt'.\leaf.\rsvk, \itkSt'.\leaf.\rssk) \gets \rskeygenL()$
          \State $(\itkSt'.\leaf.\rsvk, \itkSt'.\leaf.\rssk) \gets \sigkg()$
          \State \vspace*{-.5em}\Comment{Set all context variables and then derive epoch secrets.}
          \State $\itkSt'.\lastAct \gets (\id, \hgact)$
          \State $\itkSt' \gets \setTreeHash(\itkSt')$
          \State $\commitSecret \gets \pathSecrets[\len(\pathSecrets)]$
          \State $(\itkSt', \joinerSecret) \gets \deriveKeys(\itkSt', \commitSecret)$
          \State \Return $(\itkSt', \pathSecrets, \joinerSecret)$
        \end{algorithmic}

        \medskip
        {\bf {helper $\applyact(\itkSt', \id_s, \hgact)$}}
				\begin{algorithmic}
          \State \KwReq{} $\id_s \in \itkSt'.\tree.\roster()$
          \If{$\hgact=\hglr\md\id_t$}
            \State \KwReq{} $\id_s \neq \id_t \land \id_t \in \itkSt'.\tree.\roster()$
            \State $\itkSt'.\tree.\blankpath(\itkSt'.\tree.\leafof(\id_t))$
            \State $\itkSt'.\tree.\leafof(\id_t).\blank()$
          \ElsIf{$\hgact=\hgla\md\id_t\md(\mmpkepk_t, \rsvk_t)$}
            \State \KwReq{} $\id_t \notin \itkSt'.\tree.\roster()$
            \State $v \gets \itkSt'.\tree.\getleaf()$
            \State $(v.\id, v.\mmpkepk, v.\rsvk) \gets (\id_t, \mmpkepk_t, \rsvk_t)$
            \State $\itkSt.\tree.\unmergeleaf(v)$
          \EndIf
        \end{algorithmic}

      \end{minipage}
		\hfill
			\begin{minipage}[t]{0.49\linewidth}
        {\bf {helper $\nextState(\itkSt', \pathSecCtxtInd, \redactedUpPks, \id_s, \hgact)$}}
				\begin{algorithmic}
          \State \smashedComment{Set keys on the re-keyed path.}
          \State $v_s \gets \itkSt'.\tree.\leafof(\id_s)$
          \State $\directpath \gets \itkSt'.\tree.\directpath(v_s)$
          \State $(v_s.\mmpkepk, v_s.\rsvk) \gets \redactedUpPks[1]$
          \State $i \gets 1$
          \While{$\directpath[i] \notin \{\itkSt'.\tree.\lca(\itkSt'.\leaf, v_s), \itkSt'.\tree.\rt\}$}
            \State \smashedComment{If message contains too few ek's, reject it.}
            \State \KwReq{} $i+1\leq\len(\redactedUpPks)$
            \State $\directpath[i].\mmpkepk \gets \redactedUpPks[i+1]$
            \State $i\inc$
          \EndWhile
          \State \smashedComment{Decrypt the path secret using the updated tree.}
          \State \KwTry{} $\pathSecret \gets \decSecrets(\itkSt', \id_s, \pathSecCtxtInd)$
          \While{$i < \len(\directpath)$}
            \State $v \gets \directpath[i]$
					  \State $r \gets \hkdfexp(\pathSecrets[i], \literal{node})$
            \State $(v.\pkpk, v.\pksk) \gets \mmpkeKeyGenL(r)$
            \State $\pathSecret \gets \hkdfexp(\pathSecret, \literal{path})$
            \State $i\inc$
          \EndWhile
          \State $\commitSecret \gets \pathSecret$
          \State $\itkSt'.\tree.\mergeleaves(v_s)$
          \State \smashedComment{Set all context variables and then derive epoch secrets.}
          \State $\itkSt'.\lastAct \gets (\id_s, \hgact)$
          \State $\itkSt' \gets \setTreeHash(\itkSt')$
          \State $(\itkSt', \joinerSecret)  \gets \deriveKeys(\itkSt', \commitSecret)$
          \State \Return $\itkSt'$
        \end{algorithmic}

        \medskip
        {\bf {helper $\populateSecrets(\itkSt', \pksk', \encGroupSecret_1, \encGroupSecret_2, \id_s)$}}
				\begin{algorithmic}
					\State \KwTry{} $\pathSecret \gets \underline{\mmpkeDecL}(\pksk, \encGroupSecret_1)$
          \State \KwTry{} $\joinerSecret \gets \underline{\mmpkeDecL}(\pksk, \encGroupSecret_2)$
					\State $v \gets \itkSt'.\tree.\lca(\itkSt'.\leaf, \itkSt'.\tree.\leafof(\id_s))$
					\While{$v \neq \itkSt'.\tree.\rt$}
					  \State $r \gets \hkdfexp(\pathSecret, \literal{node})$
					  \State $(\mmpkepk, v.\mmpkesk) \gets \underline{\mmpkeKeyGenL}(r)$
					  \State $\KwReq\ v.\mmpkepk = \mmpkepk$
					  \State $\pathSecret \gets \hkdfexp(\pathSecret, \literal{path})$
					  \State $v \gets v.\parent$
					\EndWhile
          \State $\itkSt' \gets \deriveEpochKeys(\itkSt', \joinerSecret)$
          \State \Return $\itkSt'$
        \end{algorithmic}
      \end{minipage}
  \end{anybox}
%\caption{}\label{fig:prot-helpers1}
%\end{figure}
%\begin{figure}[tbp]
	\begin{tcbraster}[raster columns=2, raster equal height]
		\begin{anybox}{\sffamily\bfseries \saik : Key schedule}
				\begin{minipage}[t]{\linewidth}
					{\bf {helper $\deriveKeys(\itkSt, \itkSt', \commitSecret)$}}
					\begin{algorithmic}
						\State $\joinerSecret \gets \hkdfext(\itkSt.\initSecret, \commitSecret)$
						\State $\itkSt' \gets \deriveEpochKeys(\itkSt', \joinerSecret)$
						\State \Return $\itkSt', \joinerSecret$
					\end{algorithmic}

					\medskip
					{\bf {helper $\deriveEpochKeys(\itkSt', \joinerSecret)$}}
					\begin{algorithmic}
            \State $\epochSecret \gets \hkdfext(\joinerSecret, \itkSt'.\groupContext())$

						\State $\itkSt'.\applicationSecret \gets \hkdfexp(\epochSecret, \literal{app})$
						\State $\itkSt'.\membershipKey \gets \hkdfexp(\epochSecret, \literal{membership})$
						\State $\itkSt'.\initSecret \gets \hkdfexp(\epochSecret, \literal{init})$
            \State $\itkSt'.\confTag \gets \hkdfexp(\epochSecret, \literal{confirmation})$
						\State \Return $\itkSt'$
					\end{algorithmic}
				\end{minipage}
		\end{anybox}
		%
		\begin{anybox}{\sffamily\bfseries \saik : Tree hash}
				\begin{minipage}[t]{\linewidth}
					{\bf {helper $\setTreeHash(\itkSt')$}}
        \begin{algorithmic}
					\State $\itkSt'.\treeHash \gets \computeTreeHash(\itkSt'.\tree.\rt)$
					\State \Return $\itkSt'$
				\end{algorithmic}

        \medskip
        {\bf {helper $\computeTreeHash(v)$}}
				\begin{algorithmic}
					\If{$v.\isleaf$}
					  \State \Return $\hash(v.\nodeIndex, v.\mmpkepk, v.\ersvk)$
					\Else
            \State $\ell \gets \len(v.\children)$
            \For{$i\in[\ell]$}
  					  $h_i \gets \computeTreeHash(v.\children[i])$
            \EndFor
            \State $h \gets (h_1, \dots, h_\ell)$
            \State \Return $\hash(v.\nodeIndex, v.\mmpkepk, v.\unmergedLeaves, h)$
					\EndIf
				\end{algorithmic}
				\end{minipage}
		\end{anybox}
	\end{tcbraster}

	\caption{Additional helper methods for \saik.}
	\label{fig:prot-helpers2}
%\end{minipage}
\end{figure*}

%%% Local Variables:
%%% mode: latex
%%% TeX-master: "main"
%%% End:



\paragraph{Extraction procedure for the server.}
The task of the mailboxing server is to extract a personalized packet for a group member Alice from a packet $C$
uploaded by another member Bob. In \saik, $C$ consists roughly of a single mmPKE ciphertext, a signature, the new public
keys on the path from the sender to the root node and some metadata such as the sender's identity, the group
modification being applied etc. The signature and metadata are simply forwarded to Alice. For the mmPKE ciphertext, the
server runs the mmPKE $\mmpkeExt$ procedure with Alice's recipient index $i$ and also sends all public keys up to the
lowest common ancestor (\lca) of Alice and Bob in the ratchet tree. See \cref{fig:extract-one} for an illustration.
Observe that $i$ and the \lca are determined by the current epoch's ratchet tree and the positions of Alice and Bob in
it. Therefore, the server can obtain $i$ and the \lca in two ways: First, it can store all ratchet trees it needs
(identified by the transcript hash leading to the epoch for which a tree is stored) and them itself. Second, it can ask
Alice for $i$ and the \lca given that the sender is Bob. We note that the latter solution requires an additional round of interaction which may be problematic for some applications.
%
%\newcommand{\extract}{\method{extract}}
%\subsection{Extraction Procedure for the Server}
%Finally, we describe a procedure $\extract(C, \id) \to c$ used by the mailboxing service to take an uploaded message $C$ and compute the message $c$ delivered to user $\id$. Formally, this procedure is not part of our syntax or security definitions, since for simplicity our model does not consider correctness (see \cref{sec:simplifications}) and an untrusted service can anyway deliver arbitrary messages.
%
%Recall that $C$ contains the executed group operation $\hgact$ and the sender $\id_s$, a multi-recipient ciphertext $Ctxt$ and a vector of updated public keys $\updatedPks$. Roughly, $\extract$ only needs to compute $\id$'s individual mmPKE ciphertext $\mmpkeExtL(Ctxt, i)$ and the prefix of the first $j$ elements of $\updatedPks$. This requires that it knows the indices $i$ and $j$ for $\id$.
%%
%We notice that they can be easily computed using the public part of the ratchet tree, $\hgact$ and $\id_s$. Therefore, the indices can be obtained in two ways. First, the service can send $\hgact$ and $\id_s$ to $\id$, who replies with $i$ and $j$.  This requires interaction, but both $\id$ and the service are online at the time. Second, the service can store the current ratchet trees and compute $i$ and $j$ itself. The disadvantage of this is that it requires keeping a large state --- in case members are out of sync (e.g. a user is 10 epochs behind), the service needs to store one tree for each epoch which has an active member in it.
%%
%Once $i$ and $j$ are known, $\extract$ works as follows.
%
%If $\hgact = \hglu$, set  $ctxt = \mmpkeExtL(Ctxt, i)$ and $\redactedUpPks = (\updatedPks[1], \dots, \updatedPks[j])$. Output $c =  (\id_s, \hgact, \pathSecCtxtInd,\allowbreak  \redactedUpPks,\allowbreak  \sig)$, where $\sig$ is a field of $C$. If $\hgact = \hglr\md\id$, output $c =  (\id_s, \literal{removed}, \sig_t, \macsig_t)$ where $ \sig_t$ and $\macsig_t$ are taken from $C$. Finally, if $\hgact = \hgla\md\id$, $C$ contains $\variable{welcomeData}$, which in turn contains a ratchet tree. Based on this, compute $\id$'s index $i$ in $Ctxt$,  the number $n$ of recipients of $Ctxt$, and then $ctxt_1 = $ $\mmpkeExtL(Ctxt, i)$ and $ctxt_2 = \mmpkeExtL(Ctxt, n+1)$. Output $c =  (\id_s, \hgact, \encGroupSecret_1,\allowbreak  \encGroupSecret_2,\allowbreak  \variable{welcomeData})$.

\begin{figure}[ht]
  \begin{tikzpicture}[->,>=stealth',level/.style={sibling distance = 5cm/#1,
      level distance = 1.3cm},scale=0.6, transform shape,
    treenode/.style = {circle, draw=black, align=center, minimum size=1cm},
    ctnode/.style = {rectangle, draw=black, align=center, minimum size=1cm},
    packnode/.style = {rectangle, draw=black, align=center, minimum size=.75cm}]
    \tikzstyle{level 1}=[level distance=1cm]
    
    \node[packnode](pack){$id_R$};
    \node[packnode, right = 0cm of pack](pack2){act};
    \node[packnode, right = 0cm of pack2](pack3){$sig$};
    \node[packnode, right = 0cm of pack3](pack4){$ct_0$};
    \node[packnode, right = 0cm of pack4](pack5){$ct_1,\ldots, {\color{orange}ct_4}, \ldots$};
    \node[packnode, right = 0cm of pack5](pack6){${\color{blue}\mmpkepk_1,\mmpkepk_2,\mmpkepk_3},\mmpkepk_4, \ldots, \mmpkepk_{\text{root}}$};
    
    \node[below = 2.1cm of pack4, treenode](root){$\mmpkepk_4$}
    child[draw=blue]{
      node[treenode,draw=blue]{$\mmpkepk_3$}
      child[draw=black]{
        node[ctnode]{$ct_2$}
      }
      child[draw=blue]{
        node[treenode,draw=blue]{$\mmpkepk_{2}$}
        child{
          node[treenode,draw=blue]{$\mmpkepk_{1}$}
        }
        child[draw=black]{
          node[ctnode]{$ct_1$}
        }
      }
    }
    child[draw=orange]{
      node[treenode]{$\bot$}
      child[draw=black]{
        node[ctnode]{$ct_3$}
      }
      child{
        node[ctnode, draw=orange]{$ct_4$}
        child[draw=black]{
          node[treenode]{$\mmpkepk_{N-1}$}
        }
        child{
          node[treenode]{$\mmpkepk_{id_R}$}
        }
      }
    };
    \node[above right = .1cm of root](dots){\huge$\iddots$};
    \node[right = .6cm of dots]{\huge$\ddots$};
    \node[above right = 1cm of root, treenode]{$\mmpkepk_{\text{root}}$};
    \node at ($(root) + (0,-5)$)(arrow){\huge$\Downarrow$};
    \node[packnode, below = .2cm of arrow] {$id_R, act, sig, ct_0, \color{orange}{ct_4}, \color{blue}{\mmpkepk_1,\mmpkepk_2,\mmpkepk_3}$};
  \end{tikzpicture}  
  \caption{Server extraction algorithm. Lowest common ancestor (LCA) pof $id_R$ and $id_S$ is $\mmpkepk_4$, so all blue
    public keys are included in $id_R$'s packet. Since the sibling of $\mmpkepk_3$ is empty, there corresponding path
    secret of $\mmpkepk_4$ is encrypted to its resolution, resulting in the two ciphertext.
    % MM: Removed the part about c_0 since we din't use it in the main body
%    parts $ct_3$ and $ct_4$. $id_R$ can decrypt $ct_4$, since it lies on its path (orange) to the LCA . The signature and
%    header are included in every package, as well as the ciphertext-independent part ($ct_0$) of the \mmPKE encryption.
}
  \label{fig:extract-one}
\end{figure}

%%% Local Variables:
%%% mode: latex
%%% TeX-master: "main"
%%% End:


\paragraph{Comparison with techniques of \cite{hashimoto2021cmpke}.}
The work \cite{hashimoto2021cmpke} introduces a technique for efficient packet authentication which is quite similar to the technique used by \saik. In particular, their CGKA uses a committing mPKE, cmPKE. A cmPKE differs from mPKE in that encryption outputs a tag $T$ which is a cryptographic commitment to the plaintext and is delivered to each receiver. Since in \cite{hashimoto2021cmpke} every recipient of a commit gets the same message, authenticating $T$ is sufficient for CGKA authentication.
%
We highlight a couple of differences between that technique and ours:
First, it is not clear how to use cmPKE in a tree-based CGKA, where a commit executes multiple instances of \protCMPKE, and hence we end up with multiple tags $T$, each delivered to a different subset of the group.
%
Second, using the hash of the encrypted message as $T$ does not result in an IND-CCA secure \protCMPKE, since a hash allows to easily tell which of two messages is encrypted. Therefore, the construction of \cite{hashimoto2021cmpke} uses key-committing encryption to both hide and bind the message.

To summarize, \protCMPKE introduced by \cite{hashimoto2021cmpke} is very useful for the CGKA type they consider and may well find more use-cases beyond CGKA. On the other hand, \textsf{SAIK}’s solution fits all types o CGKA, does not require additional properties to prove CGKA security and is more direct. Albeit, it is very CGKA-specific.
%%% Local Variables:
%%% mode: latex
%%% TeX-master: "main"
%%% End:
