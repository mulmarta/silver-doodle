% !TEX root = main.tex
% !TeX spellcheck = en_US

\section{Security of \saik}\label{sec:saik-sec-int}
To define the security we prove for \saik we fix the two safety
predicates \KwConf{} and \KwAuth{} used by $\funcCGKA$. We next give the intuition; see \cref{fig:safe} in \cref{sec:bgm_prot_proof} for the pseudocode. We define two versions of the
predicates: a stronger and a weaker one. For better exposition, the
stronger version is not achieved by \saik as presented in this work. But at the cost of added complexity
\saik can easily be extended to achieve it, as
described \cref{sec:ext-sec-predicates}.

We begin with the simpler stronger version. First of all, both predicates give no
guarantees for epochs in detached trees until they are attached and so we
ignore them in this section. Then, the definition is built around the notion of
secrets which make up the protocol state. There are two types of secrets:
group secrets, stored in the state of all parties, and individual secrets,
stored in the states of some parties. Each corruption exposes a number of
secrets and each epoch change replaces a number of secrets by (possibly)
secure ones. The helper predicate $\safeGrpSecsSecure{}(E)$ decides
if the group secrets in $E$ are secure, i.e., not exposed, and the
predicate $\safeIndSecsSecure(E, \id)$ decides if $\id$'s individual
secrets in $E$ are secure.
%
Then $\KwConf{}(E)$ equals to
$\safeGrpSecsSecure{}(E)$, since the epoch key is itself a group
secret. Further, $\KwAuth{}(E, \id)$ is true if either
$\safeGrpSecsSecure{}(E)$ or
$\safeIndSecsSecure(E,\id)$ is true, because both group and $\id$'s secrets
are necessary to impersonate $\id$ in $E$.

It remains to determine when group and individual secrets are exposed. For
group secrets, $\safeGrpSecsSecure(E)$ is defined recursively. The
base case states that the group secrets in first epoch (when the group was
created) are secure if and only if no party is corrupted while in that epoch.
Intuitively, we assume the group was created by an honest party using good
randomness. Moreover, capturing perfect forward secrecy, corruptions in the
descendant epochs do not affect the confidentiality of earlier group secrets.

The induction step states that the group secrets in a non-root epoch $E$
are secure if no party is corrupted in $E$, the epoch is not created
by an injected packet from the adversary and either the group secrets in
$E$'s parent $E_p$ are secure or all individual secrets in
$E$ are secure. Intuitively, this formalizes the requirement that the
adversary can learn the group secrets in only three ways: A) by corrupting a
party currently in epoch $E$. B) by injecting the secrets (though
most injections are disallowed by the authenticity predicate). C) by
computing them the same way an honest \emph{receiver} transitioning to
$E$ would. The latter requires knowing the group secrets of
$E_p$ and the individual secrets of at least one receiver. Note that
the possible receivers are those parties that are group members in
$E$ and that are not $E$'s creator (who transitions on
sending). Note also that the fact that even knowing an epoch creator's individual
secrets in $E_p$ we can treat them as secure in $E$ which captures
so called \emph{post compromise security} (aka. \emph{healing} or
\emph{backwards security}). Indeed, in \saik, part of creating a new epoch
requires refreshing all ones individual secrets.

Finally, individual secrets of $\id$ in $E$ are exposed whenever
there is some other epoch $E'$ where $\id$'s secrets are the same as
in $E$ and where $\id$ was corrupted or its secrets were injected on
its behalf. The secrets of $\id$ are the same in two epochs if no epoch between them replaces the secrets, i.e., is created by $\id$, removes
it or adds it.

\paragraph{Weaker guarantees.}
In the weaker version of the security predicates, individual secrets of $\id$
in $E$ are not secure in an additional scenario, formalized by
\safeWeakAdd. In this scenario, an $\id_s$ first honestly adds $\id$ and the
adversary $\Adv$ injects a message adding $\id$ to some other epoch. Finally,
$\id$ joins $\Adv$'s epoch and is corrupted before sending any message. We explain why \saik is insecure in this case and how it can be modified to be secure in
\cref{sec:ext-sec-predicates}.

\paragraph{Security.} 
For the \mmPKE scheme we assume a security property called \mmowrcca, defined in \cref{sec:mmowrcca}. The notion is strictly weaker than \mmindcca; in \cref{sec:mmowrcca} we prove the implication.
%
Formally, the AKS is modeled as the functionality $\funcPKI$ defined in \cref{sec:pki}. \saik
works in the $\funcPKI$-hybrids model, i.e., $\funcPKI$ is available in the real world and emulated by the simulator in the ideal world.
The formal proof of \cref{thm:saik-security} can be found in \cref{sec:bgm_prot_proof}.

\newcommand{\ucideal}{\textnormal{\textsc{ideal}}}
\newcommand{\ucreal}{\textnormal{\textsc{real}}}
\begin{restatable}{theorem}{itkSec}\label{thm:saik-security}
	Let $\funcCGKA$ be the CGKA functionality with predicates \KwConf{} and \KwAuth{} defined in \cref{fig:safe}. Let $\saik$ be instantiated with an mmPKE \mmPKE, a signature scheme \sigscheme and \mac, and with the \hkdf functions modelled as a random oracle \hash.
	Let $\Adv$ be any environment. Denote the output of $\Adv$ from the real execution with \saik and the hybrid functionality $\funcPKI$ from \cref{fig:aks} as $\ucreal_{\saik, \funcPKI}(\Adv)$ and the output of $\Adv$ from the ideal execution with $\funcCGKA$ and a simulator $\ucsim$ as $\ucideal_{\funcCGKA, \ucsim}(\Adv)$.
	%
	There exists a simulator $\ucsim$ and adversaries $\Bdv[1]$ to $\Bdv[4]$ such that
	\begin{align*}
		\Pr[\ucideal&_{\funcCGKA, \ucsim}(\Adv) = 1] - \Pr\left[\ucreal_{\saik, \funcPKI}(\Adv) = 1\right] \leq\\
		&\textnormal{Adv}^{\gamefont{CR}}_{\hash}(\Bdv[1]) \\
	% confidentiality
	+\ & q_e^2(q_e+1) \log(q_n) \cdot\textnormal{Adv}^\mmowrcca_{\mmPKE,q_e\log(q_n),q_n}(\Bdv[2]) \\
	% authenticity asym
	+\ & 2q_e\cdot \textnormal{Adv}^{\ufcma}_\sigscheme(\Bdv[3]) \\
	+\ & q_e \cdot \textnormal{Adv}^{\ufcma}_\mac(\Bdv[4]) + 3q_hq_e^2(q_e+1)/2^\kappa,
\end{align*}
	where $q_e$, $q_n$ and $q_h$ denote bounds on the number of epochs, the group size and the number of $\Adv$'s queries to the random oracle modeling the $\hash$, respectively.
	%  In the reductions, the $\hkdf$ functions are modelled as a random oracle.
  \end{restatable}

%%% Local Variables:
%%% mode: latex
%%% TeX-master: "main"
%%% End:
