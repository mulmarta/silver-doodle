% !TEX root = main.tex
% !TeX spellcheck = en_US

\section{Details of \saik} \label{sec:saik-details}
\begin{table*}[!t]
  \begin{minipage}[t]{.48\textwidth}
  	\begin{tabularx}{\textwidth}{| l | X |}
      \hline
    	$\tree.\rt$ & The root.\\
%    	\hline
%    	$\tree.\nodes$ & The set of all nodes in $\tree$.\\
    	\hline
    	$v.\isroot$ & True iff $v = \tree.\rt$.\\
    	\hline
    	$v.\isleaf$ & True iff $v$ has no children.\\
    	\hline
    	$v.\parent$ & The parent node of $v$ (or $\bot$ if $v.\isroot$).\\
    	\hline
    	$v.\children$ & If $\neg v.\isleaf$: ordered list of $v$'s children.\\
    	\hline
    	$v.\nodeIndex$ & The node index of $v$.\\
      \hline
      $v.\depth$ & The length of the path from $v$ to $\tree.\rt$. \\
  		\hline
  		$v.\mmpkepk$ & An mmPKE encryption key.\\
  		\hline
  		$v.\mmpkesk$ & The corresponding decryption key.\\
  		\hline
      $v.\rsvk$ & If $v.\isleaf$: a signature verification key.\\
  		\hline
      $v.\rssk$ & If $v.\isleaf$: the corresponding signing key.\\
  		\hline
  		$v.\unmergedLeaves$ & The set of indices of the leaves below $v$ whose owner $\id$ does not know $v.\pkesk$.\\
  		\hline
  		$v.\id$ & If $v.\isleaf$: the $\id$ associated with that leaf.\\
  		\hline
  	\end{tabularx}

	\caption{Labels of a ratchet-tree $\tree$ and its nodes.}
	\label{tab:node_labels}
  \end{minipage}
  \hfill
  \begin{minipage}[t]{.48\textwidth}
  	\begin{tabularx}{\textwidth}{| l | X |}
  		\hline
  		$\itkSt.\groupId$ & The identifier of the group.\\
  		\hline
  		$\itkSt.\tree$ & The ratchet tree.\\
  		\hline
  		$\itkSt.\leaf$ & The party's leaf in $\tree$.\\
  		\hline
  		$\itkSt.\treeHash$ & A hash of the public part of $\tree$.\\
  		\hline
      $\itkSt.\lastAct$ & The last modification of the group state and the user who initiated it.\\
  		\hline
  		$\itkSt.\applicationSecret$ & The current epoch's CGKA key. Exposed to the application layer.\\
  		\hline
  		$\itkSt.\initSecret$ & The next epoch's init secret.\\
  		\hline
      $\itkSt.\membershipKey$ & The next epoch's membership secret for authenticating messages.\\
  		\hline
  		$\itkSt.\groupContext()$ & Returns $(\itkSt.\groupId, \itkSt.\treeHash, \itkSt.\lastAct)$.\\
  		\hline
        $\itkSt.\confTag$ & The confirmation tag, which is signed to ensure authenticity.\\
  		\hline
  	\end{tabularx}
    \caption{The protocol state of a party $\id$ and the helper method for computing the context.}
    \label{tab:prot-state}
  \end{minipage}
  \begin{minipage}{.48\textwidth}
    \begin{tabularx}{\textwidth}{| l | X |}
      \hline
      $\pathSecret$ & The path secrets $s_2,\ldots,s_n$ used to derive the keypairs in each node. Sent via the \mmPKE
                      encryption to keep tree invariant intact.\\
      \hline
      $\commitSecret$ & The path secret in the root node. Used as seed for the key schedule together with \initSecret
                        from the previous epoch\\
      \hline
      $\joinerSecret$ & Secret sent to new group members. Together
                        with the group context, enables computation of the $\epochSecret$.\\
      \hline
      $\epochSecret$ & Base secret used to derive all other secrets, i.e. \applicationSecret,
                       \membershipKey,\initSecret, \confTag \\
      \hline
    \end{tabularx}
    \caption{Intermediate values computed by the protocol that are not part of the state.}
    \label{tab:prot-intermediate-state}
  \end{minipage}
  
%  \end{table*}
%\begin{table*}[!t]
  \begin{minipage}{\textwidth}
  	\begin{tabularx}{.48\textwidth}{| l | X |}
  		\hline
  		$\tree.\clone()$ & Returns a copy of $\tree$.\\
  		\hline
  		$\tree.\public()$ & Returns a copy of $\tree$ with all labels $v.\rssk$ and $v.\pkesk$ set to $\bot$.\\
  		\hline
  		$\tree.\roster()$ & Returns $\id$'s of all parties in $\tree$.\\
  		\hline
  		$\tree.\leaves()$ & Returns the list of all leaves in the tree, sorted from left to right.\\
  		\hline
  		$\tree.\leafof(\id)$ & Returns the leaf $v$ with $v.\id = \id$.\\
  		\hline
  		$\tree.\getleaf()$ & Returns leftmost $v$ s.t. $\neg v.\inuse()$. If no such $v$ exists, adds a new leaf using $\addleaf(\tree)$ and returns it.\\
      \hline
      $\tree.\blankpath(v)$ & For all $u\in\tree.\directpath(v)$ calls $u.\blank()$. \\
      \hline
  		$\tree.\inSubtree(u,v)$ & Returns true if $u$ is in $v$'s subtree.\\
  		\hline
  		$v.\inuse()$ & Returns $\false$ iff all labels are $\bot$.\\
  		\hline
  		$v.\blank()$ & Sets all labels of $v$ to $\bot$.\\
  		\hline
  	\end{tabularx}
    \hfill
    \begin{tabularx}{.48\textwidth}{| l | X |}
  		\hline
  		$\tree.\lca(u, v)$ & Returns the lowest common ancestor of the two leafs.\\
  		\hline
  		$\tree.\directpath(v)$ & Returns the path from $v$'s parent to the root.\\
  		\hline
  		$\tree.\mergeleaves(v)$ & Sets $u.\unmergedLeaves \gets \emptyset$ for all $u \in
                                    \tree.\directpath(v)$\\
  		\hline
  		$\tree.\unmergeleaf(v)$ & Sets $u.\unmergedLeaves \setadd v$ for all $u$ returned by $\tree.\directpath(v)$\\
  		\hline
      $v.\resolution()$ &
      If $v.\inuse$, return $(v) \append v.\unmergedLeaves$. Else if $v.\isleaf$, return $()$. Else, return $v.\children[1].\resolution() $ $\append \dots \append v.\children[n].\resolution()$\\
%  		Return
%  		$\begin{cases}
%  			(v) \append v.\unmergedLeaves & \text{if } v.\inuse() \\
%        \term{concatChildResolution}(v) & \text{else if } \neg v.\isleaf \\
%  			\emptylist & \text{else},
%  		\end{cases}$\\
%      &where $\term{concatChildResolution}(v) = v.\children[1].\resolution() \append \dots \append v.\children[n].\resolution()$.\\
  		\hline
  		$v.\resolvent(u)$ & Returns the ancestor of $u$ in $v.\resolution() \setminus (v)$ (or $\bot$ if $u$ is not a descendant of $v$).\\
  		\hline
  	\end{tabularx}
	  \caption{Helper methods for a ratchet tree $\tree$ and its nodes.\vspace{-1em}}
	  \label{tab:node_helpers}
  \end{minipage}
\end{table*}

%%% Local Variables:
%%% mode: latex
%%% TeX-master: "main"
%%% End:

In this section we give the details of the \saik protocol. The pseudocode can be found in \cref{fig:prot1,fig:prot-helpers2}.

\subsection{Ratchet Trees}
% !TEX root = main.tex
% !TeX spellcheck = en_US
A ratchet tree is a left-balanced $q$-ary tree (a formal definition can be found in \cref{sec:addleafprf}). This generalizes \protITK' binary trees. Using $q\neq2$ can be beneficial in certain situations.
A ratchet tree, as well as its nodes, have a number of labels listed in \cref{tab:node_labels}.
We also define a number of helper methods in \cref{tab:node_helpers}.
%
\begin{table*}[!t]
  \begin{minipage}[t]{.48\textwidth}
  	\begin{tabularx}{\textwidth}{| l | X |}
      \hline
    	$\tree.\rt$ & The root.\\
%    	\hline
%    	$\tree.\nodes$ & The set of all nodes in $\tree$.\\
    	\hline
    	$v.\isroot$ & True iff $v = \tree.\rt$.\\
    	\hline
    	$v.\isleaf$ & True iff $v$ has no children.\\
    	\hline
    	$v.\parent$ & The parent node of $v$ (or $\bot$ if $v.\isroot$).\\
    	\hline
    	$v.\children$ & If $\neg v.\isleaf$: ordered list of $v$'s children.\\
    	\hline
    	$v.\nodeIndex$ & The node index of $v$.\\
      \hline
      $v.\depth$ & The length of the path from $v$ to $\tree.\rt$. \\
  		\hline
  		$v.\mmpkepk$ & An mmPKE encryption key.\\
  		\hline
  		$v.\mmpkesk$ & The corresponding decryption key.\\
  		\hline
      $v.\rsvk$ & If $v.\isleaf$: a signature verification key.\\
  		\hline
      $v.\rssk$ & If $v.\isleaf$: the corresponding signing key.\\
  		\hline
  		$v.\unmergedLeaves$ & The set of indices of the leaves below $v$ whose owner $\id$ does not know $v.\pkesk$.\\
  		\hline
  		$v.\id$ & If $v.\isleaf$: the $\id$ associated with that leaf.\\
  		\hline
  	\end{tabularx}

	\caption{Labels of a ratchet-tree $\tree$ and its nodes.}
	\label{tab:node_labels}
  \end{minipage}
  \hfill
  \begin{minipage}[t]{.48\textwidth}
  	\begin{tabularx}{\textwidth}{| l | X |}
  		\hline
  		$\itkSt.\groupId$ & The identifier of the group.\\
  		\hline
  		$\itkSt.\tree$ & The ratchet tree.\\
  		\hline
  		$\itkSt.\leaf$ & The party's leaf in $\tree$.\\
  		\hline
  		$\itkSt.\treeHash$ & A hash of the public part of $\tree$.\\
  		\hline
      $\itkSt.\lastAct$ & The last modification of the group state and the user who initiated it.\\
  		\hline
  		$\itkSt.\applicationSecret$ & The current epoch's CGKA key. Exposed to the application layer.\\
  		\hline
  		$\itkSt.\initSecret$ & The next epoch's init secret.\\
  		\hline
      $\itkSt.\membershipKey$ & The next epoch's membership secret for authenticating messages.\\
  		\hline
  		$\itkSt.\groupContext()$ & Returns $(\itkSt.\groupId, \itkSt.\treeHash, \itkSt.\lastAct)$.\\
  		\hline
        $\itkSt.\confTag$ & The confirmation tag, which is signed to ensure authenticity.\\
  		\hline
  	\end{tabularx}
    \caption{The protocol state of a party $\id$ and the helper method for computing the context.}
    \label{tab:prot-state}
  \end{minipage}
  \begin{minipage}{.48\textwidth}
    \begin{tabularx}{\textwidth}{| l | X |}
      \hline
      $\pathSecret$ & The path secrets $s_2,\ldots,s_n$ used to derive the keypairs in each node. Sent via the \mmPKE
                      encryption to keep tree invariant intact.\\
      \hline
      $\commitSecret$ & The path secret in the root node. Used as seed for the key schedule together with \initSecret
                        from the previous epoch\\
      \hline
      $\joinerSecret$ & Secret sent to new group members. Together
                        with the group context, enables computation of the $\epochSecret$.\\
      \hline
      $\epochSecret$ & Base secret used to derive all other secrets, i.e. \applicationSecret,
                       \membershipKey,\initSecret, \confTag \\
      \hline
    \end{tabularx}
    \caption{Intermediate values computed by the protocol that are not part of the state.}
    \label{tab:prot-intermediate-state}
  \end{minipage}
  
%  \end{table*}
%\begin{table*}[!t]
  \begin{minipage}{\textwidth}
  	\begin{tabularx}{.48\textwidth}{| l | X |}
  		\hline
  		$\tree.\clone()$ & Returns a copy of $\tree$.\\
  		\hline
  		$\tree.\public()$ & Returns a copy of $\tree$ with all labels $v.\rssk$ and $v.\pkesk$ set to $\bot$.\\
  		\hline
  		$\tree.\roster()$ & Returns $\id$'s of all parties in $\tree$.\\
  		\hline
  		$\tree.\leaves()$ & Returns the list of all leaves in the tree, sorted from left to right.\\
  		\hline
  		$\tree.\leafof(\id)$ & Returns the leaf $v$ with $v.\id = \id$.\\
  		\hline
  		$\tree.\getleaf()$ & Returns leftmost $v$ s.t. $\neg v.\inuse()$. If no such $v$ exists, adds a new leaf using $\addleaf(\tree)$ and returns it.\\
      \hline
      $\tree.\blankpath(v)$ & For all $u\in\tree.\directpath(v)$ calls $u.\blank()$. \\
      \hline
  		$\tree.\inSubtree(u,v)$ & Returns true if $u$ is in $v$'s subtree.\\
  		\hline
  		$v.\inuse()$ & Returns $\false$ iff all labels are $\bot$.\\
  		\hline
  		$v.\blank()$ & Sets all labels of $v$ to $\bot$.\\
  		\hline
  	\end{tabularx}
    \hfill
    \begin{tabularx}{.48\textwidth}{| l | X |}
  		\hline
  		$\tree.\lca(u, v)$ & Returns the lowest common ancestor of the two leafs.\\
  		\hline
  		$\tree.\directpath(v)$ & Returns the path from $v$'s parent to the root.\\
  		\hline
  		$\tree.\mergeleaves(v)$ & Sets $u.\unmergedLeaves \gets \emptyset$ for all $u \in
                                    \tree.\directpath(v)$\\
  		\hline
  		$\tree.\unmergeleaf(v)$ & Sets $u.\unmergedLeaves \setadd v$ for all $u$ returned by $\tree.\directpath(v)$\\
  		\hline
      $v.\resolution()$ &
      If $v.\inuse$, return $(v) \append v.\unmergedLeaves$. Else if $v.\isleaf$, return $()$. Else, return $v.\children[1].\resolution() $ $\append \dots \append v.\children[n].\resolution()$\\
%  		Return
%  		$\begin{cases}
%  			(v) \append v.\unmergedLeaves & \text{if } v.\inuse() \\
%        \term{concatChildResolution}(v) & \text{else if } \neg v.\isleaf \\
%  			\emptylist & \text{else},
%  		\end{cases}$\\
%      &where $\term{concatChildResolution}(v) = v.\children[1].\resolution() \append \dots \append v.\children[n].\resolution()$.\\
  		\hline
  		$v.\resolvent(u)$ & Returns the ancestor of $u$ in $v.\resolution() \setminus (v)$ (or $\bot$ if $u$ is not a descendant of $v$).\\
  		\hline
  	\end{tabularx}
	  \caption{Helper methods for a ratchet tree $\tree$ and its nodes.\vspace{-1em}}
	  \label{tab:node_helpers}
  \end{minipage}
\end{table*}

%%% Local Variables:
%%% mode: latex
%%% TeX-master: "main"
%%% End:


Importantly, the \emph{direct path} of a leaf $u$ consists of (the ordered list of) all nodes on the path from $u$ to the root, without $u$.
The \emph{resolution} of a node $v$ is the minimal set of descendant non-blank nodes that covers the whole sub-tree rooted at $v$. % i.e., such that for every descendant $u$ of $v$ there exists node $w$ in the resolution such that $w$ is non-blank and $w$ an ancestor of $u$.


%%% Local Variables:
%%% mode: latex
%%% TeX-master: "main"
%%% End:


\subsection{\saik State and Algorithms}
% !TEX root = main.tex
% !TeX spellcheck = en_US

The state of $\saik$ consists of a number of variables, outlined in \cref{tab:prot-state}. The table also includes short
descriptions of the roles of the secrets in the key schedule. The protocol will ensure that states of any two parties in
the same epoch differ at most in labels of nodes of $\itkSt.\tree$ that describe secret keys and the label
$\itkSt.\leaf$. This means that they agree on the secrets $\itkSt.\applicationSecret$ and $\itkSt.\initSecret$, as well
as on the public context, computed by the helper method $\groupContext()$ in \cref{tab:prot-state}.



%%% Local Variables:
%%% mode: latex
%%% TeX-master: "main"
%%% End:


%\subsection{\saik Algorithms}
\saik's algorithms are defined in \cref{fig:prot1,fig:prot-helpers2}. Apart from initialization, there are three main algorithms (the rest of the code are subroutines) exposed to a user (or a higher-level application). They are identified by keywords \keyword{Send}, \keyword{Receive} and \keyword{Key}, respectively. First, \keyword{Send} is used to create a new epoch. When the user inputs \keyword{Send} followed by the intended group modification (update, add or remove), the protocol applies the modification and returns a message, which the user can upload to the mailboxing service. Second, \keyword{Receive} is used to process messages downloaded from the service. Third, with \keyword{Key} user gets the current group key.

The formal syntax of saCGKA protocols is defined as part of our security definition in \cref{sec:model}. In particular, an saCGKA protocol must expose the same interface as the ideal CGKA functionality.

% !TEX root = main.tex
% !TeX spellcheck = en_US


\begin{figure*}[!p]%\vspace*{-2em}\hspace*{-1.5em}
%  \begin{minipage}{\linewidth+2em}
	\begin{anybox}{\sffamily\bfseries \saik : Algorithms}
			\begin{minipage}[t]{0.48\linewidth}
        {\bf Initialization}
        \begin{algorithmic}
          \If{$\id = \pgod$}
            \State $\itkSt \gets \method{new-state}()$
            \State $\itkSt.\groupId, \itkSt.\initSecret, \itkSt.\membershipKey, \itkSt.\applicationSecret \getsr \bits^\secparam$
            \State $\itkSt.\tree \gets \method{new-LBT}()$
            \State $\itkSt.\leaf \gets \itkSt.\tree.\leaves[0]$
            \State $(\itkSt.\leaf.\rsvk, \itkSt.\leaf.\rssk) \gets \sigkg()$
          \EndIf
        \end{algorithmic}

        \medskip
        {\bf Input $(\keyword{Send}, \hgact), \hgact \in \{\hglu , \hglr\md\id_t, \hgla\md\id_t\}$ from $\id$}
				\begin{algorithmic}
					\State $\KwReq\ \itkSt \neq \bot$
          \State \vspace*{-.5em}\Comment{In case of add, fetch $\id_t$'s keys from AKS (AKS runs $\genPKIkeys$).}
          \If{$\hgact = \hgla\md\id_t$}
            \State $(\mmpkepk_t, \rsvk_t, \mmpkepk_t') \gets \KwQuery\ (\keyword{GetPk},\id_t) \text{ to } \funcKB$
            \State $\hgact \gets \hgla\md\id_t\md(\mmpkepk_t, \rsvk_t,\mmpkepk_t')$
          \EndIf
          \State \smashedComment{Create the state and secrets for the new epoch.}
          \State \KwTry{} $(\itkSt', \pathSecrets, \joinerSecret)  \gets\provState(\hgact)$
          \State \Comment{Encrypt the path secrets using the new epoch's ratchet tree. For adds, also encrypt the joiner secret.}
          \If{$\hgact \in \{\hglu , \hglr\md\id_t\}$}
            \State $\pathSecCtxt \gets \encSecrets(\itkSt', \pathSecrets, \bot, \bot, \bot)$
          \ElsIf{$\hgact = \hgla\md\id_t\md(\mmpkepk_t, \rsvk_t,\mmpkepk_t')$}
            \State $\pathSecCtxt \gets \encSecrets(\itkSt', \pathSecrets, \id_t, \mmpkepk_t', \joinerSecret)$
          \EndIf
%          \State \smashedComment{Sign data under current epoch's secrets.}
%          \State $\updatedPks \gets ((\itkSt'.\leaf.\mmpkepk, \itkSt'.\leaf.\rsvk))$
%          \State $\updatedPks \gets \updatedPks \concat (v.\mmpkepk : v \in \itkSt'.\tree.\directpath(\itkSt'.\leaf))$
%          \State $(\vec{\variable{tbs}}, \rdclass) \gets \method{to-be-signed}(\pathSecCtxt, \updatedPks, \hgact)$
%          \State $\sig \gets \rssignL(\itkSt.\tree.\leafof(\id).\rssk, \itkSt.\membershipKey, \vec{\variable{tbs}}, \rdclass)$
%          \State $\itkSt \gets \itkSt'$
          \State $\ssk \gets \itkSt.\tree.\leafof(\id).\ssk$
          \State  $\sig \gets \sigsign(\ssk, \underline{\itkSt'.\confTag})$
          \If{$\hgact = \hglr\md\id_t$}
            \State \smashedComment{Authenticate removal message for $\id_t$}
            \State $\sig_t \gets \sigsign(\ssk, (\id, \hglr\md\id_t))$
            \State $\macsig_t \gets \mactag(\itkSt.\membershipKey, (\id, \hglr\md\id_t, \underline{\itkSt.\confTag}))$
            \State \Return $(\id, \hgact, \underline{\pathSecCtxt}, \updatedPks, \sig, \sig_t, \macsig_t)$
          \EndIf
          \State $\itkSt \gets \itkSt'$
          \If{$\hgact = \hgla\md\id_t\md(\mmpkepk_t, \rsvk_t,\mmpkepk_t')$}
            \State \smashedComment{Send additional data for $\id_t$.}
            \State $\variable{welcomeData} \gets (\itkSt.\groupId, \itkSt.\tree.\public(), \mmpkepk_t')$
            \State \Return $(\id, \hgact, \underline{\pathSecCtxt}, \updatedPks, \sig, \variable{welcomeData})$
          \EndIf
          \State \Return $(\id, \hgact, \underline{\pathSecCtxt}, \updatedPks, \sig)$
				\end{algorithmic}

      \end{minipage}
		\hfill
			\begin{minipage}[t]{0.5\linewidth}
        \bf Input {$\keyword{Key}$ from $\id$}
				\begin{algorithmic}
					\State $\KwReq\ \itkSt \neq \bot$
					\State $k \gets \itkSt.\applicationSecret$
					\State $\itkSt.\applicationSecret \gets \bot$
					\State \Return $k$
				\end{algorithmic}

        \medskip
        {\bf Input $(\keyword{Receive}, (\id_s, \literal{removed}, \sig_t, \macsig_t))$ from $\id$}

        \textnormal{\smashedComment{Receiver is removed.}}
        \begin{algorithmic}
          \State $\ersvk \gets \itkSt.\tree.\leafof(\id_s).\rsvk$
          \State \KwReq{} $\sigvrf(\ersvk, (\id_s, \hglr\md\id), \sig_t)$
          \State \KwReq{} $\macvrf(\itkSt.\membershipKey, (\id_s, \hglr\md\id, \underline{\itkSt.\confTag}), \macsig_t)$
%          \State $\vec{\variable{tbv}} \gets ((\id_s, \hglr\md\id, \itkSt.\groupId))$
%          \State \vspace*{-.5em}\Comment{\normalfont Check if removing allowed and compute the reduction pattern $\rd=(\ell,0,1)$.}
%          \State \KwTry{} $\itkSt' \gets \applyact(\itkSt.\clone(), \id_s, \hglr\md\id)$
%          \State $\ell \gets \getWeights(\itkSt', \id_s)$
%          \State $\ersvk \gets \itkSt.\tree.\leafof(\id_s).\rsvk$
%          \State \KwReq{} $\rsvrfyL(\rsvk, \itkSt.\membershipKey, \vec{\variable{tbv}}, (\ell,0,1), \redactedSig)$
          \State $\itkSt \gets \bot$
          \State \Return $(\id_s, \hglr\md\id)$
        \end{algorithmic}

        \medskip
        \textbf{Input $(\keyword{Receive}, (\id_s, \hgact, \underline{\pathSecCtxtInd}, \underline{\redactedUpPks}, \sig))$ from $\id$}

        \textnormal{\smashedComment{Receiver is a member.}}
        \begin{algorithmic}
          \State \KwTry{} $\itkSt' \gets \applyact(\itkSt.\clone(), \id_s, \hgact)$
%          \State \smashedComment{\normalfont Get the expected reduction pattern using the new state.}
%          \State \KwTry{} $\rd \gets \myReduction(\itkSt'.\tree, \id_s)$
%          \State $\vec{\variable{tbv}} \gets (\pathSecCtxtInd) \concat ((\id_s, \hgact, \itkSt.\groupId)) \concat \redactedUpPks$
%          \State $\ersvk \gets \itkSt.\tree.\leafof(\id_s).\rsvk$
%          \State \KwReq{} $\rsvrfyL(\rsvk, \itkSt.\membershipKey, \vec{\variable{tbv}}, \rd, \redactedSig)$
%          \State \smashedComment{\normalfont Transition to next epoch.}
          \State \KwTry{} $(\itkSt, \confTag) \gets \nextState(\itkSt', \underline{\pathSecCtxtInd}, \underline{\redactedUpPks}, \id_s, \hgact)$
          \State $\ersvk \gets \itkSt.\tree.\leafof(\id_s).\rsvk$
          \State \KwReq{} $\sigvrf(\ersvk, \underline{\confTag}, \sig)$
          \If{$\hgact=\hgla\md\id_t\md(\mmpkepk_t,\rsvk_t)$}
            \Return $(\id_s, \hgla\md\id_t)$
          \Else\
            \Return $(\id_s, \hgact)$
          \EndIf
        \end{algorithmic}

        \medskip
        \textbf{Input $(\keyword{Receive}, (\id_s, \hgact, \encGroupSecret_1, \encGroupSecret_2, \variable{welcomeData})))$ from $\id$}

        \textnormal{\smashedComment{Receiver joins.}}
        \begin{algorithmic}
          \State $\KwReq\ \itkSt = \bot$
          \State \KwParse{} $(\groupId, \tree, \mmpkepk') \gets \variable{welcomeData}$
          \State $\itkSt \gets \method{new-state}$
          \State $(\itkSt.\groupId, \itkSt.\tree, \itkSt.\lastAct) \gets (\groupId, \tree, (\id_s, \hgla\md\id))$
          \State $v \gets \itkSt.\tree.\leafof(\id)$
          \State \KwTry{} $(\mmpkesk, \rssk, \mmpkesk') \gets \KwQuery\ \keyword{GetSk}((v.\mmpkepk, v.\rsvk, \mmpkepk')) \textnormal{ to } \funcKB$
          \State $(v.\mmpkesk, v.\rssk) \gets (\mmpkesk, \rssk)$
          \State $\itkSt \gets \setTreeHash(\itkSt)$
          \State \KwTry{} $(\itkSt, \confTag) \gets \populateSecrets(\itkSt, \mmpkesk', \encGroupSecret_1, \encGroupSecret_2, \id_s)$
%          \State $\ersvk \gets \itkSt.\tree.\leafof(\id_s).\rsvk$
%          \State \KwReq{} $\sigvrf(\ersvk, \confTag)$
					\State \Return $(\itkSt.\tree.\roster(), \id_s)$
        \end{algorithmic}
      \end{minipage}
  \end{anybox}

  \medskip
  \begin{anybox}{\sffamily\bfseries \saik : Helpers for encryption and key generation for $\funcPKI$}
			\begin{minipage}[t]{.49\linewidth}
        {\bf {helper $\encSecrets(\itkSt', \pathSecrets, \id_t, \mmpkepk_t', \joinerSecret)$}}
				\begin{algorithmic}
          \State $L \gets \getPathSecsMap(\itkSt'.\tree, \id)$
          \State $\vec m, \vec \mmpkepk \gets ()$
          \For{$j=1$ \bf to $\len(L)$}
            \State $(i, v) \gets L[j]$
            \State $\vec m \listapp \pathSecrets[i]$
            \If{$\id_t\neq\bot \land v = \itkSt'.\tree.\leafof(\id_t)$}
              $\vec \mmpkepk \listapp \mmpkepk_t'$
            \Else\
              $\vec \mmpkepk \listapp \vec v.\mmpkepk$
            \EndIf
          \EndFor
          \If{$\id_t\neq\bot$}
            \State $\vec m \listapp \joinerSecret$
            \State $\vec\mmpkepk \listapp \mmpkepk_t'$
          \EndIf
          \State \Return $\underline{\mmpkeEncL}(\vec{\mmpkepk}, \vec m)$
        \end{algorithmic}
      \end{minipage}
		\hfill
			\begin{minipage}[t]{0.49\linewidth}
        {\bf {helper $\decSecrets(\itkSt', \id_s, \pathSecCtxtInd)$}}
				\begin{algorithmic}
          \State $v \gets \lca(\itkSt'.\tree.\leafof(\id_s), \itkSt'.\leaf).\resolvent(\itkSt'.\leaf)$
          \State \Return $\underline{\mmpkeDecL}(v.\mmpkesk, \pathSecCtxtInd)$
        \end{algorithmic}

        \medskip
        {\bf {helper $\genPKIkeys()$}}
				\begin{algorithmic}
          \State $(\mmpkepk, \mmpkesk) \gets \underline{\mmpkeKeyGenL}()$
          \State $(\rsvk, \rssk) \gets \sigkg()$
          \State $(\mmpkepk', \mmpkesk') \gets \underline{\mmpkeKeyGenL}()$
          \State \Return $((\mmpkepk, \rsvk, \mmpkepk'), (\mmpkesk, \rssk, \mmpkesk'))$
        \end{algorithmic}
      \end{minipage}
  \end{anybox}
	\vspace*{-0.7em}
	\caption{The algorithms of \saik.}
	\label{fig:prot1}
%  \end{minipage}
\end{figure*}


%\begin{figure}[!tpb]\vspace*{-5em}\hspace*{-1.5em}
%  \begin{minipage}{\linewidth+2em}
%	\begin{anybox}{\sffamily\bfseries \saik : Helpers for encryption and authentication}
%		\scalebox{0.7}{
%			\begin{minipage}[t]{.6\linewidth}
%        {\bf {helper $\encSecrets(\itkSt', \pathSecrets, \id_t, \mmpkepk_t', \joinerSecret)$}}
%				\begin{algorithmic}
%          \State $L \gets \getPathSecsMap(\itkSt'.\tree, \id)$
%          \State $\vec m, \vec \mmpkepk \gets ()$
%          \For{$j=1$ \bf to $\len(L)$}
%            \State $(i, v) \gets L[j]$
%            \State $\vec m \listapp \pathSecrets[i]$
%            \If{$\id_t\neq\bot \land v = \itkSt'.\tree.\leafof(\id_t)$}
%              $\vec \mmpkepk \listapp \mmpkepk_t'$
%            \Else\
%              $\vec \mmpkepk \listapp \vec v.\mmpkepk$
%            \EndIf
%          \EndFor
%          \If{$\id_t\neq\bot$}
%            \State $\vec m \listapp \joinerSecret$
%            \State $\vec\mmpkepk \listapp \mmpkepk_t'$
%          \EndIf
%          \State \Return $(\mmpkeEncL(\vec{\mmpkepk}, \vec m))$
%        \end{algorithmic}
%
%        \medskip
%        {\bf {helper $\decSecrets(\itkSt', \id_s, \pathSecCtxtInd)$}}
%				\begin{algorithmic}
%          \State $v \gets \lca(\itkSt'.\tree.\leafof(\id_s), \itkSt'.\leaf).\resolvent(\itkSt'.\leaf)$
%          \State \Return $\mmpkeDecL(v.\mmpkesk, \pathSecCtxtInd)$
%        \end{algorithmic}

%        \medskip
%        {\bf \mbox{helper $\method{to-be-signed}(\itkSt', \pathSecCtxt, \updatedPks, \hgact)$}}
% 				\begin{algorithmic}
%          \State $\vec w \gets \getWeights(\itkSt'.\tree, \id)$
%          \State $\vec{\variable{tbs}} \gets ()$
%          \For{$j=1$ \bf to $\abs{\vec w}$}
%            \State $\vec{\variable{tbs}} \listapp \mmpkeExtL(\pathSecCtxt,j)$
%          \EndFor
%          \State $\vec{\variable{tbs}} \listapp (\id, \hgact, \itkSt.\groupId)$
%          \State $\vec{\variable{tbs}} \listapp \updatedPks$
%          \State \Return $(\vec{\variable{tbs}}, \rdclassBGM_{\abs{\vec w},\vec w})$
%        \end{algorithmic}
%      \end{minipage}}
%		\hfill\scalebox{.7}{
%			\begin{minipage}[t]{0.75\linewidth}
%        {\bf \mbox{helper $\myReduction(\tree', \id_s)$}}
% 				\begin{algorithmic}
%          \State $L \gets \getPathSecsMap(\tree', \id_s)$
%          \For{$j=1$ \bf to $\len(L)$}
%            \State $(i, v) \gets L[j]$
%            \If{$\itkSt'.\tree.\inSubtree(\tree'.\leafof(\id), v)$}
%              \State \vspace*{-.5em}\Comment{We want the $j$-th ciphertext out of $\len(L)$ and the first $i+1$ items on the prefix list: the aux data, the leaf $\mmpkepk$ and $i-1$ $\mmpkepk$'s on $\id_s$'s direct path.}
%              \State \Return $(\len(L), j, i)$
%            \EndIf
%          \EndFor
%          \State \Return $\bot$
%        \end{algorithmic}

%        \medskip
%        {\bf {helper $\getPathSecsMap(\tree', \id_s)$}}
%				\begin{algorithmic}
%          \State\vspace*{-.7em}{\Comment{Returns a list of tuples $(i,v)$, denoting that when $\id_s$ commits in $\tree'$, the $i$-th path secret is encrypted under node $v$'s keys.}}
%          \State $L \gets ()$
%          \State $\fullpath \gets (\tree'.\leafof(\id_s)) \concat \tree'.\directpath(\tree'.\leafof(\id_s))$
%          \For{$i=1$ \bf to $\len(\fullpath)-1$}
%            \State $\vec v \gets \fullpath[i+1].\resolution() \setminus \fullpath[i].\resolution()$
%            \For{$j=1$ \bf to $\abs{\vec v}$}
%              \State $L \listapp (i, \vec v[j])$
%            \EndFor
%          \EndFor
%          \State \Return $L$
%        \end{algorithmic}

%        \medskip
%        {\bf {helper $\getWeights(\tree', \id_s)$}}
%				\begin{algorithmic}
%          \State\vspace*{-.7em}{\Comment{Returns a list of weights for $\rdclassBGM_{\ell,\vec w}$ when $\id_s$ commits in $\tree'$. Also allows to compute $\ell = \abs{\vec w}$.}}
%          \State $\vec w \gets ()$
%          \State $\fullpath \gets (\tree'.\leafof(\id_s)) \concat \tree'.\directpath(\tree'.\leafof(\id_s))$
%          \For{$i=2$ \bf to $\len(\fullpath)$}
%            \State $\vec v \gets \fullpath[i].\resolution() \setminus \fullpath[i-1].\resolution()$
%            \For{$j=1$ \bf to $\abs{\vec w}$}
%              \State \mbox{$\vec w \listapp  \big|\{u \in  \tree'.\leaves \mid u.\inuse() \land \tree'.\inSubtree(u,\vec v[j]) \}\big|$}
%            \EndFor
%          \EndFor
%          \State \Return $\vec w$
%        \end{algorithmic}
%      \end{minipage}}
%  \end{anybox}
%	\vspace*{-0.7em}
%	\caption{The algorithms of \saik.}
%	\label{fig:prot}
%  \end{minipage}
%\end{figure}


\begin{figure*}[!p]
  	\begin{anybox}{\sffamily\bfseries \saik : Creating epochs}
			\begin{minipage}[t]{.49\linewidth}
        {\bf {helper $\provState(\itkSt, \id, \hgact)$}}
				\begin{algorithmic}
          \State $\itkSt' \gets \itkSt.\clone()$
          \State \vspace*{-.7em}\Comment{Apply the action to the tree. Fails if the action is not allowed.}
          \State $\KwTry\ \itkSt' \gets \applyact(\itkSt', \id, \hgact)$
          \State \smashedComment{Re-key the direct path.}
          \State $\directpath \gets \itkSt'.\tree.\directpath(\itkSt'.\leaf)$
          \State $\pathSecrets[\wc] \gets \bot$
          \State $\pathSecrets[1] \getsr \{0,1\}^\secparam$
					\For{$i=1$ \textbf{to} $\len(\directpath)-1$}
            \State $v \gets \directpath[i]$
					  \State $r \gets \hkdfexp(\pathSecrets[i], \literal{node})$
            \State $(v.\mmpkepk, v.\mmpkesk) \gets \mmpkeKeyGenL(r)$
            \State $\pathSecrets[i+1] \gets \hkdfexp(\pathSecret[i], \literal{path})$
          \EndFor
          \State $\itkSt'.\tree.\mergeleaves(\itkSt'.\leaf)$
          \State \smashedComment{Re-key the leaf.}
          \State $(\itkSt'.\leaf.\mmpkepk, \itkSt'.\leaf.\mmpkesk) \gets \mmpkeKeyGenL()$
%          \State $(\itkSt'.\leaf.\rsvk, \itkSt'.\leaf.\rssk) \gets \rskeygenL()$
          \State $(\itkSt'.\leaf.\rsvk, \itkSt'.\leaf.\rssk) \gets \sigkg()$
          \State \vspace*{-.5em}\Comment{Set all context variables and then derive epoch secrets.}
          \State $\itkSt'.\lastAct \gets (\id, \hgact)$
          \State $\itkSt' \gets \setTreeHash(\itkSt')$
          \State $\commitSecret \gets \pathSecrets[\len(\pathSecrets)]$
          \State $(\itkSt', \joinerSecret) \gets \deriveKeys(\itkSt', \commitSecret)$
          \State \Return $(\itkSt', \pathSecrets, \joinerSecret)$
        \end{algorithmic}

        \medskip
        {\bf {helper $\applyact(\itkSt', \id_s, \hgact)$}}
				\begin{algorithmic}
          \State \KwReq{} $\id_s \in \itkSt'.\tree.\roster()$
          \If{$\hgact=\hglr\md\id_t$}
            \State \KwReq{} $\id_s \neq \id_t \land \id_t \in \itkSt'.\tree.\roster()$
            \State $\itkSt'.\tree.\blankpath(\itkSt'.\tree.\leafof(\id_t))$
            \State $\itkSt'.\tree.\leafof(\id_t).\blank()$
          \ElsIf{$\hgact=\hgla\md\id_t\md(\mmpkepk_t, \rsvk_t)$}
            \State \KwReq{} $\id_t \notin \itkSt'.\tree.\roster()$
            \State $v \gets \itkSt'.\tree.\getleaf()$
            \State $(v.\id, v.\mmpkepk, v.\rsvk) \gets (\id_t, \mmpkepk_t, \rsvk_t)$
            \State $\itkSt.\tree.\unmergeleaf(v)$
          \EndIf
        \end{algorithmic}

      \end{minipage}
		\hfill
			\begin{minipage}[t]{0.49\linewidth}
        {\bf {helper $\nextState(\itkSt', \pathSecCtxtInd, \redactedUpPks, \id_s, \hgact)$}}
				\begin{algorithmic}
          \State \smashedComment{Set keys on the re-keyed path.}
          \State $v_s \gets \itkSt'.\tree.\leafof(\id_s)$
          \State $\directpath \gets \itkSt'.\tree.\directpath(v_s)$
          \State $(v_s.\mmpkepk, v_s.\rsvk) \gets \redactedUpPks[1]$
          \State $i \gets 1$
          \While{$\directpath[i] \notin \{\itkSt'.\tree.\lca(\itkSt'.\leaf, v_s), \itkSt'.\tree.\rt\}$}
            \State \smashedComment{If message contains too few ek's, reject it.}
            \State \KwReq{} $i+1\leq\len(\redactedUpPks)$
            \State $\directpath[i].\mmpkepk \gets \redactedUpPks[i+1]$
            \State $i\inc$
          \EndWhile
          \State \smashedComment{Decrypt the path secret using the updated tree.}
          \State \KwTry{} $\pathSecret \gets \decSecrets(\itkSt', \id_s, \pathSecCtxtInd)$
          \While{$i < \len(\directpath)$}
            \State $v \gets \directpath[i]$
					  \State $r \gets \hkdfexp(\pathSecrets[i], \literal{node})$
            \State $(v.\pkpk, v.\pksk) \gets \mmpkeKeyGenL(r)$
            \State $\pathSecret \gets \hkdfexp(\pathSecret, \literal{path})$
            \State $i\inc$
          \EndWhile
          \State $\commitSecret \gets \pathSecret$
          \State $\itkSt'.\tree.\mergeleaves(v_s)$
          \State \smashedComment{Set all context variables and then derive epoch secrets.}
          \State $\itkSt'.\lastAct \gets (\id_s, \hgact)$
          \State $\itkSt' \gets \setTreeHash(\itkSt')$
          \State $(\itkSt', \joinerSecret)  \gets \deriveKeys(\itkSt', \commitSecret)$
          \State \Return $\itkSt'$
        \end{algorithmic}

        \medskip
        {\bf {helper $\populateSecrets(\itkSt', \pksk', \encGroupSecret_1, \encGroupSecret_2, \id_s)$}}
				\begin{algorithmic}
					\State \KwTry{} $\pathSecret \gets \underline{\mmpkeDecL}(\pksk, \encGroupSecret_1)$
          \State \KwTry{} $\joinerSecret \gets \underline{\mmpkeDecL}(\pksk, \encGroupSecret_2)$
					\State $v \gets \itkSt'.\tree.\lca(\itkSt'.\leaf, \itkSt'.\tree.\leafof(\id_s))$
					\While{$v \neq \itkSt'.\tree.\rt$}
					  \State $r \gets \hkdfexp(\pathSecret, \literal{node})$
					  \State $(\mmpkepk, v.\mmpkesk) \gets \underline{\mmpkeKeyGenL}(r)$
					  \State $\KwReq\ v.\mmpkepk = \mmpkepk$
					  \State $\pathSecret \gets \hkdfexp(\pathSecret, \literal{path})$
					  \State $v \gets v.\parent$
					\EndWhile
          \State $\itkSt' \gets \deriveEpochKeys(\itkSt', \joinerSecret)$
          \State \Return $\itkSt'$
        \end{algorithmic}
      \end{minipage}
  \end{anybox}
%\caption{}\label{fig:prot-helpers1}
%\end{figure}
%\begin{figure}[tbp]
	\begin{tcbraster}[raster columns=2, raster equal height]
		\begin{anybox}{\sffamily\bfseries \saik : Key schedule}
				\begin{minipage}[t]{\linewidth}
					{\bf {helper $\deriveKeys(\itkSt, \itkSt', \commitSecret)$}}
					\begin{algorithmic}
						\State $\joinerSecret \gets \hkdfext(\itkSt.\initSecret, \commitSecret)$
						\State $\itkSt' \gets \deriveEpochKeys(\itkSt', \joinerSecret)$
						\State \Return $\itkSt', \joinerSecret$
					\end{algorithmic}

					\medskip
					{\bf {helper $\deriveEpochKeys(\itkSt', \joinerSecret)$}}
					\begin{algorithmic}
            \State $\epochSecret \gets \hkdfext(\joinerSecret, \itkSt'.\groupContext())$

						\State $\itkSt'.\applicationSecret \gets \hkdfexp(\epochSecret, \literal{app})$
						\State $\itkSt'.\membershipKey \gets \hkdfexp(\epochSecret, \literal{membership})$
						\State $\itkSt'.\initSecret \gets \hkdfexp(\epochSecret, \literal{init})$
            \State $\itkSt'.\confTag \gets \hkdfexp(\epochSecret, \literal{confirmation})$
						\State \Return $\itkSt'$
					\end{algorithmic}
				\end{minipage}
		\end{anybox}
		%
		\begin{anybox}{\sffamily\bfseries \saik : Tree hash}
				\begin{minipage}[t]{\linewidth}
					{\bf {helper $\setTreeHash(\itkSt')$}}
        \begin{algorithmic}
					\State $\itkSt'.\treeHash \gets \computeTreeHash(\itkSt'.\tree.\rt)$
					\State \Return $\itkSt'$
				\end{algorithmic}

        \medskip
        {\bf {helper $\computeTreeHash(v)$}}
				\begin{algorithmic}
					\If{$v.\isleaf$}
					  \State \Return $\hash(v.\nodeIndex, v.\mmpkepk, v.\ersvk)$
					\Else
            \State $\ell \gets \len(v.\children)$
            \For{$i\in[\ell]$}
  					  $h_i \gets \computeTreeHash(v.\children[i])$
            \EndFor
            \State $h \gets (h_1, \dots, h_\ell)$
            \State \Return $\hash(v.\nodeIndex, v.\mmpkepk, v.\unmergedLeaves, h)$
					\EndIf
				\end{algorithmic}
				\end{minipage}
		\end{anybox}
	\end{tcbraster}

	\caption{Additional helper methods for \saik.}
	\label{fig:prot-helpers2}
%\end{minipage}
\end{figure*}

%%% Local Variables:
%%% mode: latex
%%% TeX-master: "main"
%%% End:


\newcommand{\extract}{\method{extract}}
\newcommand{\getExtractionIndices}{\method{getExtractionIndices}}
\subsection{Extraction Procedure for the Server}
Finally, we describe a procedure $\extract(C, \id) \to c$ used by the mailboxing service to take an uploaded message $C$
and compute the message $c$ delivered to user $\id$. Formally, this procedure is not part of our syntax or security
definitions, since for simplicity our model does not consider correctness (see \cref{sec:simplifications}) and an
untrusted service can anyway deliver arbitrary messages. It is formally defined in \cref{fig:saik-extraction-alg}.

\begin{figure}[ht]

  \begin{anybox}{\sffamily\bfseries \saik : Extraction}
    \begin{minipage}[t]{\linewidth}
      {\bf {helper $\extract(\id, \hgact, \underline{\pathSecCtxt}, \updatedPks, \sig, \id_s)$}}
      \begin{algorithmic}
        \State $i,j \gets \getExtractionIndices(\itkSt, \id)$
        \State $\pathSecCtxtInd \gets \mmpkeExt(C, i)$
        \State $\redactedUpPks \gets \updatedPks[1:j]$
        \State \Return $\id, \hgact, \pathSecCtxtInd, \redactedUpPks, \sig$
      \end{algorithmic}

      \medskip
      {\bf {helper $\getExtractionIndices(\itkSt, \id, \id_s)$}}
      \begin{algorithmic}
        \State $v_{\lca} = \itkSt.\tree.\lca(\id,\id_s)$
        \State $v_s = \itkSt.\tree.\leafof{\id_s}$
        \State $\directpath \gets \itkSt.\tree.\directpath(v_s)$
        \State \smashedComment{Compute number of public keys ``under'' lca.}
        \State $j \gets \itkSt.\tree.\leafof(\id_s).\depth - v.\depth$
        \State \smashedComment{Count number of encryptions before $\id$'s encryption.}
        \State $k = 0$
        \For{$2\leq l \leq j$}
        \State $k += \len(\itkSt.\tree.\resolution(\directpath[l].\children\setminus \directpath[l-1]))$
        \EndFor
        \State $S = \itkSt.\tree.\resolution(v_{\lca}.\children\setminus \directpath[j-1])$
        \State $i = 1$
        \While{$S[i] \cap \itkSt.\tree.\directpath(v_R) = \emptyset$}
        \State $i\inc$
        \EndWhile
        \State $i \gets i + k$
        \State \Return $i,j$
      \end{algorithmic}
    \end{minipage}
  \end{anybox}
  \caption{Helper functions for extraction.}
  \label{fig:saik-extraction-alg}
\end{figure}
%%% Local Variables:
%%% mode: latex
%%% TeX-master: "main"
%%% End:


Recall that $C$ contains the executed group operation $\hgact$ and the sender $\id_s$, a multi-recipient ciphertext $Ctxt$ and a vector of updated public keys $\updatedPks$. Roughly, $\extract$ only needs to compute $\id$'s individual mmPKE ciphertext $\mmpkeExtL(Ctxt, i)$ and the prefix of the first $j$ elements of $\updatedPks$. This requires that it knows the indices $i$ and $j$ for $\id$.
%
We notice that they can be easily computed using the public part of the ratchet tree, $\hgact$ and $\id_s$. Therefore, the indices can be obtained in two ways. First, the service can send $\hgact$ and $\id_s$ to $\id$, who replies with $i$ and $j$.  This requires interaction, but both $\id$ and the service are online at the time. Second, the service can store the current ratchet trees and compute $i$ and $j$ itself. The disadvantage of this is that it requires keeping a large state --- in case members are out of sync (e.g. a user is 10 epochs behind), the service needs to store one tree for each epoch which has an active member in it.
%
Once $i$ and $j$ are known, $\extract$ works as follows.

If $\hgact = \hglu$, set  $ctxt = \mmpkeExtL(Ctxt, i)$ and $\redactedUpPks = (\updatedPks[1], \dots, \updatedPks[j])$. Output $c =  (\id_s, \hgact, \pathSecCtxtInd,\allowbreak  \redactedUpPks,\allowbreak  \sig)$, where $\sig$ is a field of $C$. If $\hgact = \hglr\md\id$, then output $c =  (\id_s, \literal{removed}, \sig_t, \macsig_t)$ where $ \sig_t$ and $\macsig_t$ are taken from $C$. Finally, if $\hgact = \hgla\md\id$, $C$ contains $\variable{welcomeData}$, which in turn contains a ratchet tree. Based on this, compute $\id$'s index $i$ in $Ctxt$,  the number $n$ of recipients of $Ctxt$, and $ctxt_1 = $ $\mmpkeExtL(Ctxt, i)$ and $ctxt_2 = \mmpkeExtL(Ctxt, n+1)$. Output $c =  (\id_s, \hgact, \encGroupSecret_1,\allowbreak  \encGroupSecret_2,\allowbreak  \variable{welcomeData})$.

\subsection{Propose-Commit Syntax}
As discussed in \cref{sec:simplifications}, in order to tame the complexity of (sa)CGKA, we use a simplified syntax instead of the more general (and more efficient) propose-commit syntax. In this section we explain in detail how to transform \saik to the propose-commit syntax.

In the propose-commit syntax, an (sa)CGKA protocol takes the following inputs from a party $\id$:
\begin{description}
	\item[Add proposal:] $\id$ proposes to add $\id_t$. The protocol outputs a proposal packet $p$.
	\item[Remove proposal:] $\id$ proposes to remove $\id_t$. The protocol outputs a proposal packet $p$.
	\item[Update proposal:] $\id$ proposes to update their key material. The protocol outputs a proposal packet $p$.
	\item[Commit:] $\id$ inputs a list of proposal packets $(p_1, \dots, p_n)$ (after receiving them from other parties). The protocol outputs a commit packet $c$.
	\item[Process:] $\id$ inputs a commit packet $c$ and a list $(p_1, \dots, p_n)$ of proposals it commits (after receiving all these packets from other parties). The protocol outputs the semantics of applied group operations.
	\item[Key:] $\id$ fetches the current group key.
\end{description}

Observe that if an application always commits a single proposal immediately after creating it, the propose-commit syntax collapses to our (sa)CGKA syntax.

\paragraph{Proposals in \saik.}
\saik deals with proposals in the same way as \protITK. First, it computes the proposal content $\hgact$ which identifies the proposed modification:
\begin{description}
	\item[Add proposal:] To add $\id_t$, query $(\keyword{GetPk},\id_t) \text{ to } \funcKB$, receive $(\mmpkepk_t, $ $\rsvk_t, \mmpkepk_t')$ and set $\hgact = \literal{add}\md\id_t\md(\mmpkepk_t, \rsvk_t, \mmpkepk_t')$.
	\item[Remove proposal:] To remove $\id_t$, set $\hgact = \literal{rem}\md\id_t$.
	\item[Update proposal:] Sample new key pairs  $(\rsvk, \rssk) \gets \sigkg()$ and $(\mmpkepk, \mmpkesk) \gets \mmpkeKeyGenL()$ and store $\mmpkesk$ and $\rssk$ for later. Set $\hgact = \literal{upd}\md(\mmpkepk, \rsvk)$.
\end{description}
The proposal packet is $\hgact$ signed with the sender's current key $\gamma.\leaf(\id).\rssk$ and MACed with the current $\membershipKey$.

\paragraph{Commit in \saik.}
A commit in \saik is almost identical to its \keyword{Send} input. It proceeds in two steps.
\begin{enumerate}
	\item {\it Applying proposed group modifications to the ratchet tree:} Currently, \saik applies only a single modification in the $\createEpoch$ helper. The propose-commit \saik extends this step and applies all proposed actions one by one. This is done by calling the helper $\applyact(\itkSt', \id_s, \hgact)$ for each $\hgact$ (after verifying the signature and the MAC). Further, we extend the $\applyact$ helper to deal with update proposals -- such actions simply replace proposer's leaf public keys with the ones in $\hgact$. In addition, if the update proposal is applied by its sender, they replace their leaf's secret keys by $\mmpkesk$ and $\rssk$ stored when generating the update.
	\item {\it Rekeying the sender's path:} The remaining part of \keyword{Send} remains mostly unchanged. The only difference is related to the possibility of there being many add proposals:
	\begin{itemize}
		\item In line 9 of \keyword{Send} (excluding comments), the $\joinerSecret$ is currently encrypted to one new member. In the propose-commit \saik, it is instead encrypted to $N$ new members, who are the last $N$ recipients of the mmPKE ciphertext.
		\item In line 18 of \keyword{Send}, \saik currently computes the welcome data needed by new members. In the propose-commit \saik, this includes the public keys $(\mmpkepk'_1, \dots,$ $\mmpkepk'_N)$, of all $N$ new members, instead of only one. 
	\end{itemize} 
\end{enumerate}

\paragraph{Processing a commit in \saik.}
The receive procedure of \saik is modified analogous to its send procedure. In case the receiver is a member and is not removed, it first applies all proposed actions and then processes the committer's path the same way as the current \saik.

Finally, to join, a new member finds their public key in the list $(\mmpkepk'_1, \dots,\mmpkepk'_N)$ contained in the welcome data. Let $i$ denote the index of that key. They decrypt the $\joinerSecret$ as the $(N-i)$-th to last recipient of the mmPKE ciphertext. Then they proceed as in the current \saik.

%%% Local Variables:
%%% mode: latex
%%% TeX-master: "main"
%%% End:
