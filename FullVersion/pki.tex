\section{The Authenticated Key Service Functionality (AKS)}\label{sec:pki}

\begin{figure}[!tbp]
	\vspace*{-1.5em}\begin{systembox}{$\funcPKI$}
		\begin{flushleft}
			\mbox{Parameter: key-package generation algorithm $\genPKIkeys$.}

			\hrulefill
			\vspace*{-0.5em}
		\end{flushleft}
			\begin{minipage}[t]{0.49\linewidth}
				{\bf Initialization}
				\begin{algorithmic}
					\State $\mathsf{SK}[\cdot, \cdot] \gets \bot$
				\end{algorithmic}


        {\bf Input $\keyword{GetSK}(PK)$ from $\id$}
				\begin{algorithmic}
          \State $SK \gets \mathsf{SK}[\id,PK]$
          \State $\mathsf{SK}[\id,PK] \gets \bot$
          \State \Return $SK$
				\end{algorithmic}
			\end{minipage}
			\hfill
			\begin{minipage}[t]{0.49\linewidth}
				{\bf Input $(\keyword{GetPK},\id')$ from $\id$}
				\begin{algorithmic}
          \State $(PK, SK) \gets \genPKIkeys()$
					\State $\mathsf{SK}[\id',PK] \gets SK$
					\State Send $(\id', PK)$ to adv.
					\State \Return $PK$
				\end{algorithmic}
			\end{minipage}
	\end{systembox}

	\caption{The Authenticated Key service Functionality.\vspace{0.5cm}}
	\label{fig:aks}
\end{figure}
The AKS is modeled as the functionality $\funcPKI$ in \cref{fig:aks}. Formally, \saik
  works in the $\funcPKI$-hybrids model, i.e., $\funcPKI$ is available in the real world and emulated by the simulator in the ideal world.

$\funcPKI$ works as follows. When a party $\id$ wants to fetch a key package of another party $\id'$, $\funcPKI$ generates a new key package for $\id'$ using \saik's key-package generation algorithm (formally, the algorithm is a parameter of $\funcPKI$). It sends (the public part of) the package to $\id$ and to the adversary. (Note that since $\funcPKI$ exists in the real world, the adversary should be thought of as the UC environment.) The secrets for the key package can be fetched by $\id_t$ later, when it decides to join the group. Once fetched, secrets are deleted, which means that $\funcPKI$ cannot be used as secure storage.