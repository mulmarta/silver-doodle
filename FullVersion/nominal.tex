% !TEX root = main.tex
% !TeX spellcheck = en_US
\section{Nominal Groups}\label{app:nominal_groups}
We recall the definition and parameters for nominal groups from~\cite{EC:ABHKLR21_2}.
\begin{table*}[!tb]
	\center
	\begin{tabulary}{.5\textwidth}{cccccc}
		Name & \textbf{P-256  } & \textbf{P-384  } & \textbf{P-512  } & \textbf{\textsc{Curve}25519  } & \textbf{\textsc{Curve}448} \\
		\hline
		Security Level & 128 & 192 & 256 & 128 & 224 \\
		$P_\ngroup$ & $2^{-255}$& $2^{-383}$& $2^{-520}$& $2^{-250}$& $2^{-444}$\\
		$\Delta_\ngroup$ & 0 & 0 & 0 & $2^{-125}$ & $2^{-220}$ \\
		Size in bits & $256$ & $384$ & $512$ & $256$ & $512$ \\
		\hline
	\end{tabulary}
	\medskip
	\caption{Statistical parameters of NIST curves and nominal group curves.}
	\label{tab:nom_groups_params}
\end{table*}

\begin{definition}[Nominal Group]
  A nominal group $\ngroup = (\nG, g, p, $ $\nGE, \nGU, \nGexp)$ consists of a finite set of elements $\nG$, a base element $g
  \in \nG$, a prime $p$, a finite set of ``good'' exponents $\nGE \subset \Z$, a set of exponents $\nGU\subset
  \Z\setminus p\Z$ and an efficiently computable exponentiation function $\nGexp: \nG \times \Z \rightarrow \nG$. We
  write $X^y$ as shorthand for $\nGexp(X,y)$ and call elements of $\nG$ ``group elements''.
  $\ngroup$ has to fulfil the following properties:
  \begin{enumerate}
  \item $\nG$ is efficiently recognizable.
  \item $\left(X^y\right)^z = X^{yz}$ for all $X \in \nG, y,z\in\Z$
  \item the function $\phi$ defined by $\phi(x) = g^x$ is a bijection from $\nGU$ to $\{g^x | x\in[p-1]\}$.
  \end{enumerate}
  A nominal group $\ngroup$ is called \emph{rerandomisable}, when additionally
  \begin{enumerate}[resume]
  \item $g^{x+py} = g^x$ for all $x,y\in\Z$
  \item for all $y\in\nGU$, the function $\phi_y$ defined by $\phi_y(x) = g^{xy}$ is a bijection from $\nGU$ to $\{g^x |
    x \in [p-1]\}$.
  \end{enumerate}

  Property 3 (and 5) ensure that discrete logarithms are unique in $\ngroup$ in $\nGU$. 
  %
  Additionally, we define the two statistical parameters
  \[
    \Delta_{\ngroup} := \Delta[G_H, G_U],
  \]
  with $G_H$ is the uniform distribution over \nGE and $G_U$ is the uniform distribution over \nGU and
  \[
    P_\ngroup = \max_{Y \in \nG} \underset{x\getsr \nGE}{\text{Pr}}[Y = g^x].
  \]
\end{definition}

Any cyclic group, such as NIST curves, can be seen as a rerandomizable nominal group with the special properties that $\Delta_\ngroup =
0$ and $P_\ngroup = p-1$. Other popular examples of rerandomizable nominal groups are $\textsc{Curve}25519$ and
$\textsc{Curve}448$. \cref{tab:nom_groups_params} lists the parameters for these nominal groups.

For a more detailed explanation of these values, see \cite{EC:ABHKLR21_2}. Nominal groups and prime-order
groups behave indistinguishably except when group elements are sampled with exponents outside of $\nGE$ or a collision
occurs which wouldn't have been a collision in a prime-order group. Since these two events are statistical in nature and
occur with low probability, this only adds a negligible additive security loss compared to \cref{thm:mmpke_security}.

The \sdh assumption is almost identical over nominal groups except for the choice of exponents.
\begin{definition}[Double-Sided Strong Diffie-Hellman Assumption]
  Let $\ngroup = (\nG, g, p, \nGE, \nGU, \nGexp)$ be a nominal group. We define the advantage of an
  algorithm $\Adv$ in solving the \emph{Double-Sided Strong Diffie-Hellman problem(\sdh)} with respect to $\ngroup$ as
  \[
    \adv{\sdh}{\ngroup}(\Adv) =
    \left[
      Z = g^{xy}
      \middle\vert
      \begin{array}{c}
        x, y \getsr \nGU^2\\
        Z \getsr \Adv^{\oracle{O}}(\ngroup,p,g,g^x,g^y),
      \end{array}
    \right]
  \]
with $\oracle{O} = \{\oracle{O}_x(\cdot,\cdot),\oracle{O}_y(\cdot,\cdot)\}$, where $\oracle{O}_x,\oracle{O}_y$ are oracles which on input $U,V$ output $1$, iff $U^x = V$ or $U^y = V$ respectively.
The probability is taken over the random coins of the group generator, the choice of $x$ and $y$ and the adversaries
random coins.
\end{definition}

\begin{remark}
  Since $x,y$ are sampled from \nGU, the second property of nominal groups guarantees that the oracles $\oracle{O}_x$
  and $\oracle{O}_y$ are well-defined.
\end{remark}

\begin{theorem}
  Let $\ngroup = (\nG, g, p, \nGE, \nGU, \nGexp)$ be a nominal group.
  If the \sdh assumption holds relative to  $\ngroup$ and if \dem is an \indrcca secure DEM, then \mmPKE from
  \cref{fig:mmpke_constr} is $\mmindrcca$ secure with adaptive corruptions in the random oracle model and leakage
  function $\leak$ revealing the length of each plaintext. Specifically, there are adversaries $\Bdv[1],\Bdv[2]$ against
  \sdh and \indrcca of \dem respectively, s.t. for all adversaries \Adv against the \mmindrcca
  \begin{multline*}
    \adv{\mmindrcca}{\mmPKE}({\Adv}) \leq \\2e^2q_C\cdot n\cdot(\adv{\sdh}{\ngroup}({\Bdv[1]}) +
                                       \frac{q_{D_1}}{p} + \frac{q_{H}}{p}) \\ + n \cdot \adv{\indrcca}{\dem}({\Bdv[2]}) 
    \\+ 2(n+1)^2\cdot\Delta_{\ngroup}, + \mathcal{O}(q_D, q_H)\cdot P_{\ngroup},
  \end{multline*}
  where the runtime of $\Bdv[1]$ and $\Bdv[2]$ is roughly the same as \Adv and $q_{D_1},q_H$ and $q_C$ denote the number of queries to the decryption oracle $D$ in phase 1, the random oracle
  $H$ and the corruption oracle $\term{Cor}$ respectively.
\end{theorem}

\begin{proof}
  Mainly, the proof of \cref{thm:mmpke_security} still applies. That is because all operations performed by the
  adversary are well-defined over nominal groups. The main difference occurs when rerandomising the keys in each 
  hybrid. Here, not every exponent yields a valid group element, i.e. a valid key. Formally, we would add an additional
  hybrid for each chosen key, sampling its exponent from $\nGU$ instead of $\nGE$, which adds an additive term in $\Delta_{\ngroup}$ to
  the advantage function. It is imperative that \ngroup is rerandomisable as otherwise embedding the (randomised)
  challenge would be problematic.
  
  Secondly, whenever group elements are submitted to one of the oracles, there is a (tiny) probability of collisions of
  group elements. As it is comparable to the chances of guessing discrete logarithms in prime order groups, which is
  mostly ignored in proofs, we omit a complete analysis as it wouldn't contribute any meaningful insights.

  In conclusion, after sampling all keys from $\nGU$ and accounting for possible collisions in the gap oracles, the
  proof for nominal groups works as shown in \cref{thm:mmpke_security}.
\end{proof}

%%% Local Variables:
%%% mode: latex
%%% TeX-master: "main"
%%% End:
