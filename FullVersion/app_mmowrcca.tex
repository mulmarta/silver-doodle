% !TEX root = main.tex
% !TeX spellcheck = en_US
\section{One-Wayness Security of mmPKE}\label{sec:mmowrcca}

In this section, we define One-Wayness under Relaxed Chosen Ciphertext Attacks security of mmPKE, \mmowrcca. Moreover, we prove that \mmowrcca security is implied by \mmindrcca security for schemes with large message spaces.

\paragraph{Motivation.} We note that one-wayness security for mmPKE is less straightforward to define than for standard PKE schemes. Roughly, for standard PKE, one-wayness requires that given an encryption of a random message chosen by the challenger, no adversary can find the encrypted message. For mmPKE, the input to encryption is not a single message but a vector of messages. Moreover, even if the adversary corrupts recipients of some messages in the vector, it still should not be able to find the remaining messages. Therefore, it is now less clear how the challenge message vector should be chosen.
The definition presented in this section is precisely what is needed for the security proof of \protITK. We do not claim that it is the ``right'' notion, as it may not be suited to other applications.

\begin{figure*}[!tbp]
  \begin{gamebox}{\mmowrcca}\flushleft
    \begin{minipage}[t]{.35\linewidth}
      \algoHead{$\experiment{\mmindrcca}{\mmPKE, N}(\Adv)$}
      \begin{algorithmic}
        \State $(\Adv[1],\Adv[2]) \gets \Adv$
        \For{$\idxUser\in[\nUsers]$}
           $(\mmpkepk_\idxUser, \mmpkesk_\idxUser)\gets \mmpkeKeyGenL()$
        \EndFor
        \State $\term{Corr} \gets \emptyset$
        \State $(\vec\mmpkepk, \vec m, S, \mathit{st}) \gets \mathcal A_1^{\text{Dec}_1, \text{Cor}}(\mmpkepk_1, \dots, \mmpkepk_\nUsers)$

        \State \KwReq{} $\abs{\vec m}=\abs{\vec\mmpkepk} \land S \subseteq [\abs{\vec m}]$
        \State $EK^* \gets \{\vec\mmpkepk[j] : j \in S \}$
        \State $m^* \getsr \mathcal M$
        \For{$j \in S$}
          $\vec m[j] \gets m^*$
        \EndFor
        \State $m' \gets \mathcal A^{\text{Dec}_2, \text{Cor}}(\mmpkeEncL(\vec\mmpkepk, \vec m), \mathit{st})$
        \State \KwReq{} $EK^* \subseteq \{\mmpkepk_i : i \in [N] \setminus \term{Corr} \}$
        \State \Return $m^*=m'$
      \end{algorithmic}
    \end{minipage}
    \hfill
    \begin{minipage}[t]{.2\linewidth}
      \algoHead{Oracle Dec$_1(i,c)$}
      \begin{algorithmic}
        \State \KwReq{} $\idxUser\in[\nUsers]$
        \State \Return $\mmpkeDec(\vec\mmpkesk_i,c)$
      \end{algorithmic}

      \medskip
      \algoHead{Oracle Cor$(\idxUser)$}
      \begin{algorithmic}
        \State \KwReq{} $\idxUser\in[\nUsers]$
        \State $\term{Corr} \setadd \idxUser$
        \State \Return $\mmpkesk_\idxUser$
      \end{algorithmic}
      \end{minipage}\hfill\begin{minipage}[t]{.3\linewidth}
      \algoHead{Oracle Dec$_2(i,c)$}
      \begin{algorithmic}
        \State \KwReq{} $\idxUser\in[\nUsers]$
        \State $m \gets \mmpkeDecL(\vec\mmpkesk_i,c)$
        \If{$\mmpkepk_i \in EK^* \land m=m^*$}
          \State \Return $\literal{test}$
        \Else\
          \Return $m$
        \EndIf
      \end{algorithmic}
    \end{minipage}
  \end{gamebox}
  \caption{Experiment defining \mmowrcca security of mmPKE schemes.}
  \label{fig:mmpke_ow_rcca}
\end{figure*}

\paragraph{The game.} The \mmowrcca game is defined in \cref{fig:mmpke_ow_rcca}, the challenge ciphertext is computed as follows: The adversary sends a public-key vector, as well as a message vector $\vec m$ and a set of indices $S$ within this vector. The challenger then inserts the same random message $m^*$ into all positions in $\vec m$ indicated by $S$ (the previous values of $\vec m$ at these positions are ignored). It encrypts the result and sends the ciphertext to the adversary, whose goal is to find $m^*$.

\paragraph{Remarks.} First, we note that the \mmowrcca game has no notion of leakage. Instead, the leakage is implicit in how the vector encrypted by the challenger is chosen --- the ``leakage'' is everything the adversary knows about that vector, such as whether two slots contain the same message or not.

Second, the game allows the adversary to verify if some message $m'$ is the correct solution $m^*$. This can be done by sending $m'$ to the decrypt oracle and checking if it returns \literal{test}. This additional ability makes the notion stronger (i.e., more difficult to achieve). We show that \mmindrcca security is sufficient to achieve it.


\begin{definition}[\mmowrcca]\label{def:mmowrcca}
For an \mmPKE with message space $\mathcal M$, the advantage of an adversary $\Adv$ against \emph{One-Wayness under Replayable Chosen Ciphertext Attacks (\mmowrcca)} security of \mmPKE is defined as
\begin{equation*}
  \textnormal{Adv}^\mmowrcca_{\mmPKE,N}(\Adv) = \Pr\left[\experiment{\mmowrcca}{\mmPKE,N}(\Adv) \Rightarrow 1 \right],
\end{equation*}
where $\experiment{\mmowrcca}{\mmPKE,N}(\Adv)$ is defined in \cref{fig:mmpke_ow_rcca}.
\end{definition}

\paragraph{Relation to \mmindcca.}
We prove that \mmowrcca security is implied by \mmindrcca for schemes with large message spaces.

\begin{theorem}
Let \mmPKE be an mmPKE scheme with message space $\mathcal M$. For any adversary $\Adv$, there exists an adversary $\Bdv$ such that
\begin{equation*}
  \textnormal{Adv}^\mmowrcca_{\mmPKE,N}(\Adv) \leq \textnormal{Adv}^\mmindrcca_{\mmPKE,N}(\Bdv) + \frac{2}{\mathcal M}.
\end{equation*}
\end{theorem}
\begin{proof}
  The proof closely follows the typical proofs showing that IND security implies OW security for standard encryption.
  
  Given an adversary $\Adv$ against \mmowrcca security, the reduction $\Bdv$ attacking \mmindrcca simply runs $\Adv$ on the public keys it receives in the \mmindrcca experiment and forwards all $\Adv$'s oracle queries to its \mmindrcca oracles.
  When $\Adv$ outputs the triple $(\vec\mmpkepk, \vec m, S)$, $\Bdv$ computes the challenge ciphertext as follows.
  First, it initializes $\vec m_0^*, \vec m_1^* \gets \vec m$. Then, it picks two random messages $m_0^*$ and $m_1^*$ and for each $j\in S$ sets $\vec m_0^*[j] \gets m_0^*$ and $\vec m_1^*[j] \gets m_1^*$. It sends $\vec\mmpkepk$ together with $\vec m_0^*$ and $\vec m_1^*$ to the \mmindrcca experiment, receives the challenge ciphertext $c^*$ and sends it to $\Adv$.
  At the end of the experiment, $\Adv$ outputs a guess $m'$. If $m'=m_1^*$, then $\Bdv$ outputs 1. Else, it outputs 0.

  First, it is easy to see that if $\Adv$ does not violate any \KwReq{} statements in the emulation, then $\Bdv$ does not violate any \KwReq{} statements in the \mmindrcca game. In particular, $\vec m_0^*$ and $\vec m_1^*$ clearly have the same leakage.
  It is also easy to see that if $\Adv$ does not trivially win by corruptions then $\Bdv$ does not either.

  Second, observe that if $\Bdv$'s challenger uses the bit $b=1$, then $\Bdv$ emulates $\Adv$'s experiment perfectly, unless $\Adv$ inputs to $\oracle{Dec}_2$ something that decrypts to $m_0^*$. The reason is that in this case $\Bdv$ replies with \literal{test} (forwarded from its oracle), while $\Adv$ should receive $m_0^*$. Since $m_0^*$ is random and independent of $\Adv$'s view, this happens with probability at most $1/\mathcal M$. Thus, it is easy to see that
  \[\Pr\left[\experiment{\mmindrcca}{\mmPKE,N,n,1}(\Bdv)\Rightarrow 1\right] \leq \textnormal{Adv}^\mmindrcca_{\mmPKE,N,n}(\Adv) + \frac{1}{\mathcal M}.\]

  If $\Bdv$ is in the experiment with the bit $b=0$, then $m_1^*$ is independent of $\Adv$'s view, so the probability that it outputs $m'=m_1^*$ and hence also that $\Bdv$ outputs $1$ is at most $\frac{1}{\mathcal M}$.
\end{proof}

%%% Local Variables:
%%% mode: latex
%%% TeX-master: "main"
%%% End:
