% !TEX root = main.tex
% !TeX spellcheck = en_US

Continuous Group Key Agreement (CGKA) -- or Group Ratcheting -- lies at the
heart of a new generation of \emph{scalable} End-to-End secure (E2E)
cryptographic multi-party applications. One of the most important (and first
deployed) CGKAs is \protITK which underpins the IETF's upcoming Messaging
Layer Security E2E secure group messaging standard. To scale beyond the group
sizes possible with earlier E2E protocols, a central focus of CGKA protocol
design is to minimize bandwidth requirements (i.e. communication
complexity).

In this work, we advance both the theory and design of CGKA culminating in
an extremely bandwidth efficient CGKA. To that end, we first generalize
the standard CGKA communication model by introducing \emph{server-aided} CGKA
(saCGKA) which generalizes CGKA and more accurately models how most E2E protocols are deployed in
the wild. Next, we introduce the \saik protocol; a modification of \protITK,
designed for real-world use, that leverages the new capabilities available to
an saCGKA to greatly reduce its communication (and computational) complexity
in practical concrete terms.

Further, we introduce an intuitive, yet precise, security model for saCGKA.
It improves upon existing security models for CGKA in several ways. It more
directly captures the intuitive security goals of CGKA. Yet, formally it also
relaxes certain requirements allowing us to take advantage of the saCGKA
communication model. Finally, it is significantly simpler making it more
tractable to work with and easier to build intuition for. As a result, the
security proof of \saik is also simpler and more modular.

Finally, we provide empirical data comparing the (at times, quite
dramatically improved) complexity profile of \saik to state-of-the art CGKAs.
For example, in a newly created group with 10K members, to change the group
state (e.g. add/remove parties) \protITK requires each group member download
1.38MB. However, with \saik, members download no more than 2.7KB.

%%% Local Variables:
%%% mode: latex
%%% TeX-master: "main"
%%% End:
