\section{\aclp{hrs}}\label{sec:sigs}
%\msnote{This text may go to the intro, but I needed something to build on.}
Consider a scenario in which a sender uploads a vector of messages $\vec m$ on a server and later each of a number of recipients wants to download a sub-vector of $\vec m$ and verify its authenticity. One naive solution would be to sign $\vec m$ using a regular signature scheme. However, this has a high receiver-communication cost, because receivers have to download the elements of $\vec m$ they are not interested in. Another naive solution would be to sign each sub-vector of $\vec m$ that someone may be interested in. However, this has a high sender-communication cost.
Reducible signatures provide a better solution which minimizes both sender and receiver communication cost. Here, the sender uploads a single signature, which can be later personalized for different recipients by the (untrusted) server.

%\subsection{Syntax and Correctness}

\paragraph{Hedging.}
Our application requires that message vectors are authenticated under both an asymmetric key pair of the sender and a symmetric key shared among the sender and all recipients. Therefore, we define a variant called \emph{Hedged Reducible Signatures (HRS)}, which means that signing and verification algorithms take as input both an asymmetric signing/verification key and a symmetric key. Intuitively, security is guaranteed as long as either the symmetric or asymmetric signing key is secure (the other one can be arbitrary). Observe that if the symmetric key is set to a fixed value, then HRS collapses to asymmetric reducible signatures. If instead the asymmetric keys are set to a fixed value, then HRS collapses to reducible MAC's.

The reason for defining a single primitive that combines reducible signatures and reducible MAC's instead of signatures and MAC's separately is allowing more efficient schemes. For example, for our construction using separate signatures and MAC's would double the communication cost.

\paragraph{Reducible signatures.}
A reducible (asymmetric or symmetric) signature scheme allows to sign a vector of messages $\vec m$ such that later anyone, without the secret key, can compute a signature $\sigma'$ on a sub-vector $\vec m'$ of $\vec m$. The signature on the reduced vector $\vec m'$ is typically larger than the original signature on $\vec m$, because it contains hints for the verifier about the missing parts of $\vec m$.

The signer controls how the signed message $\vec m$ can be reduced by specifying a set of allowed reduction patterns $\rd$. Formally, we define
\begin{definition}
	A \emph{reduction pattern} $\rd$ for message vectors of length $n$ is any subset of $[n]$. A vector of messages $\vec m$ \emph{reduced according to} $\rd$, denoted $\rd(\vec m)$, is the sub-vector of $\vec m$ consisting of $\vec m[i]$ for $i \in \rd$.
	%Applying a reduction pattern $\rd$ to a message vector $\vec m$, denoted $\rd(\vec m)$, returns the sub-vector of $\vec m$ consisting of $\vec m[i]$ for $i \in \rd$.
\end{definition}

When checking a signature $\sigma'$ on $\vec m'$, the verifier also specifies a reduction pattern $\rd$ according to which, in their belief, the signed vector $\vec m$ was reduced to obtain $\vec m'$. (Security will require that if verification passes, then this is indeed true). We assume that the verifier knows $\rd$ out-of-band (this will be the case for CGKA). However, if needed, it can be sent together with the signature $\sigma'$.

%\paragraph{Hedging.}
%We define a variant of reducible signatures called \emph{hedged} reducible signatures. This means that signing and verification algorithms take as input a symmetric key in addition to the asymmetric signing/verification key.
%Intuitively, unforgeability is guaranteed as long as either the symmetric or asymmetric signing key is secure (the other one can be arbitrary). Observe that if the symmetric key is set to a fixed value, then the scheme collapses to asymmetric (reducible) signature. If instead the asymmetric keys are set to a fixed value, then it collapses to (reducible) MAC.
%
%For regular signatures, the above goal can be achieved by simply MAC'ing the message in addition. However, for reducible signatures, we would need a reducible MAC. Instead of defining two primitives, we define one that generalizes both.

\paragraph{Weights.}
In order to enable more efficient schemes, we define \emph{weighted} HRS. That is, the signer assigns to each allowed reduction pattern $\rd$ an integer weight $w$. This allows to construct schemes that minimize the weighted sum of signatures $\sigma'$ on all reduced message vectors. Looking ahead, $w$ will be the number of recipients downloading the sender's message reduced according to $\rd$ and the weighted sum will be the total receiver communication cost.
We note that weights have no meaning in the context of security.
%
Formally, the signing algorithm will take as input a class of allowed reduction patterns defined as follows.
\begin{definition}
  A \emph{reduction pattern class} $\rdclass$ for message vectors of length $n$ is a set of pairs $(\rd, w)$, where $\rd$ is a reduction pattern for message vectors of length $n$ and $w \in \N$ is a weight.
\end{definition}

\paragraph{Formal syntax.}
For message vectors of length $n$, we will denote by $\rdclassset_n$ the set of all reduction pattern classes supported by an HRS scheme. This means that the supported reduction patterns are all $\rd$ contained in some $\rdclass \in \rdclassset_n$.

\begin{definition}[$\ers$]
  A \emph{Hedged Reducible Signature (HRS)} scheme $\ers$ for a collection of reduction pattern classes $\rdclassset_n$ for $n \in \N$ consists of the following algorithms:
  \begin{itemize}[align=left, nosep]
    \item[] $\erskeygen \getsl (\erssk,\ersvk)$: Generates a new signing/verification key pair.
    \item[] $\erssign(\erssk, k, \vec{m},\rdclass)\getsl \sigma$: On input a signing key $\erssk$, a symmetric key
    $k \in \bits^\kappa$, a vector of messages $\vec{m}$ and a reduction pattern class $\rdclass\in\rdclassset_{|m|}$,
    outputs a signature $\sigma$.
    \item[] $\ersred(\ersvk, \sigma, \vec{m}, \rd) \getsl \sigma'\vee\bot$: On input a verification key $\ersvk$, a signature $\sigma$, a vector of messages $\vec m$ and a reduction pattern $\rd$, outputs a signature $\sigma'$ authenticating $\rd(\vec m)$ (or $\bot$ in case the operation fails).
    \item[] $\ersvrfy(\ersvk, k, \vec{m}, \rd, \sigma')\rightarrow 0\vee 1$: On input a verification key $\ersvk$,
    a symmetric key $k$, a vector of messages $\vec{m}$, a reduction pattern $\rd$ and a signature $\sigma'$, outputs $1$ for accept or $0$ for reject.
  \end{itemize}
\end{definition}

\begin{definition}
  An HRS scheme \ers for a collection of reduction pattern classes $\rdclassset_n$ for $n\in\N$ is (perfectly) correct if for all $n \in \N$, $k\in\bits^\kappa$, $\rdclass \in \rdclassset_n$, $(\rd,w) \in \rdclass$ and message vectors $\vec m$ of length $n$, we have
  \begin{equation*}
    \Pr\left[
    \begin{array}{c}
      \ersvrfy(\ersvk, k, \rd(\vec{m}), \rd, \sigma') = 1
    \end{array}
    \middle\vert
    \begin{array}{c}
      (\erssk, \ersvk) \gets \erskeygen()\\
      \sigma \gets \erssign(\erssk, k, \vec{m}, \rdclass) \\
      \sigma' \gets \ersred(\ersvk, \sigma, \vec{m},\rd)
    \end{array}
    \right] = 1.
  \end{equation*}
\end{definition}

\begin{remark}
  Reducible signatures should not be confused with redactable signatures (see e.g. \cite{ACNS:BBDFFK10, ACNS:PohSam14,
    ICISC:DPSS15, EPRINT:HabHorZha16}). The latter allow to sign a message $m$ such that later a censor can, without the secret key, redact parts of $m$ and compute a valid signature on the result. An important security goal is to hide from the verifier the redacted contents, or even that a redaction took place. In contrast, reducible signatures allow the verifier to \emph{check} which reduction pattern was applied, which is in conflict with the goal of redactable signatures.
\label{rem:rsig}\end{remark}

\subsection{Security}

We adapt the usual \ufcma security notion for \aclp{hrs}. Intuitively, an adversary who knows
only one of the symmetric secret and the asymmetric secret key should not be able to generate a valid signature. This is
formalized in \cref{def:ers}, where we define \emph{\underline{S}ymmetric and \underline{A}symmetric
  \underline{E}xistential \underline{U}n\underline{\smash{f}}orgeability against \underline{R}educible \underline{C}hosen \underline{M}essage
  \underline{A}ttacks}(\gamefont{S/A}\ersufcma).

\begin{definition}\label{def:ers}
  For $\gamefont{ATK} \in \{\ahrsufcma,\shrsufcma\}$ and an \ers, we define $\Exp_\ers^{\gamefont{ATK}}$ in \cref{fig:ersufcma} and the advantage of an adversary \Adv against the
  \gamefont{ATK} security of \ers as
  \begin{equation*}
    \adv{\gamefont{ATK}}{\ers}(\Adv) = \Pr\left[\Exp^{\gamefont{ATK}}_{\ers}(\Adv) = 1\right].
  \end{equation*}
\end{definition}

\begin{figure}[!tbp]
  \begin{gamebox}{\ahrsufcma, \shrsufcma}\scalebox{0.7}{\begin{minipage}[t]{0.6\linewidth}
      \algoHead{$\Exp^{\gamefont{ATK}}_{\ers}(\Adv)$, $\gamefont{ATK}\in\{\ahrsufcma,\shrsufcma\}$}
      \begin{algorithmic}
        \State $Q \gets \emptyset$
        \If{$\gamefont{ATK}=\ahrsufcma$}
          \State $(\erssk^*, \ersvk^*) \getsr \erskeygen$
          \State $(\vec m^*, \sigma^*, k^*, \rd^*)\getsr\Adv^{\oracle{Sign}_{A}(\cdot,\cdot)}(\ersvk)$
        \ElsIf{$\gamefont{ATK}=\shrsufcma$}
          \State $k^*\getsr\mathcal{K}$
          \State $(\vec m^*, \sigma^*, \ersvk^*, \rd^*)\getsr\Adv^{\oracle{Sign}_{S}(\cdot,\cdot,\cdot),
          \oracle{Verify}(\cdot,\cdot,\cdot)}()$
        \EndIf
%          \State \KwReq{} $\neg \exists (\vec m, \rdclass) \in Q: (\rd^*,*) \in \rdclass \land \vec m^* = \rd^*(\vec m) $
        \For{\bf each $(\vec m, \rdclass) \in Q$}
          \State \KwReq{} $\nexists w : (\rd^*,w) \in \rdclass \land \vec m^* = \rd^*(\vec m)$
        \EndFor
        \State\Return $\ersvrfy(\ersvk^*, k^*, m^*, \rd^*, \sigma^*)$
      \end{algorithmic}
%      \medskip
%      \algoHead{$\Exp^{\shrsufcma}_{\ers}(\Adv)$}
%      \begin{algorithmic}
%        \State $Q \gets \emptyset$
%        \State $k\getsr\mathcal{K}$
%        \State $(\vec m^*, \sigma^*, \ersvk^*, \rd^*)\getsr\Adv^{\oracle{Sign}_{S}(\cdot,\cdot,\cdot),
%          \oracle{Verify}(k,\cdot,\cdot)}(\ersvk)$
%        \For{\bf each $(\vec m, \rdclass) \in Q$}
%          \State \KwReq{} $\nexists w : (\rd^*,w) \in \rdclass \land \vec m^* = \rd^*(\vec m)$
%        \EndFor
%        \State\Return $\ersvrfy(\ersvk^*, k, \vec m^*, \rd^*, \sigma^*)$
%      \end{algorithmic}

    \end{minipage}}\hfill\scalebox{0.7}{\begin{minipage}[t]{.35\linewidth}
      \algoHead{Oracle $\oracle{Sign}_{A}(k', \vec{m}, \rdclass)$}
      \begin{algorithmic}
        \State \KwReq{} $\rdclass \in \rdclassset_{\vert\vec{m}\vert}$
        \State $Q \setadd (\vec m, \rdclass)$
        \State \mbox{\Return $\erssign(\erssk^*, k', \vec m, \rdclass)$}
      \end{algorithmic}
    \end{minipage}}\hfill\scalebox{0.7}{\begin{minipage}[t]{.37\linewidth}
      \algoHead{Oracle $\oracle{Sign}_{S}(\erssk', \vec{m}, \rdclass)$}
      \begin{algorithmic}
        \State \KwReq{} $\rdclass \in \rdclassset_{\vert\vec{m}\vert}$
        \State $Q \setadd (\vec m, \rdclass)$
        \State \mbox{\Return $\erssign(\erssk', k^*, \vec m, \rdclass)$}
      \end{algorithmic}
      \medskip
      \algoHead{Oracle $\oracle{Verify}(\ersvk', \vec m, \rd, \sigma)$}
      \begin{algorithmic}
        \State \mbox{\Return $\ersvrfy(\ersvk', k^*, \vec m, \rd, \sigma)$}
      \end{algorithmic}

    \end{minipage}}
  \end{gamebox}
  \caption{Asymmetric and symmetric unforgeability games for HRS.}
  \label{fig:ersufcma}
\end{figure}

In addition to unforgeability, we also require our \acl{hrs} to be \emph{Key Committing}. Intuitively, it should not be
possible for an adversary to find different symmetric keys, such that a signature is valid under both keys. We define
the security notion in \cref{sec:rkc}, \cref{def:rkc} and prove that our construction achieves this notion in \cref{sec:app-hrs-eo}.

\subsection{Construction}\label{sec:red_sig}
\paragraph{Supported reductions.}
We observe that the messages sent in the CGKA protocol consists of three chunks of data, from which each user requires a
different type of subset. First, there is data that every user needs, which is mainly auxiliary data. Then it includes
all new public keys from the sender's leaf in the tree to the root. Here, each user only needs the prefix up to its
lowest common ancestor with the sender. Lastly, it contains a list of ciphertexts, from which each user can decrypt
exactly one. Note that the common part for all users can equivalently be handled by including this fixed message as the
first message of the list and requiring prefixes to be non-empty. Since such a reduction pattern seems more versatile,
we opt to define our patterns this way.

\begin{definition}\label{def:red_class}
  We let $\rdclasssetBGM_n = \{\rdclassBGM_{\ell,\vec w} \mid \ell\in[n], \vec w :  \vec w\in\N^{\ell+1}\}$ for $n\in\N$ with
  \begin{equation*}
    \rdclassBGM_{\ell,\vec w} \coloneqq \left\{\left([i:\min(i,\ell)]\cup [\ell + 1:\ell+j], \vec w[i]\right) \mid
      i\in[\ell+1], j\in [0:n-\ell] \right\}
  \end{equation*}
  $\rdclassBGM_{l,\vec w}$ is completely described by $\ell$ and $\vec{w}$ (for a fixed $n$), so we will use them as
  input to all algorithms instead. Similarly, for every $(\rd, w) \in \rdclassBGM_{l,\vec w}$, $\rd$ is uniquely defined
  by the integers $\ell, i, j$, so we use that representation here too.
\end{definition}

\paragraph{Construction.}
Now we show the construction of the \acl{hrs} scheme \ers itself. We adapt the construction of
\cite{ICISC:DPSS15} to efficiently support the structure of \Cref{def:red_class} by using a collision-resistant hash
function in addition to a \ufcma secure signature, a (weighted) cryptographic accumulator and a MAC.
Note that the auxiliary accumulator information isn't included in the signature after the reduction as $\rdclassBGM_{l,
  w}$ allows only one singleton element.
In \cref{sec:app-hrs}, we prove the following theorem.

\begin{figure}[!t]\vspace*{-2em}
  \begin{algobox}{$\bgmHrs[\hash, \acc, \sigscheme, \mac]$}
  \scalebox{.7}{
    \begin{minipage}[t]{.39\linewidth}
      \algoHead{\erskeygen}
      \begin{algorithmic}
        \State \Return $\sigkg()$
      \end{algorithmic}

      \medskip
      \algoHead{\erssign$(\erssk, k, \vec{m}, (\ell, \vec{w}))$}
      \begin{algorithmic}
        \State $X \gets \{((i, \vec m[i]), \vec w[i]) \mid i\in[\ell]\}$
        \State $(\accValue, \accaux) \getsr \acc.\accEval(X)$
        \State $\hat{h} \gets 0$
        \For{$t = \abs{\vec m}$ to $\ell+1$}
          \State $\hat{h} \gets \hash(\vec m[t], \hat{h})$
        \EndFor
        \State $h \gets \hash(\ell, \accValue, \hat{h})$
        \State $\mtag \gets \mactag(k, h)$
        \State $\sig \gets \sigsign(\ssk, (h, \mtag))$
        \State\Return $(\sig, \mtag, \accValue, \accaux)$
      \end{algorithmic}
    \end{minipage}
  }
  \scalebox{.7}{\hspace*{-1em}
    \begin{minipage}[t]{.42\linewidth}
      \algoHead{\ersred$(\ersvk, \vec{m}, \sigma, \rd=(\ell, i, j))$}
      \begin{algorithmic}
        \State \KwParse{} $(\sig, \mtag, \accValue, \accaux) \gets \sigma$
        \State \KwReq{} $i \in [\ell + 1] \land j \in [0:\abs{\vec m}-\ell]$
        \If{$i \in [\ell]$}
          \State $\accProof \gets \acc.\accProve(\accValue, \vec m[i], \accaux)$
        \Else
          \State $\accProof \gets \bot$
        \EndIf
        \State $\hat{h} \gets 0$
        \For{$t = \abs{\vec m}$ to $\ell+j+1$}
          \State $\hat{h} \gets \hash(\vec m[t], \hat{h})$
        \EndFor
        \State \Return $(\sig, \mtag, \accValue, \accProof, \hat h)$
      \end{algorithmic}
    \end{minipage}
  }
  \scalebox{.7}{\hspace*{-1em}
    \begin{minipage}[t]{.53\linewidth}
      \algoHead{\ersvrfy$(\ersvk, k, \vec{m}, \rd = (\ell, i, j), \sigma')$}
      \begin{algorithmic}
        \State \KwParse{} $(\sig, \mtag, \accValue, \accProof, \hat h) \gets \sigma'$
        \If{$i \notin [\ell+1]$} \Return $0$ \EndIf
        \If{$i\neq \ell + 1$}
          \State \KwReq{} \mbox{$\abs{\vec m} = j+1 \land \acc.\accVrfy(\accValue, (i, \vec m[1]), \accProof)$}
          \State $t_0 \gets 2$
        \Else
          \State \KwReq{} {$\abs{\vec m} = j$}
          \State $t_0 \gets 1$
        \EndIf
        \For{$t = \abs{\vec m}$ to $t_0$}
           $\hat{h} \gets \hash(\vec m[t], \hat{h})$
        \EndFor
        \State $h \gets \hash(\ell, \accValue, \hat{h})$
        \State \Return $\macvrf(k, h, \mtag) \ \wedge $ \\\hspace*{2em} $\sigvrf(\ersvk, (h, \mtag)), \sig)$
      \end{algorithmic}
    \end{minipage}
  }
\end{algobox}
\caption{Hedged reducible signature scheme \ers from accumulator \acc, hash function $H$, signature
    scheme $\sigscheme$ and mac \mac. If the message wasn't reduced, verification consists of recomputing $h$ and
    verifying the signature.}
  \label{fig:red_sig}
\end{figure}

\begin{restatable}{theorem}{hrsSecurity}\label{thm:acc_sec}
  Let $\ers = \bgmHrs[\hash,$ $ \acc, \sigscheme, \mac]$ denote the scheme from \cref{fig:red_sig} instantiated with a hash $\hash$, a signature scheme $\sigscheme$, a cryptographic accumulator $\acc$ and a $\mac$.
  For any adversary $\Adv$, there exist adversaries $\Bdv[1]$, $\Bdv[2]$ and $\Bdv[3]$, and $\Bdv[1]'$, $\Bdv[2]'$ and $\Bdv[3]'$, all with roughly the same runtime as $\Adv$, s.t.
  \begin{equation*}\label{eq:hrs-sec1}
    \adv{\ahrsufcma}{\ers}({\Adv}) \leq \adv{\text{\ufcma}}{\sigscheme}({\Bdv[1]}) +
    \adv{\gamefont{CR}}{\hash}({\Bdv[2]}) +
    \adv{\gamefont{UF}}{\acc}({\Bdv[3]}) \text{ and }
  \end{equation*}
  \begin{equation*}\label{eq:hrs-sec2}
    \adv{\shrsufcma}{\ers}({\Adv}) \leq \adv{\ufcma}{\mac}({\Bdv[1]'}) +
    \adv{\gamefont{CR}}{\hash}({\Bdv[2]'}) +
    \adv{\gamefont{UF}}{\acc}({\Bdv[3]'}).
  \end{equation*}
\end{restatable}

%%% Local Variables:
%%% mode: latex
%%% TeX-master: "main"
%%% End:
