% !TEX root = main.tex
% !TeX spellcheck = en_US

\section{Details of the (sa)CGKA Security Model}\label{sec:model}
In this section, we formally define $\funcCGKA$. The code of $\funcCGKA$ is in \cref{fig:CGKA-func-active,fig:CGKA-func-active-helpers}.
% !TEX root = main.tex
% !TeX spellcheck = en_US

\begin{figure*}[!tbp]
\begin{systembox}{\normalsize$\funcCGKA$}
  \begin{flushleft}
    Parameters: $\KwConf{}(\hgnodeid)$, $\KwAuth{}(\hgnodeid, \id)$, $\pgod$

    \hrulefill
  \end{flushleft}

  {\begin{minipage}[t]{0.47\linewidth}
    \vspace*{-1em}
    {\bf Initialization} \Comment{Executed on first input.}
    \begin{algorithmic}
      \State $\ptr[\wc], \setvert[\wc] \gets \bot$
	    \State $\hgnctr \gets 0$
      \State $\ep \gets \hgnew$; $\ep.\hgorig \gets \pgod$; $\ep.\hgm \gets \{\pgod\}$
      \State $\setvert[0] \gets \ep$
      \State $\ptr[\pgod] \gets 0$
    \end{algorithmic}

    \medskip
    {\bf Input $(\keyword{Send}, \hgact), \hgact \in \{\hglu , \hgla\md\id_t, \hglr\md\id_t\}$ from $\id$}
    \begin{algorithmic}
      \State \smash{\Comment{Send inputs to sim. and allow it to reject them.}}
      \State Send $(\keyword{Send},\id,\hgact)$ to the sim. and receive $\mathit{ack}$.
      \State\KwReq{} $\mathit{ack}$
      \State \smashedComment{Compute the new epoch $\ep$ created by the action.}
      \State $\ep \gets \hgnew$; $\ep.\hgpar \gets \ptr[\id]$; $\ep.\hgorig \gets \id$
      \State $\ep.\hgact \gets \hgact$; $\ep.\hgm \gets \members(\ptr[\id],\hgact)$
      \State $\hgnctr\inc$; $\setvert[\hgnctr]\gets \ep$
      \State \smashedComment{Enforce security after possible changes to HG.}
%\Comment{Creating $\ep$ must not destroy the HG (else, both the protocol and the simulator must reject the action).}
      \State \KwAss{} $\hgConsistent \land \authPreserved$
      \State \smashedComment{Immediately transition $\id$ into the created epoch.}
      \State $\ptr[\id] \gets \hgnctr$
      \State \smashedComment{Output the idealized message chosen by adv.}
      \State Receive from the simulator $\msgS$.
      \State \Return $\msgS$		% \State \Return $(\msgS,\wmsg)$
    \end{algorithmic}

    \medskip
    {\bf Input $\keyword{GetKey}$ from $\id$}
    \begin{algorithmic}
      \State Send $(\keyword{Key},\id)$ to the simulator and receive $\ckagk$.
      \State $\hgnodeid \gets \ptr[\id]$
      \State \KwReq{} $\hgnodeid \neq \bot$
      \If{$\setvert[\hgnodeid].\hgk = \bot$}
        \ExtendedState{\Comment{{Set the key according to \KwConf{}.}}}
        \If{$\KwConf{}(\hgnodeid)$}
          \State $\setvert[\hgnodeid].\hgk \getsr \bits^\secparam$
          \State $\setvert[\hgnodeid].\hgc \gets \true$
        \Else
          \State $\setvert[\hgnodeid].\hgk \gets \ckagk$
        \EndIf
      \EndIf
      \State \Return $\setvert[\hgnodeid].\hgk$
    \end{algorithmic}

    \medskip
    {\bf Corruption $(\keyword{Expose},\id)$}
    \begin{algorithmic}
      \If{$\ptr[\id]\neq\bot$}
        \smash{\Comment{Record exposure.}}
    	  \State $\setvert[\ptr[\id]].\hgexp \setadd \id$
      \EndIf
      \State \vspace*{-.1em}\Comment{Disallow adaptive corruptions to avoid commitment problem.}
      \State $\textbf{only allowed if}\ \nexists\hgnodeid : \setvert[\hgnodeid].\hgc \land \neg\KwConf{}(\hgnodeid)$
    \end{algorithmic}


  \end{minipage}}\hspace*{1em}\hfill
  {\begin{minipage}[t]{0.52\linewidth}
    {\bf Input $(\keyword{Receive},\msgR)$ from $\id$}
    \begin{algorithmic}
      \State \smashedComment{Send inputs to sim. and allow it to reject them.}
      \State Send $(\keyword{Receive},\id,\msgR)$ to the simulator and receive $\mathit{ack}$.
      \State\KwReq{} $\mathit{ack}$
      \State \smashedComment{Ask sim. to interpret the packet.}
      \State Receive from the simulator $(\hgorig', \hgact')$.
      \If{$\hgact' = \hglr\md\id$}
        \ExtendedState{\Comment{Check that $\hgorig'$ removed $\id$ or auth. not guaranteed.}}
        \State $\variable{honestRem} \gets \exists \hgnodeid : \big( \setvert[\hgnodeid].\hgpar = \ptr[\id]$\\\strut\hfill$ \land\  \setvert[\hgnodeid].\hgorig = \hgorig' \ \land  \setvert[\hgnodeid].\hgact = \hglr\md\id\big)$
        \State \KwAss{} $\variable{honestRem} \lor \neg \KwAuth{}(\setvert[\ptr[\id]], \hgorig')$
        \State $\ptr[\id] \gets \bot$
        \State \Return $(\hgorig', \hgact')$
      \EndIf

      \State \Comment{Ask sim, to identify the epoch $\hgnodeid$ where $\id$ transitions. If $\hgnodeid=\bot$, a new injected epoch is created.}
      \State Receive from the simulator $\hgnodeid$.
      \If{$\hgnodeid = \bot$}
        \State $\ep \gets \hgnew$
        \State $\ep.\hgorig \gets \hgorig'$; $\ep.\hgact \gets \hgact'$; $\ep.\hginj \gets \true$
        \If{$\ptr[\id]\neq\bot$}
          \State\vspace*{-.2em} \Comment{If $\id$ is a member, compute $\ep.\hgpar$ and $\ep.\hgm$ using its epoch.}
          \State $\ep.\hgpar \gets \ptr[\id]$
          \State $\ep.\hgm \gets \members(\ptr[\id],\hgact')$
        \Else
          \State\vspace*{-.2em} \Comment{If $\id$ joined, $\ep$ is a detached root with arbitrary member set.}
          \State Receive $\ep.\hgm$ from the simulator; set $\ep.\hgpar\gets\bot$.
        \EndIf
        \State $\hgnctr\inc$; $\setvert[\hgnctr] \gets V$
        \State $\hgnodeid \gets \hgnctr$
      \EndIf
      \State \KwAss{} $\setvert[\hgnodeid] \neq \bot$

      \State \Comment{If a current group member transitions to a detached root, attach it.}
      \If{$\ptr[\id]\neq\bot \land \setvert[\hgnodeid].\hgpar=\bot$}
         $\setvert[\hgnodeid].\hgpar\gets\ptr[\id]$
      \EndIf

      \State \KwAss{} $\ptr[\id]=\bot \lor \setvert[\hgnodeid].\hgpar = \ptr[\id]$
      \State \KwAss{} $\ptr[\id]\neq\bot \lor \setvert[\hgnodeid].\hgact = \hgla\md\id$
      \State \Comment{Enforce security after possible changes to HG.}
  		\State \KwAss{} $\hgConsistent \land \authPreserved$


      \State \Comment{Transition $\id$ and compute its output.}
      \State $\ptr[\id] \gets \hgnodeid$
      \If{$\setvert[\hgnodeid].\hgact=\hgla\md\id$}
         \Return $(\setvert[\hgnodeid].\hgorig,\setvert[\hgnodeid].\hgm)$
      \Else
        \ \Return $(\setvert[\hgnodeid].\hgorig,\setvert[\hgnodeid].\hgact)$
      \EndIf
    \end{algorithmic}

  \end{minipage}}


\end{systembox}
%\vspace*{-0.7em}
%\caption{}\label{fig:CGKA-func-active}
%\end{figure}
%
%\begin{figure}[p]

\caption{The ideal CGKA functionality.}\label{fig:CGKA-func-active}
\end{figure*}

\begin{figure*}[tbp]
  \begin{systembox}{$\funcCGKA$: Additional Methods}
     {\begin{minipage}[t]{0.47\linewidth}
%    \vspace*{-1em}
%    \algoHead{Helpers}

    {\bf Helper $\hgnew$}
    \begin{algorithmic}
      \State \Return new epoch with $\hgorig=\bot$, $\hgpar=\bot$, $\hgact=\bot$, $\hgm=\emptyset$, $\hginj=\false$, $\hgk=\bot$, $\hgexp=\emptyset$, $\hgc=\false$.
    \end{algorithmic}

    \medskip
    {\bf Helper $\members(\hgnodeid,\hgact)$}
    \begin{algorithmic}
      \State $G \gets \setvert[\hgnodeid].\hgm $
      \If{$\hgact=\hgla\md\id_t$}
        \State $G \setadd \id_t$
      \ElsIf{$\hgact=\hglr\md\id_t$}
        \State $G \setrem \id_t$
      \EndIf
      \If{$\hgact\neq\hglu \land G=\setvert[\hgnodeid].\hgm $}
        \State \Return $\bot$
      \EndIf
      \State \Return $G$
    \end{algorithmic}
  \end{minipage}}\hfill
  {\begin{minipage}[t]{0.49\linewidth}

    {\bf Helper \hgConsistent}
    \begin{algorithmic}
      \State \vspace*{-.6em} \Comment{True if HG is a forest and membership is consistent.}
      \State \Return $\true$ iff
      \State \hspace*{1.5em} a) $\forall \id$ s.t. $\ptr[\id] \neq \bot \ : \ \id \in \setvert[\ptr[\id]].\hgm$
      \State \hspace*{1.5em} b) HG has no cycles
      \State \hspace*{1.5em} c) $\forall \hgnodeid \in [\hgnctr]\ :\ \setvert[\hgnodeid].\hgm\neq\bot$
      \State \hspace*{1.5em} d) $\forall \hgnodeid \in [\hgnctr]$ s.t. $\setvert[\hgnodeid].\hgpar\neq\bot\ :$\\\hspace*{4em} $\ \setvert[\hgnodeid].\hgm = \members(\setvert[\hgnodeid].\hgpar,\setvert[\hgnodeid].\hgact)$
    \end{algorithmic}

    \medskip
    {\bf Helper \authPreserved}
    \begin{algorithmic}
      \State \vspace*{-.6em}\Comment{{True if there is no authentic epoch created by injected packet. Observe that the root $\hgnodeid=0$ cannot be injected by definition.}}
      \State \Return $\nexists\hgnodeid :  1 \leq \hgnodeid \leq \hgnctr \land \setvert[\hgnodeid].\hginj$ \\\strut\hfill $ \land\ \KwAuth(\setvert[\hgnodeid].\hgpar, \setvert[\hgnodeid].\hgorig)$
    \end{algorithmic}

  \end{minipage}}
  \end{systembox}

  \caption{The ideal CGKA functionality: additional methods.}\label{fig:CGKA-func-active-helpers}
\end{figure*}

%%% Local Variables:
%%% mode: latex
%%% TeX-master: "main"
%%% End:

%
\paragraph{Notation.}
We use the keyword \KwAss{} \textit{cond} to restrict the simulator's actions. Formally, if the condition \textit{cond}
is false, then the functionality permanently halts, making the real and ideal worlds easily distinguishable. Further, we
use {\bf only allowed if} \textit{cond} to restrict the environment. That is, our statements quantify only over
environments who, when interacting with $\funcCGKA$ and any simulator, never make \textit{cond} false\footnote{A relaxed
  restriction would require that $\Adv$ makes \textit{cond} false with a small probability $\epsilon$. In our case
  $\Adv$ knows if it violates \textit{cond}, so fixing $\epsilon=0$ is without loss of generality.}.
Finally, we write \emph{receive from the simulator} to denote that the functionality sends a dummy value to it, waits until it sends a value back and asserts via \KwAss{} that the received value is of the correct format.

\paragraph{State.}
$\funcCGKA$ maintains a history graph represented as an array $\setvert$, where $\setvert[\hgnodeid]$ denotes the epoch identified by an integer $\hgnodeid$. We use the standard object-oriented notation for epochs. In particular, each epoch $\ep$ has a number of attributes listed in \cref{tab:state} ($\ep.\hginj$, $\ep.\hgexp$ and $\ep.\hgc$ are related to corruptions).
 Apart from $\setvert$, $\funcCGKA$ stores an array $\ptr$, where $\ptr[\id]$ denotes the current epoch of the party $\id$.

\begin{table}[!tb]
\noindent	\begin{center}\scalebox{.9}{\begin{tabularx}{.5\textwidth}{| l | X |}
		\hline
		$\ep.\hgpar$ & The integer identifier of the parent epoch.\\
		\hline
    $\ep.\hgorig$ & The party who created the epoch by performing a group operation.\\
		\hline
    $\ep.\hgact$ & The group modification performed when $E$ was created: either $\hglu$ for update, or $\hgla\md\id_t$ for adding $\id_t$, or $\hglr\md\id_t$ for removing $\id_t$.\\
		\hline
    $\ep.\hgm$ & The set of group members.\\
		\hline
    $\ep.\hgk$ & The shared group key.\\
		\hline
    $\ep.\hginj$ & A boolean flag indicating if the epoch is injected.\\
		\hline
    $\ep.\hgexp$ & The set of group members exposed (i.e., corrupted) in this epoch.\\
		\hline
    $\ep.\hgc$ & A flag indicating if a random group key has been outputted.\\
		\hline
	\end{tabularx}}\end{center}
\caption{Attributes on an epoch in $\funcCGKA$.}\label{tab:state}.
\end{table}

\paragraph{Inputs from parties.}
The first two inputs, Send and Receive, are handled quite similarly. First, all inputted values are sent to the simulator (there are no private inputs). Second, the simulator sends a flag {\it ack} which decides if sending/receiving succeeds (or fails with output $\bot$). Third, $\funcCGKA$ updates the history graph and enforces that this does not destroy authenticity and consistency by checking that $\authPreserved$ and $\hgConsistent$ are true. Finally, $\funcCGKA$ transitions the sender/receiver to the new epoch (or removes its pointer in case it is removed) and computes the output using the new graph.

One aspect that needs more explanation is updating the graph when a party $\id$ receives $c$. In this case, the simulator interprets $c$ for $\funcCGKA$ (which abstracts away ciphertexts) by providing the sender $\hgorig'$ and the action $\hgact'$. If $\hgact'$ removes $\id$, then the only possible authenticity check is that either $\hgorig'$ removed $\id$ in its current epoch or the epoch is not authentic for $\hgorig'$.
%
If $\id$ is not removed, the simulator identifies the epoch $\hgnodeid$ into which $\id$ transitions or joins. The epoch can be $\bot$, in which case $\funcCGKA$ creates a new epoch $\ep$ with the infected flag $\hginj$ set. If $\id$ is a current group member, then $\ep$ is a child of its current epoch. Otherwise, if $\id$ joins, then $\ep$ is a detached root.
Afterwards, $\funcCGKA$ checks if $\hgnodeid$ identifies a detached root into which a current group member $\id$ transitions. If this is the case, the root is attached as a child of $\id$'s current epoch. For instance, this implies that any other party transitioning to $\hgnodeid$ must do so from $\id$'s current epoch and the epoch semantic must be consistent between it, $\id$ and the party who joined into $\hgnodeid$.

The last input to $\funcCGKA$ is GetKey, which simply outputs the group key from the party's current epoch. The key is set to a random or arbitrary value the first time it is retrieved by some party.

\paragraph{Corruptions.}
When a party is corrupted, $\funcCGKA$ simply adds it to the exposed set $\hgexp$ of its current epoch. The set is later used by the security predicates. Then $\funcCGKA$ disallows corruptions in case extending the $\hgexp$ set switched \KwConf{} of some epoch $E$ with $E.\hgc$ set from true to false.



%%% Local Variables:
%%% mode: latex
%%% TeX-master: "main"
%%% End:
