% Continuous Group Key Agreement (CGKA), also called Group Ratcheting, lies at
% the heart of a new generation of End-to-End (E2E) secure group messaging
% (SGM) protocols supporting very large groups. The communication model,
% efficiency and security of these SGM protocols is inherited from their
% underlying CGKA protocol. In particular, a primary cost limiting practically
% viable group sizes is the communication complexity of certain CGKA
% operations.

\section{Introduction}
End-to-end (E2E) secure applications have become one of the most widely used
class of cryptographic applications on the internet with billions of daily
users. Accordingly, the E2E protocols upon which these applications are built
have evolved over several distinct generations, adding functionality and new
security guarantees along the way. Modern protocols are generally expected to
support features like multi-device accounts, continuous refreshing of secrets
and asynchronous communication. Here, \emph{asynchronous} refers to the
property that parties can communicate even when they are not simultaneously
online. To make this possible, the network provides an (untrusted) mailboxing
service for buffering packets until recipients come online.

The growing demand for E2E security motivates increasingly capable E2E
protocols; in particular, supporting ever larger groups. For example, in the
enterprise setting organizations regularly have natural sub-divisions with
far more members than practically supported by today's real-world E2E
protocols. Support for large groups opens the door to entirely new
applications; especially in the realm of machine-to-machine communication
such as in mesh networks and IoT. The desire for large groups is compounded
by the fact that many applications treat each device registered to an account
as a separate party at the E2E protocol level. For example, a private chat
between Alice and Bob who each have a phone and laptop registered to their
accounts is actually a 4-party chat from the point of view of the underlying
E2E protocol.

\paragraph{Next Generation E2E Protocols.}
The main reason current protocols (at least those enjoying state-of-the-art
security, {e.g. post compromise forward security}) only support small groups
is that their communication complexity grows linearly in the group size. This
has imposed natural limits on real-world group sizes (generally at or below
1000 members).

Consequently, a new generation of E2E protocols are being developed both in
academia (e.g.~%
%\cite{IST-MLS-papers,ETHZ-MLS-papers,Continous-CGKA-papers}
\cite{CCS:CCGMM18,EC:AlwCorDod19,TCC:ACJM20,EPRINT:AlwJosMul20,C:ACDT20,TCC:AABNKPPW21,SP:ACC+21})
and industry~\cite{MLS}. Their primary design goal is to support extremely
large groups (e.g. 10s of thousands of users) while still meeting, or
exceeding, the security and functionality of today's state-of-the-art
deployed E2E protocols. Technically, the new protocols do this by reducing
their communication complexity down to \emph{logarithmic} in the group size;
albeit, only under favorable conditions in the execution. This informal
property is sometimes termed the \emph{fair-weather complexity} of a
protocol.

To date, the most important of these new E2E protocols is the IETF's upcoming
secure group messaging (SGM) standard called the \emph{Messaging Layer
Security} (MLS) protocol.
% It is the product of a collaboration between
% industry practitioners and academic cryptographers.
% including Amazon, Cisco, Cloudlfare, Facebook, Mozilla,
% Twitter, Wickr and Wire and academic cryptographers (e.g. from Alto
% University, IST Austria, INRIA, MIT, Naval Postgraduate School, Oxford and
% Royal Holloway).
MLS is in the final stages of standardization and its core
components are already seeing initial deployment~\cite{Cisco-Webex-MLS}.

\paragraph{Continuous Group Key Agreement.}
To the best of our knowledge, all next gen. E2E protocols share the following
basic design paradigm. At their core lies a \emph{Continuous Group Key
Agreement} (CGKA) protocol; a generalization to the group setting of the
\emph{Continuous Key Agreement} 2-party
primitive~\cite{EC:AlwCorDod19,CSCML:DG19} underlying the Double Ratchet.

Intuitively, a CGKA protocol provides \emph{E2E secure group management} for
dynamic groups, i.e., groups whose properties may change mid-session. By
properties we mean things like the set of members currently in the group,
their attributes, the group name, the set of moderators, etc. Any change to a
group's properties initiates a fresh \emph{epoch} in the session. A CGKA
protocol ensures all group members in an epoch agree on the group's current
properties. Members will only transition to the same next epoch if they agree
on which properties were changed and by whom. Each epoch is equipped with its
own symmetric \emph{epoch key} known to all epoch members but
indistinguishable from random to anyone else. Higher-level protocols will
typically then (deterministically) expand the epoch key into a complete key
schedule which in turn can be used to, say, protect application data sent
between members (e.g. messages or VoiP data).
% \footnote{E.g. many CGKAs also ensure agreement about the sequence of changes
% making up the groups history. In contrast, \emph{concurrent} CGKAs ``only''
% ensure agreement on the \emph{set} of changes making up the
% history.~\cite{ConcurrentCGKAs} The new CGKAs in this work ensure agreement
% on the sequence of changes.}
% Moreover, the communication and computational complexity of the underlying
% CGKA typically dominates the complexity of the higher-level E2E protocol (and
% even the entire E2E secure application).
%  \msnote{does this fit? I somehow needed a link with the previous paragraph}

MLS too, is (implicitly) based on a CGKA, originally dubbed
\emph{TreeKEM}~\cite{TreeKEM}. Since its inception, TreeKEM has undergone several
substantial
revisions~\cite{TreeKEM-with-blanking-email,TreeKEM-prop-and-comm-email}
before reaching its current
form~\cite{mls-protocol-latest,EPRINT:AlwJosMul20}. For clarity, we refer to
its current version at the time of this writing as \emph{Insider-Secure
TreeKEM} (\protITK) (using the terminology of~\cite{EPRINT:AlwJosMul20} where
that version was analyzed). \protITK has already seen its first real world
deployment as a core component of Cisco's Webex conferencing
protocol~\cite{Cisco-Webex-MLS}.

\paragraph{Why Consider CGKA?}
CGKA is interesting because of the following two observations. First, CGKA
seems to be the minimal functionality encapsulating almost all of the
cryptographic challenges inherent to building next generation E2E protocols.
Second, building typical higher-level E2E applications (e.g. SGM or
conference calling) from a CGKA can be done via relatively generic, and
comparatively straightforward mechanisms. Moreover, the resulting application
directly inherits many of its key properties from the underlying CGKA;
notably their security guarantees and their communication and computational
complexities. In this regard, CGKA is to, say, SGM what a KEM is to hybrid
PKE. For the case of SGM, this intuitive paradigm and the relationship
between properties of the CGKA and resulting SGM was made formal
in~\cite{CCS:ACDT21}. In particular, that work abstracts and generalizes
MLS's construction from \protITK.

\subsection{Our Contributions}
This work makes progress on the central challenge in CGKA protocol design:
reducing communication complexity so as to support larger groups (without
compromising on security or functionality).

\paragraph{Server-Aided CGKA.}
To do this, we begin by revisiting one of the most basic assumptions about
CGKA in prior work; namely that participants communicate via an insecure
broadcast channel. Instead, we note that in almost all modern deployments of
E2E protocols parties actually communicate via an untrusted mailboxing
service implemented using an (often highly scalable) \emph{server}.
%
% Thus, we introduce a new (type) of CGKA that leverages the computational
% capabilities of the server to \emph{greatly} reduce the bandwidth
% requirements (especially for recipients) over state-of-the art CGKA
% protocols, and in particular over \protITK.
In light of this, we modify the standard communication model to make the
server explicit. Correspondingly we define a generalization of CGKA which we
call \emph{server-aided} CGKA (saCGKA). In contrast to CGKA, an saCGKA
protocol includes arbitrary instructions for the server. CGKA corresponds to
the special case where the server acts as an insecure broadcast channel.
Intuitively, the server remains untrusted and security should hold no matter
what it does. However, should it choose to follow the instructions, the
saCGKA protocol ensures correctness and availability.
%Intuitively, the untrusted server is
%\emph{not} a group member; e.g. it does not learn epoch keys. However, should
%the untrusted server choose to follow these instructions (and the network
%delivers packets faithfully) a sound saCGKA protocol ensures correctness and
%availability. Nevertheless, security is guaranteed regardless of the server's
%(and more generally, the adversary's) behaviour.


\paragraph{Semantic Security for CGKA.}
We define a new security notion for saCGKA capturing the same intuitive
guarantees as those shown for \protITK~\cite{EPRINT:AlwJosMul20} for example.
Like other notions based on the history graph paradigm of~\cite{CCS:ACDT21},
our notion is parameterized by \emph{safety} predicates that together decide
the security of a target epoch key in a given execution.

However, at a technical level our notion departs significantly from past
ones. Essentially, it relaxes the requirement that group members in an epoch
agree on and authenticate the \emph{history of network traffic} leading to
the epoch. Instead, the new notion ``only'' ensures they agree on and
authenticate the \emph{semantics} of the history; i.e. the ``meaning'' of the
traffic rather than exact packet contents. This has several interesting
consequences. First, it more directly captures our intuitive security goals.
E.g. it avoids subtle questions about what intuition is really captured when,
say, an AEAD ciphertext in a packet can be decrypted to different plaintexts
using different keys.\footnote{This can happen for widely used AEADs like
AES-GCM~\cite{C:DGRW18}.} Second, the relaxation creates wiggle
room we can use to prove security despite group members no longer having the
same view of network traffic. Finally, it allows us to relax the security of
the encryption scheme used in our construction from CCA to
\emph{replayable CCA} (RCCA)\cite{C:CanKraNie03}.\footnote{This makes sense
as RCCA was designed to relax the ``syntactic non-malleability`` of CCA to a
form of ``semantic non-malleability''.}

The new saCGKA security notion is significantly simpler (though just as
precise) compared to past ones. Past notions have been criticised for being
all but inaccessible to non-domain experts due to their complexity. In an
effort to improve this, our new notion omits/simplifies various security
features of a CGKA as long as A) they can be formalized using known
techniques and B) they can be easily achieved by known, practical and
straightforward extensions of a generic CGKA protocol (including \saik)
satisfying our notion. Thus we obtain a definition focused on the basic
properties of an (sa)CGKA with the idea that a protocol satisfying our notion
can easily be extended to a ``full-fledged'' (sa)CGKA using standard
techniques.

\paragraph{The \saik Protocol.}
Next, based on \protITK, we introduce a new saCGKA protocol called
\emph{Server-Aided \protITK} (\saik), designed for real-world use.
For example, it relies exclusively on standard cryptographic primitives and
can be implemented using the API of various off-the-shelf cryptographic
libraries. Taking advantage of the added flexibility of a saCGKA, \saik
includes a special \emph{Extract} procedure run by the server to convert a
``full packet'' uploaded by a sender into an individualized ``sub-packet'' for a particular recipient. To obtain \saik, we make the following
modifications to \protITK (and its security proof).

\paragraph{Multi-message multi-recipient PKE.} First, we
replace \protITK's use of standard (CCA secure) PKE with multi-message
multi-recipient PKE (mmPKE) ~\cite{ASIACCS:PinPoeSch14}.
Directly constructing mmPKE can result in a significantly more efficient
scheme than produced by parallel composition of standard PKE schemes (both in
terms of ciphertext sizes and computation cost of encryption).

We introduce a new security notion for mmPKE, more aligned with the needs of
(the security targeted by) \saik. It both strengthens and weakens past
notions: On the one hand, proving \saik secure demands that we equip the
mmPKE adversary of~\cite{ASIACCS:PinPoeSch14} with adaptive key compromise
capabilities. On the other hand, thanks to the relaxation to semantic
agreement, we ``only'' require RCCA
security rather than full-blown CCA used in previous'
works~\cite{TCC:ACJM20,EPRINT:AlwJosMul20}.

We prove the mmPKE construction of~\cite{ASIACCS:PinPoeSch14} satisfies our
new notion based on a form of gap Diffie-Hellman assumption, the same as in
\cite{ASIACCS:PinPoeSch14}. The reduction is tight in that the security loss
is independent of the number of parties (i.e. key pairs) in the execution
(although it does depend on the number of corrupted key pairs). Moreover, we
extend the proof to capture mmPKE constructions based on ``nominal
groups''~\cite{EC:ABHKLR21_2}. Nominal groups abstract the algebraic
structure over bit-strings implicit to the X25519 and X448 scalar
multiplication functions and corresponding twisted Edwards
curves.\cite{rfc7748}. In practical terms, this means our proofs also apply
to instantiations of~\cite{ASIACCS:PinPoeSch14} that are based on the X25519
and X448 functions.

\paragraph{Authentication and Agreement.}
Second, we modify the mechanisms used by \protITK to ensure members
transitioning to a new epoch authenticate who is making changes to the
group's properties when a new epoch is announced. Rather than sign the full
packet like in \protITK, a sender in \saik only signs a small tag common to
view of all receivers. This reduces the amount of data receivers must
download to verify the signature. We also modify how changes to the group's
properties are incorporated in the derivation of the new epoch key. Rather
include a hash of the full packet representing the change, \saik uses a hash
of a symbolic representation of the change. Proving this secure leverages the
new wiggle room in our CGKA security created by introducing semantic
agreement.

%Second, we modify some of the authentication mechanisms in \protITK. In that
%protocol, the instigator of a change to group's properties is authenticated
%by other members by verifying the instigator's signature on the packet
%announcing the change. To avoid forcing all members to download the full
%packet simply to check the signature \saik instead has the sender sign only a
%small part of the full packet. However, this leaves the remaining part of the
%packet open manipulation by the adversary.

% Second, we modify the authentication mechanisms in \protITK. In that
% protocol, the instigator of a change to group's properties is authenticated
% by other members by verifying the instigator's signature on the packet
% announcing the change. To avoid forcing all members to download the full
% packet simply to check the signature \saik instead has the sender sign only a
% short value which binds the new epoch's key and the semantics of the change
% (but not other redundant parts of the packet).
% This intuitively achieves the desired authenticity but formally proving it
% secure relies crucially on the new wiggle room in our CGKA security created
% by introducing semantic agreement.











% \paragraph{Reducible signatures.}
% Second, we replace \protITK's use of standard (EUF-CMA secure) signatures
% with a new type called \emph{Reducible Signature (RS)}. Intuitively, an RS
% allows signing a message vector $\vec m=(m_1,\ldots, m_n)$ such that later,
% anyone, e.g. the untrusted mailboxing service, can compute signatures
% authenticating sub-vectors (i.e., reductions) of $\vec m$. The verifier
% authenticates both the values in the sub-vector and their original positions
% in $\vec m$.\footnote{This distinguishes RS from redactable signatures
% \cite{ACNS:BBDFFK10}.}
%
% We show how to build Reducible Signatures from standard EUF-CMA secure
% signatures and a new type of accumulator called Weighted Accumulators. These
% can be constructed from various assumptions including RSA, lattice-based and
% pairing-based assumptions. However, we introduce a more practically efficient
% construction from a collision resistant hash function.
%
% We believe RS to be of interest in their own right, as they naturally lend
% themselves to a wider class of applications where reducible messages are
% delivered via resources with computational capabilities. One example is
% outsourced storage, where a number of files is uploaded to an untrusted
% cloud. With RS, a data producer can upload a single signature over all files
% such that later, a consumer can efficiently verify authenticity of a couple
% downloaded files. Two other use cases are the E2E protocols used by the Ring
% service~\cite{RingE2E} and the Wickr Messaging Protocol~\cite{Wickr}. In both
% cases, signing uploaded encrypted content with an RS would allow each
% receiver to download just the parts of the header (i.e. the "manifest" in
% Ring parlance) the recipient needs for decryption.\footnote{For a messaging
% application like Wickr which tends to have short plaintexts, redundant data
% in the header can make up the majority of data downloaded.}
%
% \msnote{I added the storage example. Joel, can you add the amazon stuff?}


% \paragraph{Simpler security model.}
% To analyze the security of \saik we introduce a new, greatly simplified,
% security notion for (sa)CGKA which we view as a contribution in its own
% right. Indeed, past work on CGKA has struggled to provide security notions
% for CGKA that are both simple enough to be intuitive yet still meaningfully
% capture the necessary properties. The notion put forth in this work
% omits/simplifies various security features of a CGKA as long as they can be
% easily achieved by known practical extensions of a generic CGKA protocol.
% Thus we obtain a definition focused on the basic properties of a CGKA with
% the idea that a protocol satisfying our notion can be easily extended to a
% ``full-fledged'' CGKA using known techniques.

\paragraph{Performance Evaluation.}
Finally, we provide empirical data comparing the communication complexity for
senders and receivers running various instantiations of \saik and \protITK
for a variety of execution profiles. Our results show that for senders \saik
reduces communication complexity (and halves the number of public key
operations) compared to \protITK. Specifically, for 10K parties, sender
communication complexity decreases from 4.4KB down to 3.6KB in the best case
and from 1.5MB to 0.77MB in the worst case. Meanwhile for receivers the
communication complexity goes from anywhere between logarithmic and even
linear in the group size of \protITK down to at most logarithmic for \saik. Concretely, in a freshly created group with
10K parties a receiver in a \protITK session needs to download 1.38MB to transition into a
new epoch while the same receiver in \saik downloads no more than 2.7KB.

\paragraph{Outline of the paper.}
The paper is structured as follows. \cref{sec:prelims}
(and \cref{sec:addPrelim}) covers basic preliminaries. \cref{sec:mmpke}
focuses on mmPKE while \cref{sec:cgka} describes the new security model for
saCGKA. \cref{sec:saik} describes the \saik protocol.
The intuition for its security (i.e. its safety predicates) are in
\cref{sec:saik-sec-int}. Finally, \cref{sec:eval} contains empirical
evaluation and comparison of \saik to previous constructions.
The formal specification of the (sa)CGKA security
model is in~\cref{sec:model}. \cref{sec:saik_sec} contains the exact
safety predicate, concrete security statement and security proof for \saik.

\subsection{Related Work}

\paragraph{Next generation CGKA protocols.}
The study of next generation CGKA protocols for very large groups was
initiated by Cohn-Gorden et al. in~\cite{CCS:CCGMM18}. This was soon followed
by the first version of TreeKEM~\cite{TreeKEM-original-email} which
evolved to add stronger
security~\cite{TreeKEM-original-email,TreeKEM-with-blanking-email,TreeKEM-tree-signing-email}
and more flexible functionality~\cite{TreeKEM-prop-and-comm-email}
culminating in its current form \protITK{}~\cite{EPRINT:AlwJosMul20}
reflected in the current draft of the MLS RFC~\cite{mls-protocol-latest}.

Reducing the communication complexity of TreeKEM and its descendants is not a
new goal. \emph{Tainted TreeKEM}~\cite{SP:ACC+21} exhibits an alternate
complexity profile optimized for a setting where the group is managed by a
small set of moderators. Recently,~\cite{TCC:AABNKPPW21} introduced new
techniques for `multi-group'' CGKAs (i.e. CGKAs that explicitly accommodate
multiple, possibly intersecting, groups) with better complexity than obtained
by running a ``single-group'' CGKA for each group. Other work has focused on
stronger security notions for CGKA both in theory~\cite{TCC:ACJM20} and with
an eye towards practice~\cite{C:ACDT20,EPRINT:AlwJosMul20}. Supporting more
concurrency has also been a topic of focus as witnessed by the protocols
in~\cite{Eprint:BDR20,TreeKEM-prop-and-comm-email,Wei19}.
Recently~\cite{EPRINT:EKNOO22} present CGKA with novel membership hiding
properties.

\paragraph{Cryptographic models of CGKA security.}
Defining CGKA security in a simple yet meaningful way has proven to be a
serious challenge. Many notions fall short in at least one of the two
following senses. Either they do not capture key guarantees desired (and
designed for) by practitioners (such as providing guarantees to newly joined
members) or they place unrealistic constraints on the adversary. Above all,
they do not consider fully active adversaries. For instance,
in~\cite{SP:ACC+21}, the adversary is not allowed to modify packets while
in~\cite{C:ACDT20,CCS:ACDT21}, new packets can be injected but only when
authenticity can be guaranteed despite past corruptions (thus limiting what
is captured about how session's regain security after corruptions).
Meanwhile, the work of~\cite{Eprint:BCK21} permits a large class of active
attacks but only in the context of the key derivation process of \protITK{}.
So while their adversaries can arbitrarily modify secrets in an honest
party's key derivation procedure, they can not deliver arbitrary packets to
honest parties. This is a significant limitation, e.g., it does not capture
adversaries that deliver packets with ciphertexts for which they do not know
the plaintexts.

Indeed, a good indication that such simplifications can be problematic can be
found in~\cite{EPRINT:AlwJosMul20}. They present an attack on TreeKEM
(that can easily be easily adapted to the CGKAs in the above works except
for~\cite{Eprint:BCK21}) which uses honest group members as decryption oracles
to clearly violate the intuitive security expected of a CGKA. Yet, each of
the above works (except for~\cite{Eprint:BCK21}) proves security of their
CGKA using only IND-CPA secure encryption.

In contrast to the above works,~\cite{TCC:ACJM20} aimed to capture the full
capabilities a realistic adversary might have. Thus, they model a fully
active adversary that can leak parties local states at will and even set
their random coins. In~\cite{EPRINT:AlwJosMul20} this setting is extended to
capture \emph{insider} security. That is adversaries which can additionally
corrupt the PKI. This captures the standard design criterion for deployed E2E
applications that key servers are \emph{not} considered trusted third
parties. Unfortunately, this level of real-world accuracy has resulted in a
(probably somewhat inherently) complicated model.

\paragraph{Symbolic models of CGKA security.}
Complementing the above line of work, several versions of TreeKEM have been
analyzed using a symbolic approach and automated provers
\cite{bhargavan:hal-02425229}. Their models consider fully active attackers
and capture relatively wide ranging security properties which the authors are
able to convincingly tackle by using automated proofs.

\paragraph{Concurrent Work.}
Concurrently and independently to this work~\cite{hashimoto2021cmpke} present
an interesting saCGKA and corresponding adaptation of the insider security
notion of~\cite{EPRINT:AlwJosMul20}. Like our work, they consider a server
that extracts receiver specific sub-packets from a senders full packet.
However, their security notion still enforces representational rather than
semantic agreement on the transcript history. They use a novel
multi-recipient (but not multi-message) PKE which is shown to be
post-quantum secure (though their CGKA is only proven classically secure).

The efficiency of their protocol is incomparable to ours. While with \saik a
receiver's sub-packet contains between $1$ to $\log(n)$ public keys (for
groups of size $n$) in~\cite{hashimoto2021cmpke} a receiver never needs to
download more than a single public key. However, both the computational and
communication complexity for senders is \emph{always} linear in
$n$.\footnote{In fact, this complexity is roughly a half that of a CGKA
construction built from a network of 2-party double-ratchet channels similar
to the sender keys protocol used by WhatsApp.} Meanwhile \saik enjoys
logarithmic fair-weather complexity. That is, sender complexity may vary
between linear down logarithmic (or even constant). In all but the most
pathological executions, other than when a group is freshly created we expect
complexity to be roughly logarithmic for the vast majority of the execution.

%However, because
%symbolic analysis treats data and primitives as ideal objects, the results do
%not capture security to the same depth as more classic analysis can. For
%example, symbolic analysis won't capture attacks leveraging the bit
%representation of an object nor does it elucidate (to the same level of
%detail) what exact security properties are sufficient for the component
%primitives.

% \paragraph{Reducible signatures.}
% The Reducible Signatures introduced in this work are conceptually, relatively
% similar to Redactable Signatures (see e.g.
% \cite{ACNS:BBDFFK10,ACNS:PohSam14,ICISC:DPSS15,EPRINT:HabHorZha16}). The
% latter allow an untrusted censor to remove parts of the signed message
% vector, hopefully hiding the removed parts from the verifier. This secrecy
% property is not a goal of our reducible signatures, hence, our constructions
% are different. See also \cref{rem:rsig}.

\paragraph{mmPKE.}
mmPKE was first proposed by Kurosawa~\cite{PKC:Kurosawa02} though their
security model was flawed as pointed out and fixed by Bellare et.al
\cite{PKC:BelBolSta03,IEEE:BelBolKur07}. Yet, those works too lacked
generality as they demanded malicious receivers know a secret key for their
public key. This restriction was lifted by Poettering et.al.
in~\cite{ASIACCS:PinPoeSch14} who show that well-known PKE schemes such as
ElGamal\cite{C:ElGamal84} and Cramer-Shoup \cite{EC:CraSho02} are secure even
when reusing coins across ciphertexts. Indeed, reusing coins this way can
also reduce the computational complexity of encapsulation and the size of
ciphertexts for KEMs as shown in the Multi-Recipient KEM (mKEM)
of~\cite{SCN:Smart04,ICICS:CLQY18a,AC:KKPP20} for example.
%which yield mmPKE via parallel execution and the KEM/DEM paradigm. Katsumata et.al. \cite{AC:KKPP20}
%claim that their mKEM constructions can be used to increase efficiency in \protITK but do not give a formal proof.
% which, similar to mmPKE, can be used to improve efficiency of \protITK.
% \msnote{I downtoned a bit -- parallel mKEM yield mmPKE but with different
% leakage. It's insecure according to \cite{ASIACCS:PinPoeSch14}.}
%
All previous security notions (for mmPKE and mKEM) allow an adversary to
provide malicious keys (with or without knowing corresponding secret keys),
but none allow for adaptive corruption of honest keys, which is necessary for
\protITK's security against adaptive adversaries.


%%% Local Variables:
%%% mode: latex
%%% TeX-master: "main"
%%% End:
