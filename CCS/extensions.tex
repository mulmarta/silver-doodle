% !TEX root = main.tex
% !TeX spellcheck = en_US

\section{Extensions}
In this section we describe extensions of \saik which we did not include for simplicity. The first extension allows to achieve slightly better security predicates at a relatively small cost. The second extension deals with primitives with imperfect correctness, such as mmPKE based on lattices.

% \subsection{Trading-off Server Computation for Communication}
% Our new (sa)CGKA \saik assumes that the mailboxing service is a full featured server capable of performing the
% individualization of messages responsible for the improved bandwidth of our protocol. This individualization mainly
% consists of generating proofs of accumulation. We assume strong accumulators which allow creation of these proofs
% without secret knowledge. However, we can trade off this server computation for increased sender communication by having the
% sender pre-compute all proofs and send them along with the rest of the packet. Receiver communication is unaffected
% by this change.

% For the hash-based accumulator described in \cref{sec:accumulators}, an easy optimization is possible. Instead of
% computing \emph{all} proofs individually, the sender can instead include the whole Huffman tree in its package. The
% server then only selects the co-paths in the tree relevant for each user, which is similarly complex to selecting the
% correct message for each user, which is exactly the task of a standard mailboxing service. This approach increases
% communication by approximately $2N$ hashes, where $N$ is the number of receiving public keys, i.e. linear in the
% number of group members in the worst case and logarithmic in the best case.

% In conclusion, our saCGKA can be transformed into a regular CGKA at the cost of sender communication while avoiding
% server computation.

% Note that the variant where each submessage is signed individually requires less communication (i.e. $N$ signatures
% instead of $2N$ hashes) than the transformed \saik variant and also doesn't require server computation. However,
% computing each signature requires an expensive public key operation, leading to a shorter sender message, but more computation for the
% sender than required by the server in \saik.

\subsection{Better Security Predicates}\label{sec:ext-sec-predicates}
We sketch the reason why \saik does not achieve the better security predicates and how it can be modified to achieve them.

Roughly, \saik achieves the worse security predicates because of the following attack: Say $\id_s$, the only corrupted party, creates a new epoch $E$ adding a new member $\id$. According to \saik, in this case $\id_s$ fetches from the Authenticated Key Service, AKS, (a type of PKI setup) a public key $\mmpkepk$ for \mmPKE and a verification key $\spk$ for \sigscheme, both registered earlier by $\id$. In epochs after $E$, parties use $\mmpkepk$ to encrypt messages to $\id$ (even before $\id$ actually joins) and $\spk$ to verify messages from $\id$. Now the adversary $\Adv$ can create a fake epoch $E'$ adding $\id$ with the same $\mmpkepk$ and $\spk$. Then, $\id$ joins $E'$ and is corrupted, leaking $\mmpkesk$ and $\ssk$. This allows $\Adv$ to compute the group key in $E$ and inject messages to parties in $E$. However, the expectation is that this is not possible, since no party is corrupted in $E$ (and $\id_s$ healed).
%
The better security predicates (formally, the predicates in \cref{fig:safe} in \cref{sec:bgm_prot_proof}) achieve just this: security in an honest epoch $E$ does not depend on whether some member joins a fake group in $E'$.

The following modification to \saik achieves better security: We note that in \saik, $\id$ registers in the AKS an additional public key $\mmpkepk'$ which is used to send secrets needed for joining. The corresponding $\mmpkesk'$ is deleted immediately after joining. In the modified \saik, when $\id_s$ adds $\id$, it generates for $\id$ new key pairs $(\mmpkepk_s,\mmpkesk_s)$ and $(\spk_s,\ssk_s)$. It sends $\mmpkesk_s$ and $\ssk_s$ to $\id$, encrypted under $\mmpkepk'$. Now messages to $\id$ are encrypted such that \emph{both} $\mmpkesk$ and $\mmpkesk_s$ are needed to decrypt them. In particular, to encrypt $m$, a sender chooses a random $r$ and encrypts $r$ under $\mmpkepk$ and $m \oplus r$ under $\mmpkepk_s$. Similarly, messages from $\id$ have two signatures, one verified under $\spk$, and one under $\spk_s$. As soon as $\id$ creates an epoch, it generates a new single \mmPKE key pair and a single \ers key pair.

The attack is prevented, because even after corrupting $\id$ in $E'$, $\Adv$ does not know $\mmpkesk'$ needed to decrypt $\mmpkesk_s$ and $\ssk_s$. Therefore, confidentiality and authenticity in $E$ is not affected.

\subsection{Primitives with Imperfect Correctness}
While the proofs of \saik security assume primitives with perfect correctness, they can be easily modified to work with
imperfect correctness. This is achieved by adding one game hop where we
abort in the new game if a correctness error occurs. This loses an additive term in the security bound that depends on
the correctness parameter and the number of possible occurrences. Additionally, the usage of primitives with imperfect
correctness generally yields imperfect correctness guarantees for the application as well (potentially with
multiplicative correctness error when using multiple primitives). For completeness, we give definitions of imperfect
correctness of the primitives used directly by \saik in this section.

\begin{definition}
We call an \mmPKE scheme \emph{$\delta$-correct}, if for all $n\in \N$, $(\mmpkepk_i,\mmpkesk_i)\in
  \mmpkeKeyGen $ for $i\in[n]$,
  $(m_1,\ldots, m_n)\in\mathcal{M}^n$ and $\forall j\in[n]$
  \[
    \Pr\left[
      \begin{array}{c}
        c_j\gets \mmpkeExt(j, C)\\
        m_j \neq \mmpkeDec(\mmpkesk_j, c_j)
      \end{array}
      \middle\vert
      C \getsr \mmpkeEnc\left(
      \begin{array}{c}
        (\mmpkepk_1,\ldots, \mmpkepk_n),\\(m_1,\ldots,m_n)
      \end{array}
      \right)
    \right] \leq \delta
  \]
\end{definition}

% \begin{definition}
%   An HRS scheme \ers for a collection of reduction pattern classes $\rdclassset_n$ for $n\in\N$ is $\delta$-correct if for all $n \in \N$, $\rdclass \in \rdclassset_n$, $(\rd,w) \in \rdclass$ and message vectors $\vec m$ of length $n$, we have
%   \begin{equation*}
%     \Pr\left[\ersvrfy\left(
%     \begin{array}{c}
%       \ersvk, k, \rd(\vec{m}),\\ \rd, \sigma'
%     \end{array}
%     \right) \neq 1
%     \middle\vert
%     \begin{array}{c}
%       (\erssk, \ersvk) \gets \erskeygen()\\
%       k \getsr \bits^\kappa\\
%       \sigma \gets \erssign(\erssk, k, \vec{m}, \rdclass) \\
%       \sigma' \gets \ersred(\ersvk, \sigma, \vec{m},\rd)
%     \end{array}
%     \right] \leq \delta.
%   \end{equation*}
% \end{definition}

% Below, we also define $\delta$-correct weighted accumulator. It is easy to see that our construction of \ers instantiated with a $\delta$-correct accumulator is also $\delta$-correct.

% \begin{definition}
%   A weighted accumulator scheme $\wacc$ is \emph{$\delta$-correct} if for every set $X$ of element-weight pairs $(x,w)$ and each $x$ s.t. $(x,\wc)\in X$, we have
%   \[
%   \Pr
%   \left[
%   \accVrfy(\accValue, x, \accProof) \neq 1
%   \middle\vert
%   \begin{array}{c}
%     (\accValue, \accaux) \getsr \accEval(X) \\
%     \accProof \getsr \accProve(\accValue, x, \accaux)
%   \end{array}
%   \right]\leq \delta.
%   \]
% \end{definition}



%%% Local Variables:
%%% mode: latex
%%% TeX-master: "main"
%%% End:
