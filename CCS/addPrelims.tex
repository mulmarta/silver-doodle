\section{Additional Preliminaries} \label{sec:addPrelim}

\subsection{Universal Composability}\label{sec:uc}
We formalize security in the universal composability (UC) framework \cite{FOCS:Canetti01}. We moreover use the modification of responsive environments introduced by Camenisch et
al.~\cite{AC:CEKKR16} to avoid artifacts arising from seemingly local operations (such as sampling randomness or
producing a ciphertext) to involve the adversary.

The UC framework requires a real-world execution of the protocol to be indistinguishable from an ideal world, to an an interactive environment.
%
The real-world experiment consists of the group members executing the protocol (and interacting with the PKI setup).
In the ideal world, on the other hand, the protocol gets replaced by dummy instances that just forward all inputs and outputs to
an \emph{ideal functionality} characterizing the appropriate guarantees.

The functionality interacts with a so-called simulator, that translates the real-world adversary's actions into corresponding ones in the ideal world. Since the ideal functionality is secure by definition, this implies that the real-world execution cannot exhibit any attacks either.


\paragraph{The Corruption Model.}
We use the --- standard for CGKA/SGM but non-standard for UC --- corruption model of continuous state leakage (transient passive corruptions) \cite{TCC:ACJM20}.\footnote{Passive
  corruptions together with full network control allow to emulate active corruptions.}
%
In a nutshell, this corruption model allows the adversary to
repeatedly corrupt parties by sending corruption messages of the form $(\keyword{Expose}, \id)$, which causes the party $\id$ to send its current state to the adversary (once).

\paragraph{Restricted Environments.}
In order to avoid the so-called commitment problem, caused by adaptive corruptions in simulation-based frameworks, we restrict the environment not to corrupt parties at certain times. (This roughly corresponds to ruling out ``trivial attacks'' in game-based definitions. In simulation-based frameworks, such attacks are no longer trivial, but security against them requires strong cryptographic tools and is not achieved by most protocols.)
%In order to avoid the so-called commitment problem\footnote{The commitment problem causes standard UC definitions to be extremely strong in the presence of adaptive corruptions, ruling out any practical protocols such as \mls.}
%of simulation-based security notions such as UC, we restrict the environment not to corrupt parties at certain times. (This roughly corresponds to ruling out ``trivial wins'' in game-based definitions.)
%
To this end, we use the technique used in \cite{TCC:ACJM20} (based on prior work by Backes et al.~\cite{ESORICS:BDDK06} and Jost et al.~\cite{TCC:JosMauMul19}) and consider a weakened variant of UC security that only quantifies over a restricted set of so-called admissible environments that do not exhibit the commitment problem.
%
%Whether an environment is admissible or not is defined by the ideal functionality $\Func$ with statements of the form $\KwRestrEnv\ \mathit{cond}$ and an environment is called {admissible (for $\Func$)}, if it has negligible probability of violating any such $\mathit{cond}$ when interacting with $\Func$.
Whether an environment is admissible or not is defined as part of the ideal functionality $\Func$: The functionality can specify certain boolean conditions, and an environment is then called {admissible (for $\Func$)}, if it has negligible probability of violating any such condition when interacting with $\Func$.

% \subsection{Collision-Resistant Hashing}
% We define collision resistance for hash functions in \cref{def:cr}.
% \begin{definition}[Collision Resistance]\label{def:cr}
%   Let $\hash:\bits^* \rightarrow \bits^\kappa$ be a hash function. We define the advantage of an adversary \Adv against
%   the collision resistance of $\hash$ as
%   \[
%     \adv{\CR}{\hash}(\Adv) = \Pr\left[\hash(x_1) = \hash(x_2)\mid (x_1,x_2)\getsr\Adv\right].
%   \]
% \end{definition}

% Note that for simplicity, we define \emph{unkeyed} hash functions. Generally, these hash functions aren't collision
% resistant, since there always exists some algorithm that has a hard-coded collision. However since such algorithms are
% \emph{unknown} for real-world hash functions and we give constructive reductions (i.e. fully black-box reductions where
% access to an algorithm breaking our building blocks directly yields an algorithm finding a collision), we ignore these
% existing but unknown algorithms. For a more in-depth discussion, see \cite{VIETCRYPT:Rogaway06}.

\subsection{Assumptions}
The security of our mmPKE construction, same as that of \cite{ASIACCS:PinPoeSch14}, is based on a variant of the Computational
Diffie-Hellman(CDH) assumption called the \emph{Double-Sided Strong Diffie-Hellman Assumption} (or just \emph{Static
  Diffie-Hellman Assumption} in \cite{ASIACCS:PinPoeSch14}). We recall it in Definition~\ref{def:SDH}. Intuitively, it states that
CDH is hard given access to a DDH-oracle for both CDH inputs.

\begin{definition}[Double-Sided Strong Diffie-Hellman Assumption]\label{def:SDH}
  Let $\Group = (\Grp, p, g)$ be a cyclic group of prime order $p$ with generator $g$. We define the advantage of an
  algorithm $\Adv$ in solving the \emph{Double-Sided Strong Diffie-Hellman problem(\sdh)} with respect to $\Group$ as
  \[
    \adv{\sdh}{\Group}(\Adv) =
    \Pr\left[
      Z = g^{xy}
      \middle\vert
      \begin{array}{c}
        x, y \getsr \Z_p^2\\
        Z \getsr \Adv^{\oracle{O}}(\Grp,p,g,g^x,g^y),
      \end{array}
    \right]
  \]
with $\oracle{O} = \{\oracle{O}_x(\cdot,\cdot),\oracle{O}_y(\cdot,\cdot)\}$, where $\oracle{O}_x,\oracle{O}_y$ are oracles which on input $U,V$ output $1$, iff $U^x = V$ or $U^y = V$ respectively.
The probability is taken over the random coins of the group generator, the choice of $x$ and $y$ and the adversary's
random coins.
\end{definition}

\subsection{Multi-Recipient Multi-Message PKE(mmPKE) Definitions}\label{app:mmpke}
The notion of \mmindrcca security for \mmPKE is described by the experiment in \cref{fig:mmpke_rcca}.

\begin{figure}[!tbp]
  \begin{gamebox}{\mmindrcca}
    \begin{minipage}[t]{\linewidth}
      \algoHead{$\text{Exp}^{\mmindrcca}_{\mmPKE, \nUsers, b}(\Adv = (\Adv[1], \Adv[2]))$}
      \begin{algorithmic}
        \For{$\idxUser\in[\nUsers]$}
           $(\mmpkepk_\idxUser, \mmpkesk_\idxUser)\gets \mmpkeKeyGen()$
        \EndFor
        \State $\term{Corr} \gets \emptyset$
        \State \mbox{$(\vec\mmpkepk^*,\vec m_0^*,\vec m_1^*,\mathit{st}) \gets \Adv[1]^{\oracle{Dec}_1, \oracle{Cor}}((\mmpkepk_i)_{i\in[\nUsers]})$}

        \State \KwReq{} $\abs{\vec m_0^*}=\abs{\vec m_1^*}=\abs{\vec\mmpkepk^*}$
        \State $c^* \getsr \mmpkeEnc(\vec\mmpkepk^*,\vec{m}_b^*)$

        \State $b' \gets \Adv[2]^{\oracle{Dec}_2, \oracle{Cor}}(c^*,\mathit{st})$
        \State \KwReq{} $\leak(\vec{m_0}) = \leak(\vec{m_1})$
        \State \KwReq{} $\forall j : \vec \mmpkepk^*[j] \in \{\mmpkepk_i : i\in[\nUsers]\} \setminus \term{Corr} \lor m_0^*[j]=m_1^*[j]$
        \State \Return $b'$
      \end{algorithmic}
    \end{minipage}

  \medskip
    \begin{minipage}[t]{.4\linewidth}
      \algoHead{Oracle $\oracle{Dec}_1(i,c)$}
      \begin{algorithmic}
        \State \KwReq{} $\idxUser\in[\nUsers]$
        \State \Return $\mmpkeDec(\vec\mmpkesk[i],c)$
      \end{algorithmic}

      \medskip
      \algoHead{Oracle $\oracle{Cor}(\idxUser)$}
      \begin{algorithmic}
        \State \KwReq{} $\idxUser\in[\nUsers]$
        \State $\term{Corr} \setadd \idxUser$
        \State \Return $\mmpkesk_\idxUser$
      \end{algorithmic}
    \end{minipage}\hfill\begin{minipage}[t]{.59\linewidth}
      \algoHead{Oracle $\oracle{Dec}_2(i,c)$}
      \begin{algorithmic}
        \State \KwReq{} $\idxUser\in[\nUsers]$
        \State $m \gets \mmpkeDec(\vec\mmpkesk[i],c)$
        \If{$\exists j : \vec\mmpkepk^*[j] = \mmpkepk_i $ \\\strut\hfill $\land\ m \in \{\vec m_0^*[j], \vec m_1^*[j]\}$}
        \State \Return $\literal{test}$
        \Else\
        \Return $m$
        \EndIf
      \end{algorithmic}
    \end{minipage}
  \end{gamebox}
  \caption{\mmindrcca security game for \mmPKE with leakage function  $\leak(\vec m) = (\len(\vec m[1]), \dots, \len(\vec m[n]))$.}
  \label{fig:mmpke_rcca}
\end{figure}

We define the security of an \mmPKE in a left-right style in the following Definition~\ref{def:mmindrcca}.

\begin{definition}[\mmindrcca]\label{def:mmindrcca}
Let $N\in\N$. For a scheme \mmPKE, we define the advantage of an adversary \Adv against \emph{Indistinguishability Against Replayable Chosen Ciphertext Attacks (\mmindrcca)} security of \mmPKE as
\begin{multline*}
  \adv{\mmindrcca}{\mmPKE, \nUsers}(\Adv) = \Pr\left[\Exp^{\mmindrcca}_{\mmPKE,\nUsers, 0}(\Adv) = 1\right] \\ -
      \Pr\left[\Exp^{\mmindrcca}_{\mmPKE,\nUsers,1}(\Adv) = 1\right],
\end{multline*}
where $\textnormal{Exp}^{\mmindrcca}_{\mmPKE,N,b}$ is described in \cref{fig:mmpke_rcca}.
\end{definition}


\subsection{Data Encapsulation Meachanism(DEM)}
A DEM is the symmetric equivalent of a PKE scheme. We recall it in Definition \ref{def:dem}.

\begin{definition}[DEM]\label{def:dem}
  A data encapsulation mechanism (DEM) \dem is described by a (efficiently samplable) keyspace $\demKeyset$ and the two
  algorithms $\demEnc, \demDec$:
  \begin{itemize}[align=left]
    \item[$\demEnc(k, m)\getsl c$:] The encryption algorithm takes a key $k\in\demKeyset$ and a message $m$. It returns a ciphertext $c$.
    \item[$\demDec(k, c) \getsl m'\lor\bot$:] The decryption algorithm takes a key $k\in\demKeyset$ and a
      ciphertext $c$ and outputs either a decrypted message or $\bot$.
    \end{itemize}
    A DEM \dem is $\delta$-correct, if for all messages $m$ and all keys $k\in\demKeyset$
    \[
      \text{Pr}[\demDec(k, \demEnc(k,m)) = m] \geq \delta
    \]
  \end{definition}

  Analogue to mmPKE, we consider \indrcca security for DEMs. It is described in Definition~\ref{def:dem_rcca}.

  \begin{definition}\label{def:dem_rcca}
    The advantage of an adversary \Adv against the \indrcca security of a DEM \dem is defined as
    \begin{multline*}
      \adv{\indrcca}{\dem}(\Adv) = \text{Pr}[\experiment{\indrcca}{\dem, 0}(\Adv) = 1] \\-
      \text{Pr}[\experiment{\indrcca}{\dem, 1}(\Adv) = 1],
    \end{multline*}
    where $\experiment{\indrcca}{\dem, b}(\Adv)$ is defined in \cref{fig:dem_rcca}.
  \end{definition}

  \begin{figure}[!tbh]
    \begin{gamebox}{\indrcca for \dem}
      \begin{minipage}[t]{.5\linewidth}
        \algoHead{Exp$^{\indrcca}_{\dem, b}(\Adv)$}
        \begin{algorithmic}
          \State $k\getsr \demKeyset$
          \State $(St, m_0^*, m_1^*) \getsr \Adv^{\text{Dec},\text{Enc}}$
          \State $c^* \getsr \demEnc(k, m_b)$
          \State \Return $\Adv^{\text{Dec},\text{Enc}}(St, c^*)$
        \end{algorithmic}
      \end{minipage}
    \hfill
      \begin{minipage}[t]{.48\linewidth}
        \algoHead{Oracle Dec$(c)$}
        \begin{algorithmic}
          \State $m' \getsr \demDec(k, c)$
          \If{$m' \in \{m_0^*, m_1^*\}$}
          \State \Return \keyword{test}
          \Else
          \State \Return $m'$
          \EndIf
        \end{algorithmic}
        \medskip
        \algoHead{Oracle Enc$(m)$}
        \begin{algorithmic}
          \State \Return $\demEnc(k, m)$
        \end{algorithmic}
      \end{minipage}
    \end{gamebox}
    \caption{\indrcca security for DEMs.}
    \label{fig:dem_rcca}
  \end{figure}

\subsection{Message Authentication Codes (MAC)}
Message authentication codes are defined in Definition~\ref{def:mac}.

\begin{definition}\label{def:mac}
  A message authentication code $\mac = (\mactag,$ $ \macvrf)$ consist of a keyspace $\mackeyspace$ and the following two
  algorithms:
  \begin{itemize}[align=left]
    \item[$\mactag(k, m)\getsl \mtag$:] The tagging algorithm takes a key $k$ and a message $m$ and outputs
      a tag $t$.
    \item[$\macvrf(k, m, \mtag)\getsl\bits$:] The verification algorithm takes a key $k$, a message $m$ and
      a tag $\mtag$ and outputs $0$ or $1$.
    \end{itemize}
    A mac \mac is correct, if for all $k\in\mackeyspace$ and messages $m$
    \[
      \text{Pr}[\macvrf(k, m, \mactag(k,m)) = 1] = 1
    \]
\end{definition}

The security notion for MACs we consider is \emph{Unforgeability against chosen message attacks}(\ufcma).
\begin{definition}
  A mac \mac is \ufcma secure, if for all PPT adversaries \Adv the advantage
  \begin{multline*}
    \adv{\ufcma}{\mac}(\Adv) = \\\Pr
    \left[
      \begin{array}{c}
        m \not \in Q \land\\
        \macvrf(k, m^*, \mtag^*) = 1
      \end{array}
      \middle\vert
      \begin{array}{c}
        k\getsr \mackeyspace\\
        (m^*, t^*) \getsr \Adv^{\text{Tag}, \text{Ver}}
      \end{array}
    \right]
  \end{multline*}
  is negligible, where the Tag oracle computes a tag under key $k$ on a given message $m$ and adds it to $Q$ and
  \emph{Ver} takes a message and a tag and outputs the result of the \macvrf algorithm on the two inputs with $k$.
\end{definition}



%%% Local Variables:
%%% mode: latex
%%% TeX-master: "main"
%%% End:
